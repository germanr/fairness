	\section{Introduction}


Several theoretical and empirical papers study how objective measures of income inequality affect individual-level behavior.\footnote{Researchers have studied how income inequality affects cooperation \citep{cozzolino2011trust}, demand for redistribution \citep{meltzer1981rational, finseraas2009income}, dishonesty \citep{neville2012economic}, social cohesion \citep{alesina1996income}, subjective wellbeing \citep{oishi2011income}, and trust \citep{gustavsson2008inequality}.} Implicit in much of this literature is the assumption that inequality shapes individuals' beliefs about whether the income distribution is fair (``fairness views,'' for short).\footnote{For example, one important reason why social cohesion might be related to income inequality is that, as inequality increases, more individuals perceive the income distribution as unfair, making them more prone to mobilize. Similarly, we would not expect inequality to be negatively linked to subjective wellbeing if increases in income disparities were perceived as fair.} However, two pieces of evidence suggest that the link between inequality and fairness views is not straightforward. First, fairness views are not informed by objective measures of inequality---since these are not directly observable by individuals---but instead by perceived inequality. Research about how accurately people perceive income inequality shows large gaps between individuals' perceptions and the actual levels of inequality \citep{norton2011building,kuziemko2015elastic, gimpelson2018misperceiving, choi2019revisiting}. Second, even absent any misperceptions, individuals do not consider all inequities to be unfair. Specifically, individuals largely accept income disparities derived from personal choices and effort, but deem inequities driven by luck or chance as unfair \citep{cappelen2007pluralism, alesina2011preferences, cappelen2013just, almaas2020cutthroat}. Thus, the extent to which fairness perceptions are shaped by income inequality remains an important empirical question.

In this paper, we empirically study the link between fairness views and income inequality in a particular scenario: a region of highly unequal countries---Latin America---but during a period in which inequality pronouncedly declined. First, we assess the extent to which fairness views are linked to income inequality, both across countries and over time. Then, we analyze how individual-level factors such as education, political ideology, and religious views relate to fairness views. Finally, we investigate the predictive power of fairness views for individuals' propensity to protest and mobilize.

In Section \ref{sec_context_data}, we describe the institutional context and our data. Our setting is Latin America, one of the most unequal regions in the world \citep{alvaredo2015recent}. We focus on the 2000s, an unusual period in that income inequality saw a widespread decrease across countries in the region \citep{gasparini2011recent}. To relate income inequality to fairness views, we combine data from two harmonization projects. Our source for income inequality data is SEDLAC, a project that increases cross-country comparability of household surveys. These data enable us to compare the evolution of income inequality across countries and over time. The data on fairness perceptions come from public opinion polls conducted by Latinobar\'ometro in 18 Latin American countries since the 1990s. 

In Section \ref{sec_ineq_fairness}, we document a series of stylized facts about fairness views in the region. A strikingly high, albeit decreasing, share of the population believes that the income distribution in their country is unfair. In 2002, almost nine out of ten individuals perceived the income distribution as unfair (86.6\%). By 2015, this figure declined to 75.1\%. This decline is particularly striking considering that previous research highlights people's tendency to report perceptions of stable or increasing inequality, regardless of its actual evolution \citep{gimpelson2018misperceiving}. %The reduction in unfairness perceptions was mainly driven by a decline in the share of individuals who hold strong unfairness beliefs, i.e., individuals who believe that the income distribution is very unfair as opposed to merely unfair. 

Next, we link fairness views to income inequality. We study how the Gini coefficient---our main measure of inequality---correlates with fairness views across countries and over time. We find a strong linear correlation between the Gini coefficient and the percent of the population that perceives inequality as unfair across country-years. Fairness views evolved in the same direction as the Gini in 17 out of the 18 countries in our sample during the 2000s. 

The decline in income inequality during the 2000s---although remarkable by historical standards---was not enough to substantially modify the view of Latin Americans with respect to distributive fairness in their societies. Three out of four citizens of the region still believe that the income distribution is unfair. A one percentage point decrease in the Gini is associated with a 1.4 percentage point decrease in the share of the population perceiving the income distribution as unfair. Holding constant this elasticity and the pace of inequality reduction of the 2000s, reducing the population that perceives inequality as unfair to 50\% would take roughly more than a decade.

Next, we investigate whether inequality measures other than the Gini coefficient are stronger predictors of unfairness beliefs. This question is of interest in its own right, given the ongoing debate on whether income inequality should be measured with relative or absolute indicators \citep{ravallion2003debate, atkinson2010analyzing}. We take an agnostic approach and correlate a large number of relative and absolute measures of inequality with unfairness views. We find that relative indicators are strongly positively correlated with people's perceptions of unfairness. In contrast, absolute indicators tend to be \textit{negatively} correlated with unfairness views. This is because absolute income gaps became wider in Latin America in the booming years under analysis, yet perceptions about unfairness went down.

In Section \ref{sec_corr_fairness}, we assess whether the correlation between income inequality and fairness views is robust to controlling for observable variables and investigate which individual-level characteristics are predictive of fairness views. The relationship between unfairness views and the Gini coefficient is positive and statistically significant even after controlling for country fixed effects, year fixed effects, and a large set of individual-level characteristics. We find that older, unemployed, and left-wing individuals are more likely to perceive income distribution as very unfair. A decomposition exercise shows that the decline in unfairness perceptions during the 2000s is better accounted for by income inequality trends rather than changes in the composition of the population.

In Section \ref{sec_unrest}, we analyze the link between fairness perceptions and individuals' self-reported propensity to protest. A vast literature relates income inequality to social cohesion, conflict, and activism. One might expect this link to be partly mediated by fairness views. Hence, we study whether fairness views have predictive power for social unrest, conditional on income inequality. To do this, we measure individuals' likelihood of participating in different political activities, such as mobilizing in a demonstration, signing a petition, refusing to pay taxes, or complaining on social media. Participation in these activities is self-reported, so the results should be interpreted with caution. With this caveat in mind, we find that both fairness views and inequality have predictive power independent of each other for some political activities, such as complaining on social media. Other behaviors are exclusively predicted by fairness views (e.g., signing a petition) or by income inequality (e.g., refusing to pay taxes). This suggests that both fairness views and income inequality are important determinants of the propensity to engage in political activism.

This paper contributes to the literature that links objective measures of the income distribution to individuals' perceptions of such measures. Previous papers have shown that individuals tend to misperceive their relative incomes \citep{cruces2013biased, karadja2017richer, hvidberg2020social, fehr2021your} and other relevant features of the income distribution, such as the level of inequality, poverty, and mobility \citep{kuziemko2015elastic, page2016subjective, alesina2018intergenerational, preuss2020}.\footnote{Mismatches between beliefs and reality are important because there is mounting evidence that perceptions of facts, more than facts themselves, affect individual behavior. For example, demand for redistribution is affected more by perceptions of the income distribution than by the actual distribution \citep{gimpelson2018misperceiving, choi2019revisiting}. Thus, understanding what people believe about the income distribution is important from a policy perspective. If there are mismatches between perceptions and reality, interventions that make information less costly or more salient might be desirable.} Evidence on the relationship between fairness perceptions and income inequality, particularly in Latin America, is rather scarce. To our knowledge, the only other paper that studies the link between inequality and fairness views in the region is \cite{zmerli2015income}, although the focus of this paper is on political trust. Using the same data that we use, the authors find a positive association between unfairness views and the Gini. However, the authors only use data from one year. Hence, they cannot study the joint evolution of both variables over time or control for unobserved heterogeneity at the country or year level, which, as we argue below, could generate a spurious correlation between income inequality and fairness views. We contribute to this literature by providing novel empirical evidence linking fairness views to income inequality in a highly unequal region, but during a period of falling inequality.\footnote{A related literature exploits opinion surveys to study distributive issues in Latin America. \cite{cepal2010america} documents patterns of perceptions of distributive inequity during 1997--2007. Using data from Argentina, \cite{rodriguez2014percepciones} shows that people who consider their income to be fair tend to perceive lower levels of inequality. \cite{martinez2020latin} explore the effect of immigration on preferences for redistribution in Latin America.}

This paper also contributes to the literature on inequality measurement. This literature makes a crucial distinction between two types of inequality indicators: the relative ones (such as the Gini coefficient) and absolute ones (such as the variance). Relative and absolute indicators often provide different answers to important issues such as the distributive effects of globalization or trade openness \citep{ravallion2003debate, atkinson2010analyzing}. Hence, it is important to understand whether people think about distributive fairness through the lens of relative or absolute indicators. We show that relative indicators have a much stronger correlation with fairness views than absolute indicators.

Finally, we make a small contribution to the growing literature that relates income inequality---and more recently, measures of polarization---to conflict and political activism \citep{esteban2011linking, esteban2012ethnicity}. Previous papers have shown that income inequality is predictive of conflict and social unrest. We contribute to this literature by showing that fairness views have predictive power for social unrest above and beyond income inequality (and vice-versa).

%The rest of the paper is organized as follows. In section \ref{sec_context_data}, we provide institutional background on Latin America, describe the data, and provide descriptive statistics on our sample. In section \ref{sec_ineq_fairness}, we establish a series of stylized facts about the relationship between income inequality and fairness views. In section \ref{sec_corr_fairness}, we analyze the individual-level determinants of fairness views. In section \ref{sec_unrest}, we link fairness perceptions to social unrest. Section \ref{sec_conclusions} concludes.


