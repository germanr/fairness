\section{The Predictive Power of Fairness Views for Social Unrest} \label{sec_unrest}

There is a vast literature that relates economic inequality---and more recently, measures of polarization---to social cohesion, conflict, and activism.\footnote{For instance, \cite{gasparini2008income} find a strong empirical correlation between inequality and conflict in Latin America. Most previous studies linking inequality and conflict are based on cross-country regressions, and therefore have a notably smaller sample size than our paper.} Arguably, the relationship between income inequality and conflict is partly mediated by fairness views. That is, many individuals mobilize in part because they believe existing inequities are unfair. However, a given level of income inequality might not be seen as unfair by some individuals due to, for example, misperceptions of the actual level of inequality \citep{gimpelson2018misperceiving} or a perception that income gaps are mainly driven by differences in effort \citep{alesina2001doesn}. For these reasons, a regression that links social unrest to income inequality can contain substantial measurement error. We sidetrack these issues by directly measuring the link between social unrest and fairness views.

We measure propensity to engage in social unrest using the opinion polls data. For several political activities, Latinobar\'ometro asks respondents whether they (i) have ever done the activity; (ii) would do the activity; or (iii) would never do the activity. We investigate eight different types of demonstrations: making a complaint on social media, making a complaint to the media, signing a petition, protesting with authorization, protesting without authorization, refusing to pay taxes, participating in riots, and occupying land, factories or buildings. We also create a composite index of political participation which takes the value one if the individual engaged in tax evasion, an illegal protest, signed a petition, or complained to the media, and zero otherwise.\footnote{We use those four measures to construct the index because we do not have data on the other political activities during 2015. Other years have data on fewer political activities.} Participation in these activities is self-reported. Given that respondents had no financial incentives for truth-telling, our results should be taken with caution.

For each activity (and the index), we consider two measures of social unrest. First, we use an indicator that takes the value one if an individual says she has done the activity in the past and zero otherwise. Second, we use an indicator that takes the value one if the individual did the activity in the past \textit{or} says that she is willing to do the activity. We use these measures as dependent variables in Logit regressions. The regressions control for unfairness perceptions, the Gini, and individual-level covariates. Unfortunately, participation in political activities is available only in a few years, so our sample size for these regressions is substantially smaller.

Table \ref{very_unfair_unrest_past}, Panel A shows that unfairness perceptions correlate with participating in political activities in the past. We find positive and statistically significant effects for complaining on social media (column 1) and signing a petition (column 3). Conditional on income inequality, individuals who perceive the income distribution as very unfair are 1.6 percentage points more likely to have complained through social media in the past (from a baseline of 7.8\%) and 1.3 percentage points more likely to have signed a petition in the past (from a baseline of 18.6\%). The rest of the effects tend to be positive, although not statistically different from zero. Conditional on fairness views, we find a statistically significant correlation between the Gini and complaining on social media (column 1), taking part in an authorized demonstration (column 5), and the composite index (column 9). The effect of income inequality on the rest of the activities is statistically indistinguishable from zero. 

Table \ref{very_unfair_unrest_past}, Panel B shows the results when the dependent variable also includes the willingness to participate in the political activities. The set of political activities predicted by fairness views are somewhat different than in Panel A. In Panel B, we find positive and statistically significant effects for complaining through media (either social media or traditional media, columns 1 and 2, respectively) and refusing to pay taxes (column 6). The magnitude of the coefficients that are statistically significant tends to be larger than in the baseline specification. For example, the effect of unfairness views on the propensity to complain on social media is twice as large in Panel B than in Panel A (3.3 vs. 1.6 percentage points, correspondingly). Finally, we find that---holding fairness views constant---income inequality is predictive of refusing to pay taxes (column 4). The effect of income inequality on the rest of the political activities is not statistically different from zero. 

Taken together, these results show that there are political activities for which fairness views and income inequality have predictive power independent of each other (like complaining through social media). However, there are also activities that are exclusively predicted by income inequality (like participating in an unauthorized protest) or fairness views (like signing a petition). This suggests that both fairness views and income inequality capture different channels through which changes in the income distribution can affect social unrest.
