\section{Conclusions} \label{sec_conclusions}

In this paper, we analyze perceptions of distributive justice in a context of falling income inequality. We show that fairness beliefs moved in line with the evolution of objective inequality indicators: both unfairness perceptions and income inequality declined across countries and over time in our sample. Some individual-level characteristics, such as employment status and political ideology, are systematically correlated with fairness views. Fairness views have predictive power for self-reported propensity to mobilize above and beyond income inequality (and vice-versa).

Our findings are relevant for both researchers and policymakers. For researchers, our results suggest that, in some contexts, one can proxy fairness views using relative measures of income inequality, such as the widely used Gini coefficient. This is reassuring since inequality measures are much more widely available than fairness views in standard datasets. 

For policymakers, our findings warn about concerning levels of dissatisfaction with existing income disparities. Three in four Latin Americans believe that the income distribution is unfair, and such perceptions have proved to be relatively inelastic to a large compression of the income distribution by historical standards. If fairness perceptions are interpreted as preferences for some leveling of income, our results indicate that a striking majority is in favor of reducing the existing disparities between the rich and the poor, while relatively few people believe that income disparities should remain the same. A second actionable insight for policymakers is that fairness views act as a thermometer of individuals' latent propensity to engage in political activities. Thus, if policymakers want to prevent social unrest, they ought to pay attention to the evolution of fairness views and take preventive measures before the majority of people perceive the income distribution as unfair.

A caveat with our results is that we cannot tell whether most individuals believe that the income distribution is unfair because (i) they have inaccurate views about the level of income inequality (perhaps, believing that income is more unequally distributed than it objectively is); or (ii) individuals accurately assess the level of inequality and believe that existing inequities are unfair (perhaps, because the process that generates income differences is not fair or because existing inequities are too large). Disentangling the contribution of these and other channels is a challenge for future research.
