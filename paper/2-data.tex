\section{Institutional Context, Data, and Descriptive Statistics} \label{sec_context_data}

This section provides institutional context on Latin America, describes our data, and provides summary statistics on our sample.

\subsection{Context} \label{subsec_context}

Latin America has long been characterized as a region with high levels of income inequality. Together with South-Saharan Africa, Latin America is one of the two most unequal regions in the world \citep{alvaredo2015recent,world2016poverty}. After a period of increasing inequality during the 1980s and 1990s, the region experienced a ``turning point'' in the 2000s, when income inequality saw a widespread decrease \citep{gasparini2011recent, gasparini2011rise, lustig2013declining}.\footnote{The widespread decline in inequality contrasts to what happened in other developing regions in the world, where inequality modestly decreased (e.g., such as in the Middle East and North Africa), or even increased (such as in East Asia and Pacific, \citealp[c.f.][]{alvaredo2015recent}). In developed countries, inequality tended to increase \citep{atkinson2011top}.}

\subsection{Data} \label{subsec_data}

We use data on fairness views and income inequality from 18 Latin American countries from 1997--2015. The data comes from two harmonization projects, known as Latinobar\'ometro and SEDLAC (Socio-Economic Database for Latin America and the Caribbean).

We use public opinion polls conducted by Latinobar\'ometro to measure fairness perceptions. Latinobar\'ometro conducts opinion surveys in Latin American countries, interviewing about 1,200 individuals per country. The survey is designed to be representative of the voting-age population at the national level (in most Latin American countries, individuals aged over 18).\footnote{In Appendix Table \ref{tab-coverage} we show the percentage of the voting-age population represented by the opinion polls in our sample for the years in which fairness data is available.} The main variable for our empirical analysis is individuals' fairness views, which we measure using the following question: \textit{``How fair do you think the income distribution is in [country]? Very fair, fair, unfair or very unfair?''} We construct binary variables indicating whether individuals believe that the income distribution is unfair or very unfair.\footnote{Latinobar\'ometro does not ask respondent \textit{why} they believe that the income distribution is unfair. It is possible that some people view the distribution as unfair because  existing disparities are not sufficiently large. We think that this is unlikely and interpret unfairness views as reflective of too much inequality.}

Income inequality data comes from SEDLAC, a joint project between CEDLAS-UNLP and The World Bank, which increases cross-country comparability from official household surveys. We measure inequality in household income per capita (measured in 2005 USD at purchasing power parity). Whenever possible, we use comparable annual household surveys to calculate inequality indicators. However, some countries do not conduct surveys every year, and some of the household surveys available in a given country are not comparable over time (usually, due to important methodological changes). In Appendix \ref{sec_data} we describe the partial fixes we implement to maximize the sample size. In two countries of our sample (Argentina and Uruguay), the household survey is representative of urban areas only.


\subsection{Sample and Summary Statistics} \label{subsec_sample}

For our regression analysis, we use individual-level data. Our sample includes all individuals in the 11 different waves of Latinobar\'ometro surveys over 1997--2015.

Appendix Tables \ref{tab-sum-stats} and \ref{tab-fairness-grp} show descriptive statistics on our sample. The average respondent is 39.7 years old. Roughly half of the respondents are men (49\%), over half (56.3\%) are married or in a civil union, and 68\% are Catholics. The majority of respondents (76\%) completed at least elementary school, while a third of them (33.6\%) had completed high school or more. Almost two-thirds of the sample (64\%) were part of the labor force, and 9.9\% of the economically active individuals were unemployed.\footnote{In Appendix \ref{app_lb_sedlac}, we show that Latinobar\'ometro's sample is similar to the sample in SEDLAC.} %Access to basic services among respondents is relatively high: 88.8\% of individuals had access to running water inside their dwelling, and over two-thirds (69.6\%) had access to a sewerage. %Ownership of durable goods ranges from relatively low levels regarding cars and computers (28.2\% and 33.8\%, respectively) to relatively high levels regarding fridges and mobile phones (79.2\% and 80.6\%).


