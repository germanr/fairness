\section{The Relationship between Fairness Views and Inequality} \label{sec_ineq_fairness}

This section first provides descriptive evidence on the evolution of fairness views in Latin America from 1997 to 2015. Then, we link fairness perceptions to income inequality both across countries and over time. Finally, we investigate whether absolute or relative inequality indicators have a stronger predictive power for fairness views.

\subsection{The Evolution of Fairness Views in Latin America during the 2000s}

Figure \ref{fig-fairness-views} shows how fairness views evolved over time (Panel A) and across countries (Panel B). Panel A plots the fraction of individuals who believe that the income distribution of their country is very unfair, unfair, fair, or very fair over 1997--2015, pooling across all countries in our sample. Panel B shows the fraction of individuals who believe that the income distribution is either unfair or very unfair in each country of our sample during 2002 and 2015.

Figure \ref{fig-fairness-views}, Panel A shows that a strikingly high, albeit decreasing, share of the population believes that the distribution of income is unfair. In 2002, almost nine out of ten individuals perceived the income distribution as unfair (86.6\%). By 2015, this figure declined to 75.1\%. The decrease in unfairness perceptions was driven mainly by individuals with strong beliefs about unfairness (i.e., individuals who believe that the income distribution was very unfair). While in 2002, 33.1\% of the population perceived the income distribution as very unfair, this figure declined to 25.8\% by 2015. In contrast, weak beliefs about unfairness (i.e., individuals who believe that the income distribution was merely unfair) behaved more erratically, increasing at the beginning of our sample and decreasing by the end of the period. Overall, the share of individuals with weak beliefs about unfairness slightly declined during the 2000s, from 53.5\% in 2002 to 49.2\% in 2015. On the other hand, the share of the population that believe that the income distribution is fair doubled from 11.3\% in 2001 to 22.6\% in 2015.% while strong beliefs about fairness (i.e., ``very fair'') remained below 5\% throughout the 2000s. 

Figure \ref{fig-fairness-views}, Panel B shows that, while most individuals perceive the income distribution as unfair, fairness perceptions improved in most countries during 2002--2013. A substantial share of the population in all countries perceived the income distribution as unfair in both 2002 and 2013. For example, in 2002, the share of the population that perceived the distribution as unfair ranged from 74.5\% in Venezuela to 97.7\% in Argentina (which, at the time, was in the midst of a severe economic crisis). Throughout the following decade, there was a widespread decrease in the share of the population that perceived income inequality as unfair or very unfair. Compared to 2002, in 2013, a lower fraction of the population perceived the income distribution as unfair in 16 out of the 18 countries in our sample. The change in fairness perceptions ranged from modest decreases, like in Chile, where the decline was of less than one percentage point, to remarkable reductions, like in Ecuador, where perceptions about unfairness declined from 87.5\% to 38.6\%.

Appendix Figure \ref{fig-fairness-grps} shows that the decline in unfairness perceptions was widespread across heterogeneous groups of the population. To show this, we study how fairness views evolved for different subgroups of the population, according to individuals’ age, gender, education, and employment. This analysis reveals that young individuals are less likely to perceive the income distribution as unfair (Panel A), while females are more likely to do so, although with some heterogeneity across time (Panel B). Similarly, individuals with a higher educational achievement (Panel C) and the unemployed (Panel D) are more likely to believe that the income distribution is unfair. Importantly, the perception of unfairness consistently fell across all these subpopulations during the 2000s.

Next, we explore the extent to which these changes in fairness views were accompanied by changes in the actual distribution of income.

\subsection{Fairness Perceptions and Income Inequality: Some Stylized Facts}

Figure \ref{fig-fairness-gini} shows how fairness views evolved vis-\`a-vis changes in income inequality. Panel A shows a binned scatterplot of the Gini coefficients and unfairness views for all country-years in our sample. Panel B plots the percentage point change in unfairness perceptions between 2002 and 2013 on the $y$-axis against the change in the Gini over the same period on the $x$-axis. 

Panel A shows that unfairness perceptions and income inequality are strongly correlated across country-years. The linear correlation between the Gini and unfairness perceptions across country-years is $0.93$ ($p < 0.01$). The share of the population who perceive income as unfair or very unfair ranges from 63\% in country-years with a Gini in the 0.40 bin (roughly, the average level of inequality in Venezuela during the 2000s), to 88\% in country-years with a Gini in the 0.60 bin (roughly, the level of inequality in Honduras during the early 2000s).\footnote{An OLS regression of unfairness views on the Gini estimated on the plotted points yields an intercept of 28.2. This implies that, even in a society where all incomes are equalized, about 28\% of the population would still perceive the income distribution as unfair. This exercise relies on the strong assumption that the relationship between fairness views and income inequality is linear. While such a relationship indeed appears to be linear in Panel A, our data only covers a very narrow range of Gini coefficients (between 0.40 and 0.60). It is likely that the relationship is non-linear for Gini coefficients close to zero or one.} The correlation between unfairness views and the Gini is driven by individuals who perceive inequality as very unfair. The correlation between perceptions of a very unfair distribution and the Gini is sizable and statistically significant. In contrast, the correlation between perceptions of a merely unfair distribution and the Gini is small and indistinguishable from zero.

Panel B shows that the evolution of fairness views tends to mirror the evolution of income inequality at country level. Fairness views moved in the same direction as the Gini in 17 out of the 18 countries in our sample. The one exception is Honduras, where, despite falling inequality, the population perceived the distribution as more unjust. Most countries lie in the third quadrant, where both the Gini and unfairness perceptions decreased. The only country where inequality increased (Costa Rica), also saw an increase in unfairness beliefs. In Appendix Figure \ref{fig-timeseries-gini-unfair} we show that the correlation between the Gini and unfairness views over time is also strong when pooling across countries. In this case, the linear correlation is equal to 0.80 ($p < 0.01$). During 2002--2013, a one percentage point decrease in the Gini was associated with a 1.4 percentage point decrease in the share of the population perceiving the distribution as unfair. To put this figure in context, this means that, at the pace of inequality reduction of the 2000s, it would roughly take Latin America more than a decade to reduce the population that perceives income inequality as unfair to 50\%.


%%%%%%%%%%%%%%%%%%%%%%%	
\subsection{Is Fairness Absolute or Relative?} \label{sec_abs_fairness}
%%%%%%%%%%%%%%%%%%%%%%%

%We have shown that a large, albeit decreasing, share of the population believes that income distribution is unfair and that such levels and evolution are consistent with a high but also declining Gini coefficient. Despite being the most widely used indicator to measure income inequality, fairness perceptions might be better captured with indices other than the Gini coefficient. We next explore this possibility.

The literature on inequality measurement makes a crucial distinction between two types of indicators: relative ones (such as the Gini) and absolute ones (such as the variance). Relative indicators fulfill the scale-invariant axiom, while the absolute indicators meet the translation-invariant axiom.\footnote{These two axioms yield different implications for how inequality responds to a proportional change in the income of the entire population. A proportional income increase does not generate changes in income inequality as measured by relative indicators, but can provoke a large increase in inequality as measured by absolute indicators.} The question of which indicator should be used in practice has led to a heated debate in the literature \citep{milanovic2016global}. This is because relative and absolute indicators often provide different answers to important issues such as the distributive effects of globalization or trade openness.\footnote{As measured by absolute indicators, globalization has deteriorated the income distribution since the absolute income difference between the rich and the poor has increased. However, under the lens of relative measurement, globalization reduced income inequality since the poor's income has grown proportionally more than the income of the rich.} 

To shed some light on this debate, we assess whether people think about distributive fairness through the lens of relative or absolute indicators. To do this, we take a data-driven approach. We calculate 13 different measures of income inequality for all the countries in our sample and correlate each inequality indicator with the share of the population that believes income distribution is unfair over time.\footnote{The indicators are the Gini coefficient, the ratio between the 75th percentile and the 25th percentile of the income distribution, the ratio between the 90th and 10th percentile, the Atkinson index with an inequality aversion parameter equal to 0.5 and 1, the mean log deviation, the Theil index, the Generalized entropy index, the coefficient of variation, the absolute Gini, the Kolm index with an inequity aversion parameter equal to one, and the variance of the per capita household income (in 2005 PPP). These last three indices correspond to the absolute inequality measures, while the other ten indicators are relative inequality measures.} 

We calculate the correlation between unfairness perceptions and each inequality indicator at the regional level using three alternative aggregation methods. First, we calculate each correlation using the individual-level data and pooling all countries and years in our sample (columns 1-3). Second, we calculate the average unfairness views in each country-year and then calculate the correlation between each inequality indicator and the average fairness views in the corresponding country-year (columns 4-6). Third, we calculate the correlation between each inequality indicator and fairness views over time for each country separately (using the individual-level data) and then average the correlations across countries (columns 7-9). Table \ref{tab-rel-fairness} shows the results.

Fairness views tend to be positively correlated with relative indicators and negatively correlated with absolute ones. In Table \ref{tab-rel-fairness}, column 1 shows that the Gini is the indicator with the highest linear correlation. On the other hand, the absolute indicators of inequality tend to be \textit{negatively} correlated with unfairness perceptions, and the magnitude of such correlations tends to be small. The high correlation between unfairness perceptions and income inequality seems to be driven by the population that perceives inequality as very unfair (columns 2, 5, and 8), rather than just unfair (columns 3, 6, and 9).\footnote{It is interesting to note that indicators sometimes mentioned in the mass media, such as the ratio between the richest 90\% and the poorest 10\%, exhibit low explanatory power. This may be due to mismeasurement of top incomes in household surveys.}

These results are consistent with experimental evidence from \cite{amiel1992measurement,amiel1999thinking}, who show that support for the scale-invariance axiom is greater than for translation invariance, reflecting greater support for relative inequality indicators. %Moreover, the results are also consistent with previous evidence that documents decreasing relative inequality but \textit{rising} absolute inequality in LA during the 2000s. Since unfairness perceptions also declined over time, the relative indicators better trace such evolution.

