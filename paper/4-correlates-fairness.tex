\section{Individual-level Fairness Determinants} \label{sec_corr_fairness}

In this section, we study the individual-level correlates of fairness views. The purpose of this section is twofold. First, to assess whether the association between fairness views and income inequality is robust to including controls. Second, to investigate which individual-level characteristics are systematically related to fairness views. 

\subsection{Empirical Design}

To assess the relationship between individuals' characteristics and fairness perceptions, we estimate two-way fixed effects Logit models. This design controls for two important sources of heterogeneity that could drive the positive association between inequality and fairness perceptions documented in the previous section. First, it controls for country-level heterogeneity. This could matter if, for example, countries with historically extractive institutions have both higher levels of income inequality and more negative fairness views as a legacy of such institutions. Insofar as institutions are stable over time, the country fixed effects deal with this potential bias. Second, the design controls for year-level heterogeneity. This is important if, for example, in some particular years, macroeconomic events such as a financial crisis or a corruption scandal increase income disparities and worsen fairness views, again generating a spurious correlation between inequality and fairness perceptions. Including year fixed effects helps to alleviate such concerns.

Given that changes in fairness views over the last decade were driven by the share of the population that perceived the income distribution as very unfair (Figure \ref{fig-fairness-views}), in our baseline specification, we focus on explaining the determinants of this variable, although we also show the results for a broader definition of unfairness. We assume that unfairness perceptions can be characterized according to the following equation:
%
\begin{align} \label{eq_fairness}
	\text{VeryUnfair}_{ict} = F(\lambda_c + \lambda_t + \gamma \text{Gini}_{ct} + \beta x_{ict}),
\end{align}
%
where the dependent variable, $\text{VeryUnfair}_{ict}$ is equal to one if individual $i$ believes that the income distribution of country $c$ during year $t$ is very unfair and zero otherwise. Equation \eqref{eq_fairness} includes country fixed effects, $\lambda_{c}$; year fixed effects, $\lambda_t$; the country's Gini coefficient in year $t$, $\text{Gini}_{ct}$; and a vector of individual characteristics, $x_{ict}$, that contains age, age squared, sex, marital status, education, employment status, an assets index, political ideology, and religious views.\footnote{The assets index takes the value one if individual $i$ has access to running water and sewerage, owns a computer, a washing machine, a telephone, and a car. In household surveys, these variables tend to be correlated with household income, although the correlation is usually small. Unfortunately, we do not observe household income in the Latinobar\'ometro data. To measure political views, we rely on the question \textit{``In politics, people normally speak of ``left'' and ``right.'' On a scale where 0 is left and 10 is right, where would you place yourself?''} We interpret values closer to zero (ten) as closer to a liberal (conservative) worldview. We measure religious views using a dummy that is equal to one if individual $i$ is a Catholic ---the predominant religion in the region---coding other religions (and lack of religion) as zero.} In our baseline specification, $F(\cdot)$ is the logistic function. We cluster the standard errors at the country-by-year level.

We are interested in $\frac{\partial \text{VeryUnfair}_{ict}}{\partial x_{ict}} = \beta f(\cdot)$ and $\frac{\partial \text{VeryUnfair}_{ict}}{\partial \text{Gini}_{ict}} = \gamma f(\cdot)$. The first of these partial derivatives captures the relationship between an individual characteristic and unfairness perceptions, controlling for the rest of the characteristics, the Gini, and the fixed effects. Similarly, $\gamma f(\cdot)$ measures the relationship between the Gini and perceived fairness after controlling for individual-level traits and fixed effects. The magnitude of the partial derivatives depends on the value at which covariates are evaluated. We compute the marginal effects by evaluating all covariates at their average value.

\subsection{Regression Results}

Table \ref{very_unfair_logit} shows the estimated marginal effects under different specifications. Column 1 presents the results controlling only for the fixed effects and the Gini coefficient. Column 2 includes basic demographic indicators (age, gender, and marital status). Column 3 includes dummies for maximum educational attainment (the omitted category is completing up to elementary school). Column 4 includes dummies for labor force participation and unemployment. Column 5 includes an index for access to basic services and asset ownership. Column 6 includes political and religious views.

Consistent with the evidence shown in Section \ref{sec_ineq_fairness}, the Gini is positively and statistically significantly related to unfairness perceptions. In a country with average characteristics, a one point decrease in the Gini (from 0.49 to 0.48) decreases the share of the population that believes that the income distribution is very unfair by about half a percentage point ($p < 0.01$). This magnitude does not vary much across specifications (columns 1--6).\footnote{It is important to stress that the interpretation is not necessarily causal. The relationship between income inequality and unfairness perceptions can go both ways. On the one hand, higher inequality can increase the share of the population that believes the distribution is very unfair. But as more people perceive inequality as very unfair, the income distribution can change through several channels (e.g., more corruption or social unrest).}

Several individual-level characteristics predict fairness views. Older people are more likely to perceive the income distribution as very unfair, although the relationship between age and unfairness perception is non-linear. On average, males are just as likely as females to perceive the income distribution as very unfair, while married individuals are slightly less likely to do so. Completing high school is negatively associated with perceptions of unfairness, although the magnitude of the coefficient is small. Being economically active does not seem to be correlated with unfairness views, but being unemployed does. On average, unemployed individuals are about two percentage points more likely to perceive the income distribution as unfair than the employed population. The coefficient on the assets index is negative, suggesting that relatively better-off individuals are less likely to view the income distribution as very unfair, although the coefficient is not statistically different from zero. Ideologically conservative people are statistically less likely to perceive the income distribution as very unfair, although the effect size is small (below half a percentage point). Finally, Catholics are less likely to perceive the income distribution as very unfair.

\subsection{Robustness}

We conduct three robustness checks. First, we use a broader measure of unfairness that equals one if an individual perceives the income distribution as unfair \textit{or} very unfair as the dependent variable in equation \eqref{eq_fairness}. Appendix Table \ref{unfair_logit} shows the results. The magnitude of the coefficient on the Gini is smaller relative to our baseline specification. Given the relatively large standard errors, the 95\% confidence interval on the Gini includes zero; but the interval also contains the estimated marginal effects in our baseline set of regressions (Table \ref{very_unfair_logit}). In other words, when we use the broader measure of unfairness, our estimates become less informative. This is because strong unfairness views are more strongly correlated with income inequality than weak unfairness views (Table \ref{tab-rel-fairness} and Figure \ref{fig-fairness-gini}).\footnote{The coefficients of some individual-level characteristics are also different when using the broader definition of unfairness views. The effect of completing college on perceptions of unfairness becomes strong and statistically significant. Civil status stops being statistically significant, while the male dummy becomes negative and statistically significant (in both cases consistently so across specifications). The coefficient on the assets index becomes larger and statistically different from zero. Finally, the effect of political ideology and religious views becomes statistically indistinguishable from zero. These results suggest that the population that perceives the income distribution as very unfair tends to be different in observable variables than the population that believes that the income distribution is merely unfair.}

As a second robustness check, we estimate an analogous specification to the one in column 6 of Table \ref{very_unfair_logit}, but controlling for inequality indicators other than the Gini coefficient. Appendix Table \ref{unfair_ineq_logit} shows the results. We find a positive correlation between income inequality and unfairness perceptions across all relative measures of inequality (columns 1--4). The Gini calculated without households with zero income, the Atkinson index, the Theil index, and the Generalized Entropy indicator are consistently correlated with unfairness, and all the coefficients are statistically different from zero at the usual levels. In contrast, the absolute Gini (the only absolute inequality indicator in the table) is negatively correlated with unfairness perceptions, although the coefficient is statistically indistinguishable from zero.

As a final robustness check, we estimate equation \eqref{eq_fairness} using a linear probability model (LPM) instead of a Logit. The choice of a LPM is consistent with the visual evidence shown in Figure \ref{fig-fairness-gini}, Panel A, where fairness views seem to be linearly related to the Gini coefficient. Appendix Table \ref{very_unfair_lpm} shows the results. The estimates are quite similar across specifications. For example, in the specification with the larger set of controls (column 6), the marginal effect of the Gini coefficient is 0.68 in the Logit model and 0.63 in the LPM.

\subsection{Decomposing Changes in Fairness Views Over Time}

Both income inequality and individual-level characteristics are associated with fairness perceptions. Next, we ask which of these two factors mainly explain (in an accounting sense) the reduction in unfairness beliefs over the 2000s. To do this, we perform a Oaxaca-Blinder decomposition, taking 2002 and 2013 as the two groups to be compared (see Appendix \ref{sec_oaxaca} for details on the Oaxaca-Blinder decomposition). In the decomposition, we use the broad definition of unfairness perceptions as the dependent variable and include controls for demographics, educational attainment, employment status, assets, political views, and religion. Figure \ref{fig-oaxaca} summarizes the results.

During 2002--2013, the share of the population perceiving the distribution as unfair decreased 14 percentage points, from 87\% to 73\%. The decomposition suggests that about 28\% of this change (4 percentage points) is accounted for by changes in the elasticity of fairness views to each covariable (i.e., changes in the values of the coefficients in the regression), while the other 72\% can be explained by changes in the covariables' values. Among the covariables included in the decomposition, the one that mainly explains the decline in unfairness perceptions is the change in the Gini, which accounts for 88.9\% of the explained component. In contrast, changes in the composition of the population only account for 11.1\% of the explained component. This result suggests that the decline in unfairness views during the 2000s in Latin America was mainly driven by changes in income inequality and not by changes in the composition of the population.
