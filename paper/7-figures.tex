\section*{Tables and Figures}

\begin{table}[htbp]{\footnotesize
		\begin{center}
			\caption{Correlation between inequality indicators and fairness views, 1997-2015}\label{tab-rel-fairness}
			\begin{tabular}{lccccccccccc}
				\midrule
				& \multicolumn{3}{c}{} &   & \multicolumn{3}{c}{} &   & \multicolumn{3}{c}{Averaging correlations} \\
				& \multicolumn{3}{c}{Individual-level data} &   & \multicolumn{3}{c}{Country-by-year level data} &   & \multicolumn{3}{c}{across countries} \\
				\cmidrule{2-4}\cmidrule{6-8}\cmidrule{10-12}      &  U./V.U.  & V.U. & U. &   &  U./V.U.  & V.U. & U. &   &  U./V.U.  & V.U. & U. \\
				&  (1)  & (2) & (3) &   &  (4)  & (5) & (6) &   &  (7)  & (8) & (9) \\
				\midrule
				Gini coefficient & 0.39 & 0.36 & 0.10 &   & 0.84 & 0.82 & 0.25 &   & 0.39 & 0.28 & 0.15 \\
				& (0.07) & (0.07) & (0.09) &   & (0.10) & (0.16) & (0.35) &   & (0.07) & (0.07) & (0.09) \\
				Theil index & 0.39 & 0.38 & 0.05 &   & 0.85 & 0.82 & 0.27 &   & 0.33 & 0.20 & 0.18 \\
				& (0.07) & (0.07) & (0.09) &   & (0.09) & (0.16) & (0.35) &   & (0.07) & (0.07) & (0.09) \\
				Atkinson, A(0.5) & 0.38 & 0.35 & 0.10 &   & 0.84 & 0.82 & 0.25 &   & 0.37 & 0.26 & 0.15 \\
				& (0.07) & (0.07) & (0.09) &   & (0.10) & (0.16) & (0.35) &   & (0.07) & (0.07) & (0.09) \\
				Atkinson, A(1) & 0.36 & 0.30 & 0.13 &   & 0.84 & 0.82 & 0.25 &   & 0.38 & 0.28 & 0.13 \\
				& (0.07) & (0.08) & (0.09) &   & (0.10) & (0.16) & (0.34) &   & (0.07) & (0.08) & (0.09) \\
				Mean log deviation & 0.35 & 0.29 & 0.14 &   & 0.84 & 0.82 & 0.25 &   & 0.38 & 0.28 & 0.13 \\
				& (0.07) & (0.08) & (0.09) &   & (0.10) & (0.16) & (0.34) &   & (0.07) & (0.08) & (0.09) \\
				Coefficient Variation & 0.33 & 0.36 & 0.00 &   & 0.78 & 0.78 & 0.19 &   & 0.18 & 0.20 & 0.10 \\
				& (0.08) & (0.08) & (0.09) &   & (0.13) & (0.16) & (0.36) &   & (0.08) & (0.08) & (0.09) \\
				Ratio 75/25 & 0.29 & 0.15 & 0.24 &   & 0.80 & 0.78 & 0.26 &   & 0.36 & 0.29 & 0.12 \\
				& (0.07) & (0.09) & (0.08) &   & (0.11) & (0.17) & (0.33) &   & (0.07) & (0.09) & (0.08) \\
				Generalized entropy & 0.29 & 0.35 & -0.04 &   & 0.80 & 0.72 & 0.35 &   & 0.18 & 0.09 & 0.19 \\
				& (0.05) & (0.08) & (0.08) &   & (0.11) & (0.18) & (0.34) &   & (0.05) & (0.08) & (0.08) \\
				Ratio 90/10 & 0.23 & 0.10 & 0.21 &   & 0.81 & 0.79 & 0.25 &   & 0.30 & 0.30 & 0.07 \\
				& (0.07) & (0.08) & (0.08) &   & (0.11) & (0.17) & (0.32) &   & (0.07) & (0.08) & (0.08) \\
				Variance & -0.08 & -0.01 & -0.12 &   & -0.25 & 0.10 & -0.71 &   & -0.06 & 0.04 & -0.12 \\
				& (0.07) & (0.08) & (0.08) &   & (0.39) & (0.43) & (0.14) &   & (0.07) & (0.08) & (0.08) \\
				Absolute Gini & -0.21 & -0.10 & -0.18 &   & -0.71 & -0.46 & -0.64 &   & -0.18 & -0.10 & -0.22 \\
				& (0.09) & (0.10) & (0.08) &   & (0.23) & (0.26) & (0.31) &   & (0.09) & (0.10) & (0.08) \\
				Kolm, K(1) & -0.31 & -0.18 & -0.22 &   & -0.80 & -0.64 & -0.50 &   & -0.22 & -0.16 & -0.22 \\
				& (0.09) & (0.10) & (0.08) &   & (0.12) & (0.18) & (0.37) &   & (0.09) & (0.10) & (0.08) \\
				\midrule
			\end{tabular}
		\end{center}
		\begin{singlespace}  \vspace{-.5cm}
			\noindent \justify \textbf{Notes:} This table presents correlations between fairness views and income inequality. In columns 1--3, we calculate the correlations at the individual-level pooling all countries and years in our sample. In columns 4--6, we calculate the average unfairness views in each country-year and then calculate the correlation between each inequality indicator in the corresponding country-year and the average fairness views. In columns 7--9, we calculate the correlation between each inequality indicator and fairness views over time for each country separately (using the individual-level data) and then average the correlations across countries. U./V.U. stands for ``Unfair or Very Unfair''; V.U. stands for ``Very Unfair''; and U. stands for ``Unfair.'' Boostrapped standard errors are in parenthesis. 
		\end{singlespace} 	
	}
\end{table}


\clearpage
\begin{table}[htpb!]{\footnotesize
		\begin{center}
			\caption{Logit regressions of unfairness perceptions (very unfair) and individuals' characteristics} \label{very_unfair_logit}
			\newcommand\w{1.30}
			\begin{tabular}{l@{}lR{\w cm}@{}L{0.43cm}R{\w cm}@{}L{0.43cm}R{\w cm}@{}L{0.43cm}R{\w cm}@{}L{0.43cm}R{\w cm}@{}L{0.43cm}R{\w cm}@{}L{0.43cm}}
				\midrule
				&& 	\multicolumn{12}{c}{Dependent Variable: Believes income distribution is very unfair}  \\\cmidrule{3-14} 
				%					&&          &&       &&          && Labor     &&  && Ideology, \\
				%					&& Baseline && Demog.   && Education && status  && Assets && Religion \\
				&& (1) && (2) && (3) && (4) && (5) && (6) \\	
				\midrule 
				\ExpandableInput{../results/very_unfair_logit}
				\midrule
			\end{tabular}
		\end{center}
		\begin{singlespace}  \vspace{-.5cm}
			\noindent \justify \textbf{Notes:} This table shows estimates of the relationship between an indicator that equals one for individuals who believe that the income distribution is very unfair and the Gini coefficient controlling for individuals' characteristics. Coefficients are estimated through Logit regressions and represent the marginal effects evaluated at the mean values of the rest of the variables. Observations are weighted by the individual's probability of being interviewed. All specifications include country and year fixed effects. $^{***}$, $^{**}$ and $^*$ denote significance at 10\%, 5\% and 1\% levels, respectively. Heteroskedasticity-robust standard errors clustered at the country-by-year level in parentheses. 
		\end{singlespace} 	
	}
\end{table}


\begin{landscape}
	
	\begin{table}[htpb!]{\footnotesize
			\begin{center}
				\caption{Logit regressions of unfairness perceptions (very unfair) and activism}\label{very_unfair_unrest_past}
				\newcommand\w{1.58}
				\begin{tabular}{l@{}lR{\w cm}@{}L{0.43cm}R{\w cm}@{}L{0.43cm}R{\w cm}@{}L{0.43cm}R{\w cm}@{}L{0.43cm}R{\w cm}@{}L{0.43cm}R{\w cm}@{}L{0.43cm}R{\w cm}@{}L{0.43cm}R{\w cm}@{}L{0.43cm}R{\w cm}@{}L{0.43cm}}
					\midrule
					&& Complain   &&            &&          &&             &&          && Refuse  &&             && Occupy    && Activism \\
					&& on social  && Complain  && Sign a    && Authorized  && Unauth.  &&  to pay && Participate && land or   && composite \\
					&& media      && to media  && petition  && protest     && protest  && taxes   && in riots    && buildings && index \\
					&& (1)        &&     (2)   && (3)       &&  (4)        && (5)      && (6)     && (7)        && (8)        && (9) \\
					\midrule 
					

									\addlinespace
					\multicolumn{14}{l}{\hspace{-1em} \textbf{Panel A.\ Dependent variable: Have done the activity in the past}} \\
					\midrule

					\ExpandableInput{../results/very_unfair_unrest_past}
					\midrule
					
					\addlinespace
					\multicolumn{14}{l}{\hspace{-1em} \textbf{Panel B.\ Dependent variable: Have done the activity in the past or would do the activity}} \\ 
					\midrule
					\ExpandableInput{../results/very_unfair_unrest_would}
					\midrule
					
					\addlinespace
					\ExpandableInput{../results/very_unfair_unrest_past_N}
					\midrule
					
					
				\end{tabular}
			\end{center}
			\begin{singlespace} % \vspace{-.5cm}
				\noindent \justify \textbf{Notes:} This table shows estimates of the determinants of  participating in the political activity listed in the column header. Coefficients are estimated through Logit regressions and represent the marginal effects evaluated at the mean values of the rest of the variables. Observations are weighted by the individual's probability of being interviewed. All specifications control for age, age squared, gender, marital status, maximum educational attainment, labor force participation, unemployment status, assets index, political ideology, and religion. The regression in column 3 also controls for country and year fixed effects. Column 9 shows a composite index which takes the value one if the individual reports having engaged in tax evasion, an illegal protest, signed a petition, or complained in the media, and zero otherwise. The rest of the political activities are available in only one year and thus we cannot include fixed effects. Participation in political activities is self-reported.  $^{***}$, $^{**}$ and $^*$ denote significance at 10\%, 5\% and 1\% levels, respectively. Heteroskedasticity-robust standard errors clustered at the country-by-year level in parentheses. 
			\end{singlespace} 	
		}
	\end{table}
	
	
\end{landscape}

\clearpage 



\begin{figure}[htp]{\scriptsize
		\begin{centering}	
			\protect\caption{Fairness views in Latin America over time and across countries} \label{fig-fairness-views}
			\begin{minipage}{.48\textwidth}
				\caption*{Panel A. Fairness views over time \\ (1997--2015)}\label{fig-fairness-views-time}
				\includegraphics[width=\linewidth]{../results/../results/fig-intensity-fairness}
			\end{minipage}\hspace{1em}
			\begin{minipage}{.48\textwidth}
				\caption*{Panel B. Fairness views across countries \\ (2003 vs. 2013)} \label{fig-fairness-views-countries}				
				\includegraphics[width=\linewidth]{../results/fig-chg-fairness-0213}
			\end{minipage}
			\par\end{centering}
		
		\singlespacing\justify
		\textbf{Notes}: \footnotesize Panel A shows the percentage of individuals who perceive the income distribution as very unfair, unfair, fair, and very fair in our sample. We calculate the shares as the unweighted average of fairness views across the 18 countries in our sample. Panel B presents the percentage of the population who perceives the income distribution as either unfair or very unfair in 2002 and 2013 in each country of our sample.
		
	}
\end{figure}



\clearpage 
\begin{figure}[htpb]
	\caption{Fairness views and income inequality in Latin America} \label{fig-fairness-gini}
	\centering
	\begin{subfigure}[t]{0.48\textwidth}
		\caption*{Panel A. Correlation between unfairness views and Gini}\label{fig-binscatter-gini-unfair}
		\centering
		\includegraphics[width=\linewidth]{../results/fig-binscatter-gini-unfair-all}
	\end{subfigure}
	\hfill		
	\begin{subfigure}[t]{0.48\textwidth}
		\caption*{Panel B. Change in fairness and Gini across countries} \label{fig-chg-gini-unfair-0213}
		\includegraphics[width=\linewidth]{../results/fig-chg-gini-unfair-0213}
	\end{subfigure}	
	\hfill				
	{\footnotesize
		\singlespacing \justify
		
		\textbf{Notes:} Panel A shows a binned scatterplot of fairness views and the Gini coefficient. To construct this figure, we group the Ginis of all country-years in bins of width equal to 0.02 Gini points and calculate the average fairness perceptions in each bin.
		
		Panel B plots the percentage point change between 2002 and 2013 in the share of the population that believes that the income distribution is either unfair or very unfair ($y$-axis) against the change in the Gini coefficient between 2002 and 2013 ($x$-axis) for countries in our sample. Due to a break in data comparability or household data unavailability, for some countries, we use inequality data from adjacent years. In 2002, we use: Argentina 2004, Chile 2003, Costa Rica 2010, Ecuador 2003, Guatemala 2006, Nicaragua 2001, Panama 2008, and Peru 2004. In 2013 we use: Guatemala 2014, Mexico 2014, Nicaragua 2014, and Venezuela 2012. See Appendix \ref{sec_data} for more details.
		
	}	
\end{figure}



\clearpage 

\begin{figure}[htp]
	\caption{Oaxaca-Blinder decomposition of unfairness perceptions, 2002-2013}\label{fig-oaxaca}  \centering
	\centering
	\includegraphics[width=.75\linewidth]{../results/fig-oaxaca-0213.png}
	\hfill				
	{\footnotesize
		\singlespacing \justify
		
		\textbf{Notes:} This figure presents estimates of the Oaxaca-Blinder decomposition (see Appendix \ref{sec_oaxaca}). The dependent variable is an indicator that equals one for individuals who believe that the income distribution is unfair or very unfair. The regressors in the decomposition include the Gini coefficient, age, age squared, and dummy variables for  marital status, gender, educational attainment, labor force participation, unemployment status, an assets index, political ideology, and religious views. The ``explained'' part of the results refers to changes in the value of the covariables, while the ``unexplained'' refers to changes in the coefficients and the interaction terms.
		
	}	
\end{figure}
