% Fairness Paper - Appendix
% This file can be compiled standalone or included in the main document
%
% For standalone compilation: compile this file directly
% For inclusion: use % Fairness Paper - Appendix
% This file can be compiled standalone or included in the main document
%
% For standalone compilation: compile this file directly
% For inclusion: use % Fairness Paper - Appendix
% This file can be compiled standalone or included in the main document
%
% For standalone compilation: compile this file directly
% For inclusion: use % Fairness Paper - Appendix
% This file can be compiled standalone or included in the main document
%
% For standalone compilation: compile this file directly
% For inclusion: use \input{fairness_appendix} after \appendix in main document

\ifdefined\appendixstandalone
\else
  % Standalone preamble - only used when compiling this file directly
  \newif\ifstandalone
  \standalonetrue

  \documentclass[11pt]{article}
  \usepackage[a4paper,pdftex]{geometry}
  \setlength{\oddsidemargin}{3mm}
  \setlength{\evensidemargin}{3mm}
  \usepackage[T1]{fontenc}
  \usepackage[utf8x]{inputenc}
  \usepackage{amsmath,amsthm,amssymb,graphicx,enumerate,booktabs,bigstrut,rotating,multirow,float,caption,etoolbox,hyperref,lmodern,comment,listings,qtree,a4wide,titlesec,moresize,bbm,subcaption,tikz,pdfescape,mathtools,flafter,enumitem,setspace,ragged2e,colortbl,lineno,lscape}
  \hypersetup{colorlinks,linkcolor={red},citecolor={blue},urlcolor={blue}}

  \usepackage[justification=centering,font=small]{caption}

  \usepackage[authoryear]{natbib}
  \usepackage[english]{babel}
  \renewcommand{\baselinestretch}{1.25}
  \DeclareMathOperator{\E}{\mathbb{E}}
  \DeclareMathOperator*{\argmax}{argmax}
  \makeatletter
  \renewcommand\subsubsection{\@startsection{subsubsection}{3}{\z@}%
  	{-3.25ex\@plus -1ex \@minus -.2ex}%
  	{-1.5ex \@plus -.2ex}%
  	{\normalfont\normalsize\bfseries}}
  \def\@biblabel#1{\hspace*{-\labelsep}}

  \newcommand*\circled[1]{\tikz[baseline=(char.base)]{
  		\node[shape=circle,draw,inner sep=2pt] (char) {#1};}}
  \newcommand*\ExpandableInput[1]{\@@input#1 }
  \makeatother

  \def\sym#1{\ifmmode^{#1}\else\(^{#1}\)\fi}
  \onehalfspacing
  \pdfpageheight\paperheight
  \pdfpagewidth\paperwidth

  \newcolumntype{L}[1]{>{\raggedright\let\newline\\\arraybackslash\hspace{0pt}}m{#1}}
  \newcolumntype{C}[1]{>{\centering\let\newline\\\arraybackslash\hspace{0pt}}m{#1}}
  \newcolumntype{R}[1]{>{\raggedleft\let\newline\\\arraybackslash\hspace{0pt}}m{#1}}

  \begin{document}

  \title{Are Fairness Perceptions Shaped by Income Inequality? \\ Evidence from Latin America \\ \vspace{10pt} \Large Online Appendix}
  \author{Leonardo Gasparini \and Germ\'an Reyes}
  \date{\vspace{15pt} February 2022}
  \maketitle

  \appendix
\fi

%%%%%%%%%%%%%%%%%%%%%%%%%%%%%%%%%%%%%%%%%%%%%%%%%%%%%%%%%%%%%%%%%%%%%%%%%%%%%%%
% APPENDIX A: ADDITIONAL FIGURES AND TABLES
%%%%%%%%%%%%%%%%%%%%%%%%%%%%%%%%%%%%%%%%%%%%%%%%%%%%%%%%%%%%%%%%%%%%%%%%%%%%%%%

\begin{center}\noindent{\LARGE \textbf{Appendix}}\end{center}

\section{Additional Figures and Tables} \label{app_add_figs}

\setcounter{table}{0}
\setcounter{figure}{0}
\setcounter{equation}{0}
\renewcommand{\thetable}{A\arabic{table}}
\renewcommand{\thefigure}{A\arabic{figure}}
\renewcommand{\theequation}{A\arabic{equation}}


\begin{figure}[htpb]
	\caption{Perceptions of unfairness and individual characteristics, 1997--2015} \label{fig-fairness-grps}
	\centering
	\begin{subfigure}[t]{.48\textwidth}
		\caption*{Panel A. By age}\label{fig-fairness-age}
		\centering
		\includegraphics[width=\linewidth]{../results/fig-fairness-age}
	\end{subfigure}
	\hfill
	\begin{subfigure}[t]{0.48\textwidth}
		\caption*{Panel B. By sex}\label{fig-fairness-sex}
		\centering
		\includegraphics[width=\linewidth]{../results/fig-fairness-sex}
	\end{subfigure}
	\hfill
	\begin{subfigure}[t]{.48\textwidth}
		\caption*{Panel C. By educational attainment}\label{fig-fairness-educ}
		\centering
		\includegraphics[width=\linewidth]{../results/fig-fairness-educ}
	\end{subfigure}
	\hfill
	\begin{subfigure}[t]{0.48\textwidth}
		\caption*{Panel D. By employment status}\label{fig-fairness-employ}
		\centering
		\includegraphics[width=\linewidth]{../results/fig-fairness-employ}
	\end{subfigure}
	\hfill
	{\footnotesize
		\singlespacing \justify

		\textbf{Notes:} This figure shows the share of individuals who perceive the income distribution as unfair or very unfair according to their age, gender, maximum educational attainment, and employment status.


	}
\end{figure}

\clearpage
\begin{figure}[htp]
	\caption{The evolution of fairness views and income inequality in Latin America}\label{fig-timeseries-gini-unfair}  \centering
	\centering
	\includegraphics[width=.75\linewidth]{../results/fig-timeseries-gini-unfair}
	\hfill
	{\footnotesize
		\singlespacing \justify

		\textbf{Notes:} This figure shows the evolution of the average Gini coefficient across countries in our sample (right-hand-side variable) and the fraction of the population who perceive the income distribution as unfair or very unfair (left-hand-side variable) over 1997--2015. To have a balanced panel of countries over time, we linearly extrapolated the Gini coefficient in years in which income microdata is not available (see Appendix \ref{sec_data}).

	}
\end{figure}



\clearpage
\begin{table}[H]{\footnotesize
		\begin{center}
			\caption{Descriptive statistics of our sample} \label{tab-sum-stats}
			\begin{tabular}{lccc}
				\midrule
				&  Mean  &  Standard Dev.  & Observations \\
				& (1) & (2) & (3) \\
				\midrule
				\textbf{\hspace{-1em} Panel A. Sociodemographic} &   &   &  \\
				Age & 39.75 & 16.23 &         225,551  \\
				Male (\%) & 48.97 & 0.50 &         225,567  \\
				Married or civil union (\%) & 56.27 & 0.50 &         224,081  \\
				Catholic religion (\%) & 68.01 & 0.47 &         222,790  \\
				Ideology (10 = right-wing) & 5.48 & 2.64 &         131,980  \\
				\textbf{\hspace{-1em} Panel B. Education and Labor market} &   &   &  \\
				Literate (\%) & 90.31 & 0.30 &         224,056  \\
				Secondary education or more (\%) & 33.65 & 0.47 &         224,056  \\
				Parents with secondary education (\%) & 17.43 & 0.38 &         184,884  \\
				Economically active (\%) & 64.14 & 0.48 &         225,222  \\
				Unemployed (\% Labor Force) & 9.89 & 0.30 &         225,222  \\
				\textbf{\hspace{-1em} Panel C. Access to services} &   &   &  \\
				Access to a sewerage (\%) & 69.59 & 0.46 &         222,530  \\
				Access to running water (\%) & 88.83 & 0.31 &         204,340  \\
				\textbf{\hspace{-1em} Panel D. Asset ownership} &   &   &  \\
				Car (\%) & 28.21 & 0.45 &         222,338  \\
				Computer (\%) & 33.79 & 0.47 &         222,645  \\
				Fridge (\%) & 79.22 & 0.41 &         146,686  \\
				Homeowner (\%) & 73.92 & 0.44 &         223,603  \\
				Mobile (\%) & 80.61 & 0.40 &         172,253  \\
				Washing machine (\%) & 54.71 & 0.50 &         223,122  \\
				Landline (\%) & 42.28 & 0.49 &         222,968  \\
				\midrule
			\end{tabular}
		\end{center}
		\begin{singlespace} \vspace{-.5cm}
			\noindent \justify \textbf{Note:} This table shows summary statistics on our sample pooling data from all countries in our sample over 1997--2015.
		\end{singlespace}
	}
\end{table}


% Fairness perceptions by population group pooling data 1997-2015, in %
\clearpage
\begin{table}[H]{\footnotesize
		\begin{center}
			\caption{Fairness views by population group} \label{tab-fairness-grp}
			\newcommand\w{1.70}
			\begin{tabular}{l@{}R{\w cm}R{\w cm}R{\w cm}R{\w cm}}
				\midrule
				& \multicolumn{4}{c}{\% of individuals who believe income distribution is:} \\
				\cmidrule{2-5}  & Very unfair & Unfair & Fair & Very fair \\
				& (1) & (2) & (3) & (4) \\
				\midrule
				All & 28.2 & 51.6 & 17.3 & 2.9 \\
				\textbf{\hspace{-1em} Panel A. Gender} &   &   &   &  \\
				Female & 28.3 & 52.2 & 16.7 & 2.8 \\
				Male & 28.0 & 51.1 & 17.9 & 3.0 \\
				\textbf{\hspace{-1em} Panel B. Age group} &   &   &   &  \\
				15-24 & 25.2 & 52.0 & 19.7 & 3.1 \\
				25-40 & 28.5 & 51.3 & 17.2 & 3.0 \\
				41-64 & 29.5 & 51.7 & 16.0 & 2.8 \\
				65+ & 29.2 & 51.6 & 16.6 & 2.5 \\
				\textbf{\hspace{-1em} Panel C. Civil status} &   &   &   &  \\
				Married & 28.3 & 51.9 & 17.0 & 2.8 \\
				Not married & 27.9 & 51.4 & 17.7 & 3.1 \\
				\textbf{\hspace{-1em} Panel D. Religion} &   &   &   &  \\
				Catholic & 28.2 & 51.7 & 17.2 & 2.9 \\
				Not catholic & 28.0 & 51.5 & 17.5 & 3.0 \\
				\textbf{\hspace{-1em} Panel E. Education level} &   &   &   &  \\
				Less than Primary & 27.7 & 51.6 & 17.7 & 3.0 \\
				Complete Primary & 27.9 & 52.2 & 17.4 & 2.6 \\
				Complete Secondary & 29.1 & 53.2 & 15.0 & 2.7 \\
				Complete Tertiary & 29.0 & 50.8 & 17.1 & 3.1 \\
				\textbf{\hspace{-1em} Panel F. Type of employment} &   &   &   &  \\
				Employee & 28.3 & 51.5 & 17.2 & 2.9 \\
				Employer & 24.3 & 53.9 & 19.0 & 2.8 \\
				Self-employed & 28.0 & 51.4 & 17.5 & 3.1 \\
				Unemployed & 30.3 & 51.6 & 15.1 & 3.0 \\
				\textbf{\hspace{-1em} Panel E. Country} &   &   &   &  \\
				Argentina & 38.17 & 50.74 & 10.26 & 0.83 \\
				Bolivia & 18.01 & 56.13 & 23.39 & 2.48 \\
				Brazil & 31.95 & 53.71 & 12.85 & 1.49 \\
				Chile & 40.20 & 49.93 & 8.42 & 1.45 \\
				Colombia & 35.15 & 51.20 & 11.40 & 2.26 \\
				Costa Rica & 23.20 & 53.55 & 20.13 & 3.12 \\
				Dominican Rep. & 32.31 & 46.52 & 17.61 & 3.56 \\
				Ecuador & 21.45 & 47.46 & 27.58 & 3.51 \\
				El Salvador & 22.73 & 53.16 & 20.45 & 3.65 \\
				Guatemala & 28.29 & 51.34 & 16.70 & 3.66 \\
				Honduras & 28.87 & 53.42 & 14.33 & 3.38 \\
				Mexico & 32.15 & 49.75 & 15.32 & 2.78 \\
				Nicaragua & 18.69 & 51.88 & 24.33 & 5.11 \\
				Panama & 27.39 & 48.01 & 20.25 & 4.34 \\
				Paraguay & 38.31 & 48.80 & 10.95 & 1.93 \\
				Peru & 25.03 & 61.89 & 11.70 & 1.38 \\
				Uruguay & 18.22 & 57.51 & 22.64 & 1.64 \\
				Venezuela & 23.51 & 42.96 & 26.62 & 6.92 \\
				\midrule
			\end{tabular}
		\end{center}
		\begin{singlespace} \vspace{-.5cm}
			\noindent \justify \textbf{Note:} This table shows the fraction of individuals in our sample who perceive the income distribution as very unfair, unfair, fair, or very fair.
		\end{singlespace}
	}
\end{table}



\clearpage
\begin{table}[htpb!]{\footnotesize
		\begin{center}
			\caption{Logit regressions of unfairness perceptions (unfair) and individual characteristics} \label{unfair_logit}
			\newcommand\w{1.30}
			\begin{tabular}{l@{}lR{\w cm}@{}L{0.43cm}R{\w cm}@{}L{0.43cm}R{\w cm}@{}L{0.43cm}R{\w cm}@{}L{0.43cm}R{\w cm}@{}L{0.43cm}R{\w cm}@{}L{0.43cm}}
				\midrule
				&& 	\multicolumn{12}{c}{Dependent Variable: Believes income distribution is unfair or very unfair}  \\\cmidrule{3-14}
				%					&&          &&       &&          && Labor     &&  && Ideology, \\
				%					&& Baseline && Demog.   && Education && status  && Assets && Religion \\
				&& (1) && (2) && (3) && (4) && (5) && (6) \\
				\midrule
				\ExpandableInput{../results/unfair_logit}
				\midrule
			\end{tabular}
		\end{center}
		\begin{singlespace}  \vspace{-.5cm}
			\noindent \justify \textbf{Notes:} This table shows estimates of the relationship between an indicator that equals one for individuals who believe that the income distribution is unfair or very unfair and the Gini coefficient controlling for individuals' characteristics. Coefficients are estimated through Logit regressions and represent the marginal effects evaluated at the mean values of the rest of the variables. Observations are weighted by the individual's probability of being interviewed. All specifications include country and year fixed effects. $^{***}$, $^{**}$ and $^*$ denote significance at 10\%, 5\% and 1\% levels, respectively. Heteroskedasticity-robust standard errors clustered at the country-by-year level in parentheses.
		\end{singlespace}
	}
\end{table}


\clearpage
\begin{table}[htpb!]{\footnotesize
		\begin{center}
			\caption{Logit regressions of unfairness perceptions (very unfair) and different inequality indicators} \label{unfair_ineq_logit}
			\newcommand\w{1.50}
			\begin{tabular}{l@{}lR{\w cm}@{}L{0.43cm}R{\w cm}@{}L{0.43cm}R{\w cm}@{}L{0.43cm}R{\w cm}@{}L{0.43cm}R{\w cm}@{}L{0.43cm}}
				\midrule
				&& 	\multicolumn{10}{c}{Dependent Variable: Believes income distribution is very unfair}  \\\cmidrule{3-12}
				&& (1) && (2) && (3) && (4) && (5)  \\
				\midrule
				\ExpandableInput{../results/very_unfair_ineq_logit}
				\midrule
			\end{tabular}
		\end{center}
		\begin{singlespace}  \vspace{-.5cm}
			\noindent \justify \textbf{Notes:} This table shows estimates of the relationship between an indicator that equals one for individuals who believe income distribution is unfair or very unfair and several inequality indicators controlling for individuals' characteristics. Coefficients are estimated through Logit regressions and represent the marginal effects evaluated at the mean values of the rest of the variables. Observations are weighted by the individual's probability of being interviewed. All specifications include country and year fixed effects. $^{***}$, $^{**}$ and $^*$ denote significance at 10\%, 5\% and 1\% levels, respectively. Heteroskedasticity-robust standard errors clustered at the country-by-year level in parentheses.

		\end{singlespace}
	}
\end{table}




\begin{table}[htpb!]{\footnotesize
		\begin{center}
			\caption{OLS regressions of unfairness perceptions (very unfair) and individual characteristics} \label{very_unfair_lpm}
			\newcommand\w{1.30}
			\begin{tabular}{l@{}lR{\w cm}@{}L{0.43cm}R{\w cm}@{}L{0.43cm}R{\w cm}@{}L{0.43cm}R{\w cm}@{}L{0.43cm}R{\w cm}@{}L{0.43cm}R{\w cm}@{}L{0.43cm}}
				\midrule
				&& 	\multicolumn{12}{c}{Dependent Variable: Believes income distribution is very unfair}  \\\cmidrule{3-14}
				%					&&          &&       &&          && Labor     &&  && Ideology, \\
				%					&& Baseline && Demog.   && Education && status  && Assets && Religion \\
				&& (1) && (2) && (3) && (4) && (5) && (6) \\
				\midrule
				\ExpandableInput{../results/very_unfair_lpm}
				\midrule
			\end{tabular}
		\end{center}
		\begin{singlespace}  \vspace{-.5cm}
			\noindent \justify \textbf{Notes:} This table presents estimates of the correlation between a dummy variable that indicates if the individual believes income distribution is unfair or very unfair and the Gini coefficient controlling for individuals' characteristics. Coefficients are estimated through a linear probability model. Observations are weighted by the individual's probability of being interviewed. All specifications include country and year fixed effects. $^{***}$, $^{**}$ and $^*$ denote significance at 10\%, 5\% and 1\% levels, respectively. Heteroskedasticity-robust standard errors clustered at the country-by-year level in parentheses.
		\end{singlespace}
	}
\end{table}


%%%%%%%%%%%%%%%%%%%%%%%%%%%%%%%%%%%%%%%%%%%%%%%%%%%%%%%%%%%%%%%%%%%%%%%%%%%%%%%
% APPENDIX B: DATA APPENDIX
%%%%%%%%%%%%%%%%%%%%%%%%%%%%%%%%%%%%%%%%%%%%%%%%%%%%%%%%%%%%%%%%%%%%%%%%%%%%%%%

\clearpage
\section{Data Appendix} \label{sec_data}

\setcounter{table}{0}
\setcounter{figure}{0}
\renewcommand{\thetable}{B\arabic{table}}
\renewcommand{\thefigure}{B\arabic{figure}}

The figures presented in this paper are based on two harmonization projects, known as Latinobar\'ometro and SEDLAC (Socio-Economic Database for Latin America and the Caribbean). In this Appendix, we describe how we make both sources compatible.

Our perceptions data come from Latinobar\'ometro, which has conducted opinion surveys in 18 LA countries since the 1990s, interviewing about 1,200 individuals per country about individuals' socioeconomic background, and preferences towards political and social issues. Unfortunately, not all years contain questions about individuals' fairness perceptions. The survey was designed to be representative of the voting-age population at the national level (in most LA countries, individuals aged over 18). In Table \ref{tab-coverage} we show what percentage of the voting-age population is represented by the survey in each country for all the years in which the fairness question is available.



\begin{table}[htbp]{\footnotesize
		\begin{center}
			\caption{Coverage of each country's population in Latinobar\'ometro overtime (in \%)}\label{tab-coverage}
			\begin{tabular}{lrrrrrrrrr}
				\midrule
				& 1997 & 2001 & 2002 & 2007 & 2009 & 2010 & 2011 & 2013 & 2015 \\
				\midrule
				Argentina &         68  &         75  &         75  &       100  &       100  &       100  &       100  &       100  &       100  \\
				Bolivia &         32  &         52  &       100  &       100  &       100  &       100  &       100  &       100  &       100  \\
				Brazil &         12  &       100  &       100  &       100  &       100  &       100  &       100  &       100  &       100  \\
				Chile &         70  &         70  &         70  &       100  &       100  &       100  &       100  &       100  &       100  \\
				Colombia &         25  &         71  &         51  &       100  &       100  &       100  &       100  &       100  &       100  \\
				Costa Rica &       100  &       100  &       100  &       100  &       100  &       100  &       100  &       100  &       100  \\
				Dominican Republic & \multicolumn{1}{c}{ N/A } & \multicolumn{1}{c}{ N/A } & \multicolumn{1}{c}{ N/A } &       100  &       100  &       100  &       100  &       100  &       100  \\
				Ecuador &         97  &         97  &       100  &       100  &       100  &       100  &       100  &       100  &       100  \\
				El Salvador &         65  &       100  &       100  &       100  &       100  &       100  &       100  &       100  &       100  \\
				Guatemala &       100  &       100  &       100  &         97  &       100  &       100  &       100  &       100  &       100  \\
				Honduras &       100  &       100  &       100  &         98  &       100  &         99  &         99  &         99  &         99  \\
				Mexico &         93  &         88  &         95  &       100  &       100  &       100  &       100  &       100  &       100  \\
				Nicaragua &       100  &       100  &       100  &       100  &       100  &       100  &       100  &       100  &       100  \\
				Panama &       100  &       100  &       100  &         99  &         99  &         99  &         99  &         99  &         99  \\
				Paraguay &         46  &         46  &         46  &       100  &       100  &       100  &       100  &       100  &       100  \\
				Peru &         52  &         52  &       100  &       100  &       100  &       100  &       100  &       100  &       100  \\
				Uruguay &         80  &         80  &         80  &       100  &       100  &       100  &       100  &       100  &       100  \\
				Venezuela &       100  &       100  &       100  &       100  &         93  &       100  &       100  &       100  &       100  \\
				Weighted average &         68  &         86  &         91  &       100  &       100  &       100  &       100  &       100  &       100  \\
				\midrule
			\end{tabular}
		\end{center}
		%		\begin{singlespace}  \vspace{-.5cm}
			%			\noindent \justify \textbf{Notes:} This table presents the percentage of the voting-age population represented each year in Latinobar\'ometro overtime. The regional average is calculated by weighting each country's population. N/A means that Latinobar\'ometro did not conduct the opinion poll in that particular country-year.
			%		\end{singlespace}
	}
\end{table}


Since our goal is to analyze how unfairness perceptions evolved vis-\`a-vis changes in income inequality, we put a lot of effort into getting income inequality data for each data point for which we have perceptions data available. We made two partial fixes to increase the number of observations available (without pushing the data too much). First, we filled the data gaps using household surveys of relatively close years in which previously unused data were available (see Appendix Table \ref{tab-circa}). For instance, Chile conducts household surveys on average every two years. In 1997, there is perceptions data available, but no data on income inequality. Therefore, we use the inequality data from an adjacent year (1998). As noted previously, we only use data from close years if the data from the adjacent year correspond to a year in which the perceptions question was not asked (and therefore, inequality data are not needed in that year).

\begin{table}[htbp]{\footnotesize
		\begin{center}
			\caption{Circa years used to fill data gaps}\label{tab-circa}
			\begin{tabular}{lcc}
				\midrule
				Country & Year without household data & Data point used instead \\
				\midrule
				Chile & 1997 & 1998 \\
				Chile & 2001 & 2000 \\
				Chile & 2002 & 2003 \\
				Chile & 2007 & 2006 \\
				Colombia & 2007 & 2008 \\
				Ecuador & 2002 & 2003 \\
				El Salvador & 1997 & 1998 \\
				Guatemala & 2001 & 2000 \\
				Guatemala & 2015 & 2014 \\
				Mexico & 1997 & 1998 \\
				Mexico & 2001 & 2000 \\
				Mexico & 2007 & 2006 \\
				Mexico & 2009 & 2008 \\
				Mexico & 2011 & 2012 \\
				Mexico & 2015 & 2014 \\
				Nicaragua & 1997 & 1998 \\
				Nicaragua & 2007 & 2005 \\
				Nicaragua & 2015 & 2014 \\
				Venezuela & 2013 & 2012 \\
				\midrule
			\end{tabular}
		\end{center}
		\begin{singlespace}  \vspace{-.5cm}
			\noindent \justify %\textbf{Notes:}
		\end{singlespace}
	}
\end{table}

Our second partial fix involves interpolating inequality indicators for some years. For some countries, a few years had perceptions data available but no comparable household survey over time and no close year available. In this case, and to analyze the same set of countries every year, interpolation was applied to the inequality indicators (see Appendix Table \ref{tab-interp}).

\begin{table}[htbp]{\footnotesize
		\begin{center}
			\caption{Years in which inequality indicators were calculated with a linear interpolation}\label{tab-interp}
			\begin{tabular}{ll}
				\midrule
				Country & Years interpolated \\
				\midrule
				Argentina & 1997, 2001, and 2002 \\
				Bolivia & 2010 \\
				Brazil & 2010 \\
				Chile & 2010 \\
				Colombia & 1997 \\
				Costa Rica & 1997, 2001, 2002, 2007, and 2009 \\
				Ecuador & 1997, 2001 \\
				Guatemala & 1997, 2002, 2009, 2010, and 2013 \\
				Mexico & 2013 \\
				Nicaragua & 2002, 2010, 2011, and 2013 \\
				Panama & 1997, 2001, 2002, and 2007 \\
				Peru & 1997, 2001, 2002 \\
				Venezuela & 2015 \\
				\midrule
			\end{tabular}
		\end{center}
		\begin{singlespace}  \vspace{-.5cm}
			\noindent \justify %\textbf{Notes:}
		\end{singlespace}
	}
\end{table}


Overall, the years in which income inequality was calculated using linear interpolations represent a relatively small share of the total data points (17\% of total). The majority of our inequality data points (69\%) are calculated using a household survey from the same year in which the perceptions polls were conducted, while the remaining 14\% of our inequality indicators are calculated using household surveys from adjacent years. Table \ref{tab-summ-data} summarizes the data sources used in years perceptions data are available.

\begin{table}[htbp]{\footnotesize
		\begin{center}
			\caption{Summary of the data used in every country-year}\label{tab-summ-data}
			\begin{tabular}{r|r|llllllll}
				\multicolumn{1}{r}{} & \multicolumn{1}{r}{1997} & \multicolumn{1}{r}{2001} & \multicolumn{1}{r}{2002} & \multicolumn{1}{r}{2007} & \multicolumn{1}{r}{2009} & \multicolumn{1}{r}{2010} & \multicolumn{1}{r}{2011} & \multicolumn{1}{r}{2013} & \multicolumn{1}{r}{2015} \\
				\cmidrule{2-10}    \multicolumn{1}{l|}{Argentina} & \cellcolor[rgb]{ .608,  .733,  .349}\textcolor[rgb]{ .608,  .733,  .349}{3} & \multicolumn{1}{r|}{\cellcolor[rgb]{ .608,  .733,  .349}\textcolor[rgb]{ .608,  .733,  .349}{3}} & \multicolumn{1}{r|}{\cellcolor[rgb]{ .608,  .733,  .349}\textcolor[rgb]{ .608,  .733,  .349}{3}} & \multicolumn{1}{r|}{\cellcolor[rgb]{ .31,  .506,  .741}\textcolor[rgb]{ .31,  .506,  .741}{1}} & \multicolumn{1}{r|}{\cellcolor[rgb]{ .31,  .506,  .741}\textcolor[rgb]{ .31,  .506,  .741}{1}} & \multicolumn{1}{r|}{\cellcolor[rgb]{ .31,  .506,  .741}\textcolor[rgb]{ .31,  .506,  .741}{1}} & \multicolumn{1}{r|}{\cellcolor[rgb]{ .31,  .506,  .741}\textcolor[rgb]{ .31,  .506,  .741}{1}} & \multicolumn{1}{r|}{\cellcolor[rgb]{ .31,  .506,  .741}\textcolor[rgb]{ .31,  .506,  .741}{1}} & \multicolumn{1}{r|}{\cellcolor[rgb]{ .969,  .588,  .275}\textcolor[rgb]{ .969,  .588,  .275}{2}} \\
				\cmidrule{2-10}    \multicolumn{1}{l|}{Bolivia} & \cellcolor[rgb]{ .31,  .506,  .741}\textcolor[rgb]{ .31,  .506,  .741}{1} & \multicolumn{1}{r|}{\cellcolor[rgb]{ .31,  .506,  .741}\textcolor[rgb]{ .31,  .506,  .741}{1}} & \multicolumn{1}{r|}{\cellcolor[rgb]{ .31,  .506,  .741}\textcolor[rgb]{ .31,  .506,  .741}{1}} & \multicolumn{1}{r|}{\cellcolor[rgb]{ .31,  .506,  .741}\textcolor[rgb]{ .31,  .506,  .741}{1}} & \multicolumn{1}{r|}{\cellcolor[rgb]{ .31,  .506,  .741}\textcolor[rgb]{ .31,  .506,  .741}{1}} & \multicolumn{1}{r|}{\cellcolor[rgb]{ .608,  .733,  .349}\textcolor[rgb]{ .608,  .733,  .349}{3}} & \multicolumn{1}{r|}{\cellcolor[rgb]{ .31,  .506,  .741}\textcolor[rgb]{ .31,  .506,  .741}{1}} & \multicolumn{1}{r|}{\cellcolor[rgb]{ .31,  .506,  .741}\textcolor[rgb]{ .31,  .506,  .741}{1}} & \multicolumn{1}{r|}{\cellcolor[rgb]{ .31,  .506,  .741}\textcolor[rgb]{ .31,  .506,  .741}{1}} \\
				\cmidrule{2-10}    \multicolumn{1}{l|}{Brazil} & \cellcolor[rgb]{ .31,  .506,  .741}\textcolor[rgb]{ .31,  .506,  .741}{1} & \multicolumn{1}{r|}{\cellcolor[rgb]{ .31,  .506,  .741}\textcolor[rgb]{ .31,  .506,  .741}{1}} & \multicolumn{1}{r|}{\cellcolor[rgb]{ .31,  .506,  .741}\textcolor[rgb]{ .31,  .506,  .741}{1}} & \multicolumn{1}{r|}{\cellcolor[rgb]{ .31,  .506,  .741}\textcolor[rgb]{ .31,  .506,  .741}{1}} & \multicolumn{1}{r|}{\cellcolor[rgb]{ .31,  .506,  .741}\textcolor[rgb]{ .31,  .506,  .741}{1}} & \multicolumn{1}{r|}{\cellcolor[rgb]{ .608,  .733,  .349}\textcolor[rgb]{ .608,  .733,  .349}{3}} & \multicolumn{1}{r|}{\cellcolor[rgb]{ .31,  .506,  .741}\textcolor[rgb]{ .31,  .506,  .741}{1}} & \multicolumn{1}{r|}{\cellcolor[rgb]{ .31,  .506,  .741}\textcolor[rgb]{ .31,  .506,  .741}{1}} & \multicolumn{1}{r|}{\cellcolor[rgb]{ .31,  .506,  .741}\textcolor[rgb]{ .31,  .506,  .741}{1}} \\
				\cmidrule{2-10}    \multicolumn{1}{l|}{Chile} & \cellcolor[rgb]{ .969,  .588,  .275}\textcolor[rgb]{ .969,  .588,  .275}{2} & \multicolumn{1}{r|}{\cellcolor[rgb]{ .969,  .588,  .275}\textcolor[rgb]{ .969,  .588,  .275}{2}} & \multicolumn{1}{r|}{\cellcolor[rgb]{ .969,  .588,  .275}\textcolor[rgb]{ .969,  .588,  .275}{2}} & \multicolumn{1}{r|}{\cellcolor[rgb]{ .969,  .588,  .275}\textcolor[rgb]{ .969,  .588,  .275}{2}} & \multicolumn{1}{r|}{\cellcolor[rgb]{ .31,  .506,  .741}\textcolor[rgb]{ .31,  .506,  .741}{1}} & \multicolumn{1}{r|}{\cellcolor[rgb]{ .608,  .733,  .349}\textcolor[rgb]{ .608,  .733,  .349}{3}} & \multicolumn{1}{r|}{\cellcolor[rgb]{ .31,  .506,  .741}\textcolor[rgb]{ .31,  .506,  .741}{1}} & \multicolumn{1}{r|}{\cellcolor[rgb]{ .31,  .506,  .741}\textcolor[rgb]{ .31,  .506,  .741}{1}} & \multicolumn{1}{r|}{\cellcolor[rgb]{ .31,  .506,  .741}\textcolor[rgb]{ .31,  .506,  .741}{1}} \\
				\cmidrule{2-10}    \multicolumn{1}{l|}{Colombia} & \cellcolor[rgb]{ .608,  .733,  .349}\textcolor[rgb]{ .608,  .733,  .349}{3} & \multicolumn{1}{r|}{\cellcolor[rgb]{ .31,  .506,  .741}\textcolor[rgb]{ .31,  .506,  .741}{1}} & \multicolumn{1}{r|}{\cellcolor[rgb]{ .31,  .506,  .741}\textcolor[rgb]{ .31,  .506,  .741}{1}} & \multicolumn{1}{r|}{\cellcolor[rgb]{ .969,  .588,  .275}\textcolor[rgb]{ .969,  .588,  .275}{2}} & \multicolumn{1}{r|}{\cellcolor[rgb]{ .31,  .506,  .741}\textcolor[rgb]{ .31,  .506,  .741}{1}} & \multicolumn{1}{r|}{\cellcolor[rgb]{ .31,  .506,  .741}\textcolor[rgb]{ .31,  .506,  .741}{1}} & \multicolumn{1}{r|}{\cellcolor[rgb]{ .31,  .506,  .741}\textcolor[rgb]{ .31,  .506,  .741}{1}} & \multicolumn{1}{r|}{\cellcolor[rgb]{ .31,  .506,  .741}\textcolor[rgb]{ .31,  .506,  .741}{1}} & \multicolumn{1}{r|}{\cellcolor[rgb]{ .31,  .506,  .741}\textcolor[rgb]{ .31,  .506,  .741}{1}} \\
				\cmidrule{2-10}    \multicolumn{1}{l|}{Costa Rica} & \cellcolor[rgb]{ .608,  .733,  .349}\textcolor[rgb]{ .608,  .733,  .349}{3} & \multicolumn{1}{r|}{\cellcolor[rgb]{ .608,  .733,  .349}\textcolor[rgb]{ .608,  .733,  .349}{3}} & \multicolumn{1}{r|}{\cellcolor[rgb]{ .608,  .733,  .349}\textcolor[rgb]{ .608,  .733,  .349}{3}} & \multicolumn{1}{r|}{\cellcolor[rgb]{ .608,  .733,  .349}\textcolor[rgb]{ .608,  .733,  .349}{3}} & \multicolumn{1}{r|}{\cellcolor[rgb]{ .608,  .733,  .349}\textcolor[rgb]{ .608,  .733,  .349}{3}} & \multicolumn{1}{r|}{\cellcolor[rgb]{ .31,  .506,  .741}\textcolor[rgb]{ .31,  .506,  .741}{1}} & \multicolumn{1}{r|}{\cellcolor[rgb]{ .31,  .506,  .741}\textcolor[rgb]{ .31,  .506,  .741}{1}} & \multicolumn{1}{r|}{\cellcolor[rgb]{ .31,  .506,  .741}\textcolor[rgb]{ .31,  .506,  .741}{1}} & \multicolumn{1}{r|}{\cellcolor[rgb]{ .31,  .506,  .741}\textcolor[rgb]{ .31,  .506,  .741}{1}} \\
				\cmidrule{2-10}    \multicolumn{1}{l|}{Dominican Rep.} & \cellcolor[rgb]{ .749,  .749,  .749}\textcolor[rgb]{ .749,  .749,  .749}{0} & \multicolumn{1}{r|}{\cellcolor[rgb]{ .31,  .506,  .741}\textcolor[rgb]{ .31,  .506,  .741}{1}} & \multicolumn{1}{r|}{\cellcolor[rgb]{ .31,  .506,  .741}\textcolor[rgb]{ .31,  .506,  .741}{1}} & \multicolumn{1}{r|}{\cellcolor[rgb]{ .31,  .506,  .741}\textcolor[rgb]{ .31,  .506,  .741}{1}} & \multicolumn{1}{r|}{\cellcolor[rgb]{ .31,  .506,  .741}\textcolor[rgb]{ .31,  .506,  .741}{1}} & \multicolumn{1}{r|}{\cellcolor[rgb]{ .31,  .506,  .741}\textcolor[rgb]{ .31,  .506,  .741}{1}} & \multicolumn{1}{r|}{\cellcolor[rgb]{ .31,  .506,  .741}\textcolor[rgb]{ .31,  .506,  .741}{1}} & \multicolumn{1}{r|}{\cellcolor[rgb]{ .31,  .506,  .741}\textcolor[rgb]{ .31,  .506,  .741}{1}} & \multicolumn{1}{r|}{\cellcolor[rgb]{ .31,  .506,  .741}\textcolor[rgb]{ .31,  .506,  .741}{1}} \\
				\cmidrule{2-10}    \multicolumn{1}{l|}{Ecuador} & \cellcolor[rgb]{ .608,  .733,  .349}\textcolor[rgb]{ .608,  .733,  .349}{3} & \multicolumn{1}{r|}{\cellcolor[rgb]{ .608,  .733,  .349}\textcolor[rgb]{ .608,  .733,  .349}{3}} & \multicolumn{1}{r|}{\cellcolor[rgb]{ .969,  .588,  .275}\textcolor[rgb]{ .969,  .588,  .275}{2}} & \multicolumn{1}{r|}{\cellcolor[rgb]{ .31,  .506,  .741}\textcolor[rgb]{ .31,  .506,  .741}{1}} & \multicolumn{1}{r|}{\cellcolor[rgb]{ .31,  .506,  .741}\textcolor[rgb]{ .31,  .506,  .741}{1}} & \multicolumn{1}{r|}{\cellcolor[rgb]{ .31,  .506,  .741}\textcolor[rgb]{ .31,  .506,  .741}{1}} & \multicolumn{1}{r|}{\cellcolor[rgb]{ .31,  .506,  .741}\textcolor[rgb]{ .31,  .506,  .741}{1}} & \multicolumn{1}{r|}{\cellcolor[rgb]{ .31,  .506,  .741}\textcolor[rgb]{ .31,  .506,  .741}{1}} & \multicolumn{1}{r|}{\cellcolor[rgb]{ .31,  .506,  .741}\textcolor[rgb]{ .31,  .506,  .741}{1}} \\
				\cmidrule{2-10}    \multicolumn{1}{l|}{El Salvador} & \cellcolor[rgb]{ .31,  .506,  .741}\textcolor[rgb]{ .31,  .506,  .741}{1} & \multicolumn{1}{r|}{\cellcolor[rgb]{ .31,  .506,  .741}\textcolor[rgb]{ .31,  .506,  .741}{1}} & \multicolumn{1}{r|}{\cellcolor[rgb]{ .31,  .506,  .741}\textcolor[rgb]{ .31,  .506,  .741}{1}} & \multicolumn{1}{r|}{\cellcolor[rgb]{ .31,  .506,  .741}\textcolor[rgb]{ .31,  .506,  .741}{1}} & \multicolumn{1}{r|}{\cellcolor[rgb]{ .31,  .506,  .741}\textcolor[rgb]{ .31,  .506,  .741}{1}} & \multicolumn{1}{r|}{\cellcolor[rgb]{ .31,  .506,  .741}\textcolor[rgb]{ .31,  .506,  .741}{1}} & \multicolumn{1}{r|}{\cellcolor[rgb]{ .31,  .506,  .741}\textcolor[rgb]{ .31,  .506,  .741}{1}} & \multicolumn{1}{r|}{\cellcolor[rgb]{ .31,  .506,  .741}\textcolor[rgb]{ .31,  .506,  .741}{1}} & \multicolumn{1}{r|}{\cellcolor[rgb]{ .31,  .506,  .741}\textcolor[rgb]{ .31,  .506,  .741}{1}} \\
				\cmidrule{2-10}    \multicolumn{1}{l|}{Guatemala} & \cellcolor[rgb]{ .969,  .588,  .275}\textcolor[rgb]{ .969,  .588,  .275}{2} & \multicolumn{1}{r|}{\cellcolor[rgb]{ .969,  .588,  .275}\textcolor[rgb]{ .969,  .588,  .275}{2}} & \multicolumn{1}{r|}{\cellcolor[rgb]{ .608,  .733,  .349}\textcolor[rgb]{ .608,  .733,  .349}{3}} & \multicolumn{1}{r|}{\cellcolor[rgb]{ .969,  .588,  .275}\textcolor[rgb]{ .969,  .588,  .275}{2}} & \multicolumn{1}{r|}{\cellcolor[rgb]{ .608,  .733,  .349}\textcolor[rgb]{ .608,  .733,  .349}{3}} & \multicolumn{1}{r|}{\cellcolor[rgb]{ .608,  .733,  .349}\textcolor[rgb]{ .608,  .733,  .349}{3}} & \multicolumn{1}{r|}{\cellcolor[rgb]{ .31,  .506,  .741}\textcolor[rgb]{ .31,  .506,  .741}{1}} & \multicolumn{1}{r|}{\cellcolor[rgb]{ .608,  .733,  .349}\textcolor[rgb]{ .608,  .733,  .349}{3}} & \multicolumn{1}{r|}{\cellcolor[rgb]{ .969,  .588,  .275}\textcolor[rgb]{ .969,  .588,  .275}{2}} \\
				\cmidrule{2-10}    \multicolumn{1}{l|}{Honduras} & \cellcolor[rgb]{ .31,  .506,  .741}\textcolor[rgb]{ .31,  .506,  .741}{1} & \multicolumn{1}{r|}{\cellcolor[rgb]{ .31,  .506,  .741}\textcolor[rgb]{ .31,  .506,  .741}{1}} & \multicolumn{1}{r|}{\cellcolor[rgb]{ .31,  .506,  .741}\textcolor[rgb]{ .31,  .506,  .741}{1}} & \multicolumn{1}{r|}{\cellcolor[rgb]{ .31,  .506,  .741}\textcolor[rgb]{ .31,  .506,  .741}{1}} & \multicolumn{1}{r|}{\cellcolor[rgb]{ .31,  .506,  .741}\textcolor[rgb]{ .31,  .506,  .741}{1}} & \multicolumn{1}{r|}{\cellcolor[rgb]{ .31,  .506,  .741}\textcolor[rgb]{ .31,  .506,  .741}{1}} & \multicolumn{1}{r|}{\cellcolor[rgb]{ .31,  .506,  .741}\textcolor[rgb]{ .31,  .506,  .741}{1}} & \multicolumn{1}{r|}{\cellcolor[rgb]{ .31,  .506,  .741}\textcolor[rgb]{ .31,  .506,  .741}{1}} & \multicolumn{1}{r|}{\cellcolor[rgb]{ .31,  .506,  .741}\textcolor[rgb]{ .31,  .506,  .741}{1}} \\
				\cmidrule{2-10}    \multicolumn{1}{l|}{Mexico} & \cellcolor[rgb]{ .969,  .588,  .275}\textcolor[rgb]{ .969,  .588,  .275}{2} & \multicolumn{1}{r|}{\cellcolor[rgb]{ .969,  .588,  .275}\textcolor[rgb]{ .969,  .588,  .275}{2}} & \multicolumn{1}{r|}{\cellcolor[rgb]{ .31,  .506,  .741}\textcolor[rgb]{ .31,  .506,  .741}{1}} & \multicolumn{1}{r|}{\cellcolor[rgb]{ .969,  .588,  .275}\textcolor[rgb]{ .969,  .588,  .275}{2}} & \multicolumn{1}{r|}{\cellcolor[rgb]{ .969,  .588,  .275}\textcolor[rgb]{ .969,  .588,  .275}{2}} & \multicolumn{1}{r|}{\cellcolor[rgb]{ .31,  .506,  .741}\textcolor[rgb]{ .31,  .506,  .741}{1}} & \multicolumn{1}{r|}{\cellcolor[rgb]{ .969,  .588,  .275}\textcolor[rgb]{ .969,  .588,  .275}{2}} & \multicolumn{1}{r|}{\cellcolor[rgb]{ .608,  .733,  .349}\textcolor[rgb]{ .608,  .733,  .349}{3}} & \multicolumn{1}{r|}{\cellcolor[rgb]{ .969,  .588,  .275}\textcolor[rgb]{ .969,  .588,  .275}{2}} \\
				\cmidrule{2-10}    \multicolumn{1}{l|}{Nicaragua} & \cellcolor[rgb]{ .969,  .588,  .275}\textcolor[rgb]{ .969,  .588,  .275}{2} & \multicolumn{1}{r|}{\cellcolor[rgb]{ .31,  .506,  .741}\textcolor[rgb]{ .31,  .506,  .741}{1}} & \multicolumn{1}{r|}{\cellcolor[rgb]{ .608,  .733,  .349}\textcolor[rgb]{ .608,  .733,  .349}{3}} & \multicolumn{1}{r|}{\cellcolor[rgb]{ .969,  .588,  .275}\textcolor[rgb]{ .969,  .588,  .275}{2}} & \multicolumn{1}{r|}{\cellcolor[rgb]{ .31,  .506,  .741}\textcolor[rgb]{ .31,  .506,  .741}{1}} & \multicolumn{1}{r|}{\cellcolor[rgb]{ .608,  .733,  .349}\textcolor[rgb]{ .608,  .733,  .349}{3}} & \multicolumn{1}{r|}{\cellcolor[rgb]{ .608,  .733,  .349}\textcolor[rgb]{ .608,  .733,  .349}{3}} & \multicolumn{1}{r|}{\cellcolor[rgb]{ .608,  .733,  .349}\textcolor[rgb]{ .608,  .733,  .349}{3}} & \multicolumn{1}{r|}{\cellcolor[rgb]{ .969,  .588,  .275}\textcolor[rgb]{ .969,  .588,  .275}{2}} \\
				\cmidrule{2-10}    \multicolumn{1}{l|}{Panama} & \cellcolor[rgb]{ .608,  .733,  .349}\textcolor[rgb]{ .608,  .733,  .349}{3} & \multicolumn{1}{r|}{\cellcolor[rgb]{ .608,  .733,  .349}\textcolor[rgb]{ .608,  .733,  .349}{3}} & \multicolumn{1}{r|}{\cellcolor[rgb]{ .608,  .733,  .349}\textcolor[rgb]{ .608,  .733,  .349}{3}} & \multicolumn{1}{r|}{\cellcolor[rgb]{ .608,  .733,  .349}\textcolor[rgb]{ .608,  .733,  .349}{3}} & \multicolumn{1}{r|}{\cellcolor[rgb]{ .31,  .506,  .741}\textcolor[rgb]{ .31,  .506,  .741}{1}} & \multicolumn{1}{r|}{\cellcolor[rgb]{ .31,  .506,  .741}\textcolor[rgb]{ .31,  .506,  .741}{1}} & \multicolumn{1}{r|}{\cellcolor[rgb]{ .31,  .506,  .741}\textcolor[rgb]{ .31,  .506,  .741}{1}} & \multicolumn{1}{r|}{\cellcolor[rgb]{ .31,  .506,  .741}\textcolor[rgb]{ .31,  .506,  .741}{1}} & \multicolumn{1}{r|}{\cellcolor[rgb]{ .31,  .506,  .741}\textcolor[rgb]{ .31,  .506,  .741}{1}} \\
				\cmidrule{2-10}    \multicolumn{1}{l|}{Paraguay} & \cellcolor[rgb]{ .31,  .506,  .741}\textcolor[rgb]{ .31,  .506,  .741}{1} & \multicolumn{1}{r|}{\cellcolor[rgb]{ .31,  .506,  .741}\textcolor[rgb]{ .31,  .506,  .741}{1}} & \multicolumn{1}{r|}{\cellcolor[rgb]{ .31,  .506,  .741}\textcolor[rgb]{ .31,  .506,  .741}{1}} & \multicolumn{1}{r|}{\cellcolor[rgb]{ .31,  .506,  .741}\textcolor[rgb]{ .31,  .506,  .741}{1}} & \multicolumn{1}{r|}{\cellcolor[rgb]{ .31,  .506,  .741}\textcolor[rgb]{ .31,  .506,  .741}{1}} & \multicolumn{1}{r|}{\cellcolor[rgb]{ .31,  .506,  .741}\textcolor[rgb]{ .31,  .506,  .741}{1}} & \multicolumn{1}{r|}{\cellcolor[rgb]{ .31,  .506,  .741}\textcolor[rgb]{ .31,  .506,  .741}{1}} & \multicolumn{1}{r|}{\cellcolor[rgb]{ .31,  .506,  .741}\textcolor[rgb]{ .31,  .506,  .741}{1}} & \multicolumn{1}{r|}{\cellcolor[rgb]{ .31,  .506,  .741}\textcolor[rgb]{ .31,  .506,  .741}{1}} \\
				\cmidrule{2-10}    \multicolumn{1}{l|}{Peru} & \cellcolor[rgb]{ .969,  .588,  .275}\textcolor[rgb]{ .969,  .588,  .275}{2} & \multicolumn{1}{r|}{\cellcolor[rgb]{ .31,  .506,  .741}\textcolor[rgb]{ .31,  .506,  .741}{1}} & \multicolumn{1}{r|}{\cellcolor[rgb]{ .31,  .506,  .741}\textcolor[rgb]{ .31,  .506,  .741}{1}} & \multicolumn{1}{r|}{\cellcolor[rgb]{ .31,  .506,  .741}\textcolor[rgb]{ .31,  .506,  .741}{1}} & \multicolumn{1}{r|}{\cellcolor[rgb]{ .31,  .506,  .741}\textcolor[rgb]{ .31,  .506,  .741}{1}} & \multicolumn{1}{r|}{\cellcolor[rgb]{ .31,  .506,  .741}\textcolor[rgb]{ .31,  .506,  .741}{1}} & \multicolumn{1}{r|}{\cellcolor[rgb]{ .31,  .506,  .741}\textcolor[rgb]{ .31,  .506,  .741}{1}} & \multicolumn{1}{r|}{\cellcolor[rgb]{ .31,  .506,  .741}\textcolor[rgb]{ .31,  .506,  .741}{1}} & \multicolumn{1}{r|}{\cellcolor[rgb]{ .31,  .506,  .741}\textcolor[rgb]{ .31,  .506,  .741}{1}} \\
				\cmidrule{2-10}    \multicolumn{1}{l|}{Uruguay} & \cellcolor[rgb]{ .31,  .506,  .741}\textcolor[rgb]{ .31,  .506,  .741}{1} & \multicolumn{1}{r|}{\cellcolor[rgb]{ .31,  .506,  .741}\textcolor[rgb]{ .31,  .506,  .741}{1}} & \multicolumn{1}{r|}{\cellcolor[rgb]{ .31,  .506,  .741}\textcolor[rgb]{ .31,  .506,  .741}{1}} & \multicolumn{1}{r|}{\cellcolor[rgb]{ .31,  .506,  .741}\textcolor[rgb]{ .31,  .506,  .741}{1}} & \multicolumn{1}{r|}{\cellcolor[rgb]{ .31,  .506,  .741}\textcolor[rgb]{ .31,  .506,  .741}{1}} & \multicolumn{1}{r|}{\cellcolor[rgb]{ .31,  .506,  .741}\textcolor[rgb]{ .31,  .506,  .741}{1}} & \multicolumn{1}{r|}{\cellcolor[rgb]{ .31,  .506,  .741}\textcolor[rgb]{ .31,  .506,  .741}{1}} & \multicolumn{1}{r|}{\cellcolor[rgb]{ .31,  .506,  .741}\textcolor[rgb]{ .31,  .506,  .741}{1}} & \multicolumn{1}{r|}{\cellcolor[rgb]{ .31,  .506,  .741}\textcolor[rgb]{ .31,  .506,  .741}{1}} \\
				\cmidrule{2-10}    \multicolumn{1}{l|}{Venezuela} & \cellcolor[rgb]{ .31,  .506,  .741}\textcolor[rgb]{ .31,  .506,  .741}{1} & \multicolumn{1}{r|}{\cellcolor[rgb]{ .31,  .506,  .741}\textcolor[rgb]{ .31,  .506,  .741}{1}} & \multicolumn{1}{r|}{\cellcolor[rgb]{ .31,  .506,  .741}\textcolor[rgb]{ .31,  .506,  .741}{1}} & \multicolumn{1}{r|}{\cellcolor[rgb]{ .31,  .506,  .741}\textcolor[rgb]{ .31,  .506,  .741}{1}} & \multicolumn{1}{r|}{\cellcolor[rgb]{ .31,  .506,  .741}\textcolor[rgb]{ .31,  .506,  .741}{1}} & \multicolumn{1}{r|}{\cellcolor[rgb]{ .31,  .506,  .741}\textcolor[rgb]{ .31,  .506,  .741}{1}} & \multicolumn{1}{r|}{\cellcolor[rgb]{ .31,  .506,  .741}\textcolor[rgb]{ .31,  .506,  .741}{1}} & \multicolumn{1}{r|}{\cellcolor[rgb]{ .969,  .588,  .275}\textcolor[rgb]{ .969,  .588,  .275}{2}} & \multicolumn{1}{r|}{\cellcolor[rgb]{ .608,  .733,  .349}\textcolor[rgb]{ .608,  .733,  .349}{3}} \\
				\cmidrule{2-10}    \multicolumn{1}{r}{} & \multicolumn{1}{r}{} &   &   &   &   &   &   &   &  \\
				\cmidrule{2-2}      & \cellcolor[rgb]{ .31,  .506,  .741} & \multicolumn{8}{l}{Both perceptions and inequality data available} \\
				\cmidrule{2-2}      & \cellcolor[rgb]{ .969,  .588,  .275} & \multicolumn{8}{l}{Inequality was calculated with a close survey} \\
				\cmidrule{2-2}      & \cellcolor[rgb]{ .608,  .733,  .349} & \multicolumn{8}{l}{Inequality was calculated with a linear interpolation} \\
				\cmidrule{2-2}      & \cellcolor[rgb]{ .749,  .749,  .749} & \multicolumn{8}{l}{Latinobar\'ometro did not conduct survey in this year} \\
				\cmidrule{2-2}    \end{tabular}%
		\end{center}
		\begin{singlespace}  \vspace{-.5cm}
			\noindent \justify %\textbf{Notes:}
		\end{singlespace}
	}
\end{table}

\subsection{Imputation of Missing Values for the Regression Analysis}

Two of our individual-level variables (political ideology and religion) have many missing values in some country-years. To deal with this in our regressions, we imputed the average value of each variable to individuals with a missing value. In those cases, we included in the regression a dummy that takes the value one if the value of the variable was imputed and zero otherwise. The results are similar if we do not impute the values, but the sample size of the regressions is smaller.

\subsection{Comparison between Latinobar\'ometro's and SEDLAC's samples} \label{app_lb_sedlac}

To assess whether there are systematic differences between Latinobar\'ometro's sample and the household surveys' sample, in Appendix Table \ref{tab-lat-sedlac} we compare a set of variables available in both datasets during 2013. To ensure comparability across databases, we restrict the calculations to individuals over age 18 and countries with data available in both harmonization projects.

In general, the samples are similar in observable characteristics. For instance, the average age in Latinobar\'ometro's 2013 sample is 40.6 years, while in SEDLAC it is 42.7 years. Similarly, the percentage of males is 48.9\% in Latinobar\'ometro and 47.6\% in SEDLAC. The main difference arises from educational attainment. On average, the SEDLAC sample is more educated: 46.1\% of the population has secondary education or more, while this figure is 38.8\% in Latinobar\'ometro.


% Comparison of descriptive statistics in Latinobar\'ometro and SEDLAC, 2013
\begin{table}[H]{\scriptsize
		\begin{center}
			\caption{Descriptive statistics in Latinobar\'ometro and SEDLAC, 2013 (selected countries)} \label{tab-lat-sedlac}
			\begin{tabular}{lcccccccc}
				\midrule
				& \multicolumn{2}{c}{Mean} &   & \multicolumn{2}{c}{Standard Dev.} &   & \multicolumn{2}{c}{Observations} \\
				\cmidrule{2-3}\cmidrule{5-6}\cmidrule{8-9}      &  Latinob.  &  SEDLAC  &   &  Latinob.  &  SEDLAC  &   &  Latinob.  &  SEDLAC  \\
				& (1) & (2) &   & (3) & (4) &   & (5) & (6) \\
				\midrule
				\textbf{\hspace{-1em} Panel A. Sociodemographic} &   &   &   &   &   &   &   &  \\
				Age & 40.59 & 42.68 &   & 16.43 & 17.25 &   &      14,855  &   1,004,894  \\
				Male (\%) & 48.97 & 47.63 &   & 0.50 & 0.50 &   &      14,855  &   1,004,894  \\
				Married or civil union (\%) & 56.77 & 36.41 &   & 0.50 & 0.48 &   &      14,804  &      915,117  \\
				\multicolumn{3}{l}{\textbf{\hspace{-1em} Panel B. Education and Labor market}} &   &   &   &   &   &  \\
				Literate (\%) & 91.18 & 92.17 &   & 0.28 & 0.27 &   &      14,855  &   1,004,744  \\
				Secondary education or more (\%) & 38.83 & 46.11 &   & 0.49 & 0.50 &   &      14,855  &   1,001,672  \\
				Economically active (\%) & 65.14 & 68.66 &   & 0.48 & 0.46 &   &      14,855  &   1,004,718  \\
				Unemployed (\%) & 5.78 & 4.08 &   & 0.23 & 0.20 &   &      14,855  &   1,004,718  \\
				\textbf{\hspace{-1em} Panel C. Assets and Services} &   &   &   &   &   &   &   &  \\
				Access to a sewerage (\%) & 68.76 & 63.41 &   & 0.46 & 0.48 &   &      13,799  &      975,726  \\
				Car (\%) & 26.37 & 21.09 &   & 0.44 & 0.41 &   &      11,612  &      643,350  \\
				Computer (\%) & 46.55 & 47.82 &   & 0.50 & 0.50 &   &      12,747  &      894,003  \\
				Fridge (\%) & 82.76 & 88.89 &   & 0.38 & 0.31 &   &      12,763  &      894,003  \\
				Homeowner (\%) & 74.09 & 69.64 &   & 0.44 & 0.46 &   &      14,761  &   1,003,306  \\
				Mobile (\%) & 86.91 & 91.78 &   & 0.34 & 0.27 &   &      12,754  &      896,079  \\
				Washing machine (\%) & 60.49 & 56.88 &   & 0.49 & 0.50 &   &      11,816  &      848,350  \\
				Landline (\%) & 40.22 & 39.47 &   & 0.49 & 0.49 &   &      12,736  &      896,425  \\
				\midrule
			\end{tabular}
		\end{center}
		\begin{singlespace}
			\noindent \justify \textbf{Note:} This table compares the observable characteristics of individuals in Latinobar\'ometro and SEDLAC. Summary statistics were calculated on a restricted sample (individuals aged over 18) to ensure comparability between both datasets, pooling data from 14 countries in 2013: Argentina, Bolivia, Brazil, Chile, Colombia, Costa Rica, Dominican Republic, Ecuador, El Salvador, Honduras, Panama, Peru, Paraguay, and Uruguay.
		\end{singlespace}

	}
\end{table}


%%%%%%%%%%%%%%%%%%%%%%%%%%%%%%%%%%%%%%%%%%%%%%%%%%%%%%%%%%%%%%%%%%%%%%%%%%%%%%%
% APPENDIX C: OAXACA-BLINDER DECOMPOSITION
%%%%%%%%%%%%%%%%%%%%%%%%%%%%%%%%%%%%%%%%%%%%%%%%%%%%%%%%%%%%%%%%%%%%%%%%%%%%%%%

\clearpage
\section{The Oaxaca-Blinder Decomposition} \label{sec_oaxaca}

\setcounter{table}{0}
\setcounter{figure}{0}
\setcounter{equation}{0}
\renewcommand{\thetable}{C\arabic{table}}
\renewcommand{\thefigure}{C\arabic{figure}}
\renewcommand{\theequation}{C\arabic{equation}}

The starting point to decompose changes in unfairness perceptions between 2002 and 2013 is the following equation:
%
\begin{align}
	\text{Unfair}_{ict} = \beta_t X_{ict} + \gamma_t \text{Gini}_{ct} + \varepsilon_{ict} \quad \text{for} \quad t \in \{2002, 2013\},
\end{align}
%
where $t$ indicates the year in which perceptions are elicited and $X_{ict}$ is a vector that contains individual-level controls. The fraction of individuals who perceive the income distribution as unfair in year $t$ can be calculated as
%
\begin{align}
	\overline{\text{Unfair}}_{t} = \hat{\beta}_{t} \bar{X}_{t} + \hat{\gamma}_{t} \overline{\text{Gini}}_{t}  \quad \text{for} \quad t \in \{2002, 2013\},
\end{align}
%
where $\bar{X}_t$ is a vector of the average values of the explanatory variables in year $t$, and $\hat{\beta}$ the vector of OLS-estimated coefficients. The change in unfairness beliefs between 2013 and 2002 is given by
%
\begin{align} \label{eq-diff-fair}
	\underbrace{\overline{\text{Unfair}}_{2013} - \overline{\text{Unfair}}_{2013}}_{\equiv \Delta \text{Unfair}} = (\hat{\beta}_{2013} \bar{X}_{2013} + \hat{\gamma}_{2013} \overline{\text{Gini}}_{2013}) - (\hat{\beta}_{2002} \bar{X}_{2002} + \hat{\gamma}_{2002} \overline{\text{Gini}}_{2002})
\end{align}

Adding and subtracting $\hat{\beta}_{2002} \bar{X}_{2013} + \hat{\gamma}_{2002} \overline{\text{Gini}}_{2013}$ to equation \eqref{eq-diff-fair} yields
%
\begin{align} \label{eq_oaxaca}
	\Delta \text{Unfair} &=
	\underbrace{\hat{\beta}_{2002} (\bar{X}_{2013} - \bar{X}_{2002})}_{\equiv \Delta \text{Demog.}} +
	\underbrace{\hat{\gamma}_{2002} (\overline{\text{Gini}}_{2013} - \overline{\text{Gini}}_{2002})}_{\equiv \Delta \text{Gini}} \notag \\ &+ \underbrace{\bar{X}_{2013} (\hat{\beta}_{2013} - \hat{\beta}_{2002})+ \overline{\text{Gini}}_{2013} (\hat{\gamma}_{2013} - \hat{\gamma}_{2002})}_{\text{Residual}}
\end{align}
%

The first two terms of equation \eqref{eq_oaxaca} are usually known as the ``composition effect.'' These effects capture the difference between the average perceptions in 2002 and the counterfactual perceptions 2013 had the $\hat{\beta}$'s and $\hat{\gamma}$---i.e., the elasticity of perceptions to the different covariables---remained constant during the 2002--13 period. The first term captures differences in individual-level demographic variables that determine unfairness perceptions in the model (such as educational attainment, age, and employment status). The second term captures changes in aggregate trends in income inequality.

The third term of \eqref{eq_oaxaca} reflects the difference between the average fairness views in 2013 and the counterfactual fairness views in 2002 with the observable attributes of 2013. Thus, this component reflects changes in fairness views due to changes in the elasticity of the different covariables between both years. Since we cannot explain why the coefficients attached to each variable changed, this term is usually viewed as the ``unexplained'' part of the decomposition and treated as the residual of the decomposition.


% End standalone document
\ifdefined\standalonetrue
  \end{document}
\fi
 after \appendix in main document

\ifdefined\appendixstandalone
\else
  % Standalone preamble - only used when compiling this file directly
  \newif\ifstandalone
  \standalonetrue

  \documentclass[11pt]{article}
  \usepackage[a4paper,pdftex]{geometry}
  \setlength{\oddsidemargin}{3mm}
  \setlength{\evensidemargin}{3mm}
  \usepackage[T1]{fontenc}
  \usepackage[utf8x]{inputenc}
  \usepackage{amsmath,amsthm,amssymb,graphicx,enumerate,booktabs,bigstrut,rotating,multirow,float,caption,etoolbox,hyperref,lmodern,comment,listings,qtree,a4wide,titlesec,moresize,bbm,subcaption,tikz,pdfescape,mathtools,flafter,enumitem,setspace,ragged2e,colortbl,lineno,lscape}
  \hypersetup{colorlinks,linkcolor={red},citecolor={blue},urlcolor={blue}}

  \usepackage[justification=centering,font=small]{caption}

  \usepackage[authoryear]{natbib}
  \usepackage[english]{babel}
  \renewcommand{\baselinestretch}{1.25}
  \DeclareMathOperator{\E}{\mathbb{E}}
  \DeclareMathOperator*{\argmax}{argmax}
  \makeatletter
  \renewcommand\subsubsection{\@startsection{subsubsection}{3}{\z@}%
  	{-3.25ex\@plus -1ex \@minus -.2ex}%
  	{-1.5ex \@plus -.2ex}%
  	{\normalfont\normalsize\bfseries}}
  \def\@biblabel#1{\hspace*{-\labelsep}}

  \newcommand*\circled[1]{\tikz[baseline=(char.base)]{
  		\node[shape=circle,draw,inner sep=2pt] (char) {#1};}}
  \newcommand*\ExpandableInput[1]{\@@input#1 }
  \makeatother

  \def\sym#1{\ifmmode^{#1}\else\(^{#1}\)\fi}
  \onehalfspacing
  \pdfpageheight\paperheight
  \pdfpagewidth\paperwidth

  \newcolumntype{L}[1]{>{\raggedright\let\newline\\\arraybackslash\hspace{0pt}}m{#1}}
  \newcolumntype{C}[1]{>{\centering\let\newline\\\arraybackslash\hspace{0pt}}m{#1}}
  \newcolumntype{R}[1]{>{\raggedleft\let\newline\\\arraybackslash\hspace{0pt}}m{#1}}

  \begin{document}

  \title{Are Fairness Perceptions Shaped by Income Inequality? \\ Evidence from Latin America \\ \vspace{10pt} \Large Online Appendix}
  \author{Leonardo Gasparini \and Germ\'an Reyes}
  \date{\vspace{15pt} February 2022}
  \maketitle

  \appendix
\fi

%%%%%%%%%%%%%%%%%%%%%%%%%%%%%%%%%%%%%%%%%%%%%%%%%%%%%%%%%%%%%%%%%%%%%%%%%%%%%%%
% APPENDIX A: ADDITIONAL FIGURES AND TABLES
%%%%%%%%%%%%%%%%%%%%%%%%%%%%%%%%%%%%%%%%%%%%%%%%%%%%%%%%%%%%%%%%%%%%%%%%%%%%%%%

\begin{center}\noindent{\LARGE \textbf{Appendix}}\end{center}

\section{Additional Figures and Tables} \label{app_add_figs}

\setcounter{table}{0}
\setcounter{figure}{0}
\setcounter{equation}{0}
\renewcommand{\thetable}{A\arabic{table}}
\renewcommand{\thefigure}{A\arabic{figure}}
\renewcommand{\theequation}{A\arabic{equation}}


\begin{figure}[htpb]
	\caption{Perceptions of unfairness and individual characteristics, 1997--2015} \label{fig-fairness-grps}
	\centering
	\begin{subfigure}[t]{.48\textwidth}
		\caption*{Panel A. By age}\label{fig-fairness-age}
		\centering
		\includegraphics[width=\linewidth]{../results/fig-fairness-age}
	\end{subfigure}
	\hfill
	\begin{subfigure}[t]{0.48\textwidth}
		\caption*{Panel B. By sex}\label{fig-fairness-sex}
		\centering
		\includegraphics[width=\linewidth]{../results/fig-fairness-sex}
	\end{subfigure}
	\hfill
	\begin{subfigure}[t]{.48\textwidth}
		\caption*{Panel C. By educational attainment}\label{fig-fairness-educ}
		\centering
		\includegraphics[width=\linewidth]{../results/fig-fairness-educ}
	\end{subfigure}
	\hfill
	\begin{subfigure}[t]{0.48\textwidth}
		\caption*{Panel D. By employment status}\label{fig-fairness-employ}
		\centering
		\includegraphics[width=\linewidth]{../results/fig-fairness-employ}
	\end{subfigure}
	\hfill
	{\footnotesize
		\singlespacing \justify

		\textbf{Notes:} This figure shows the share of individuals who perceive the income distribution as unfair or very unfair according to their age, gender, maximum educational attainment, and employment status.


	}
\end{figure}

\clearpage
\begin{figure}[htp]
	\caption{The evolution of fairness views and income inequality in Latin America}\label{fig-timeseries-gini-unfair}  \centering
	\centering
	\includegraphics[width=.75\linewidth]{../results/fig-timeseries-gini-unfair}
	\hfill
	{\footnotesize
		\singlespacing \justify

		\textbf{Notes:} This figure shows the evolution of the average Gini coefficient across countries in our sample (right-hand-side variable) and the fraction of the population who perceive the income distribution as unfair or very unfair (left-hand-side variable) over 1997--2015. To have a balanced panel of countries over time, we linearly extrapolated the Gini coefficient in years in which income microdata is not available (see Appendix \ref{sec_data}).

	}
\end{figure}



\clearpage
\begin{table}[H]{\footnotesize
		\begin{center}
			\caption{Descriptive statistics of our sample} \label{tab-sum-stats}
			\begin{tabular}{lccc}
				\midrule
				&  Mean  &  Standard Dev.  & Observations \\
				& (1) & (2) & (3) \\
				\midrule
				\textbf{\hspace{-1em} Panel A. Sociodemographic} &   &   &  \\
				Age & 39.75 & 16.23 &         225,551  \\
				Male (\%) & 48.97 & 0.50 &         225,567  \\
				Married or civil union (\%) & 56.27 & 0.50 &         224,081  \\
				Catholic religion (\%) & 68.01 & 0.47 &         222,790  \\
				Ideology (10 = right-wing) & 5.48 & 2.64 &         131,980  \\
				\textbf{\hspace{-1em} Panel B. Education and Labor market} &   &   &  \\
				Literate (\%) & 90.31 & 0.30 &         224,056  \\
				Secondary education or more (\%) & 33.65 & 0.47 &         224,056  \\
				Parents with secondary education (\%) & 17.43 & 0.38 &         184,884  \\
				Economically active (\%) & 64.14 & 0.48 &         225,222  \\
				Unemployed (\% Labor Force) & 9.89 & 0.30 &         225,222  \\
				\textbf{\hspace{-1em} Panel C. Access to services} &   &   &  \\
				Access to a sewerage (\%) & 69.59 & 0.46 &         222,530  \\
				Access to running water (\%) & 88.83 & 0.31 &         204,340  \\
				\textbf{\hspace{-1em} Panel D. Asset ownership} &   &   &  \\
				Car (\%) & 28.21 & 0.45 &         222,338  \\
				Computer (\%) & 33.79 & 0.47 &         222,645  \\
				Fridge (\%) & 79.22 & 0.41 &         146,686  \\
				Homeowner (\%) & 73.92 & 0.44 &         223,603  \\
				Mobile (\%) & 80.61 & 0.40 &         172,253  \\
				Washing machine (\%) & 54.71 & 0.50 &         223,122  \\
				Landline (\%) & 42.28 & 0.49 &         222,968  \\
				\midrule
			\end{tabular}
		\end{center}
		\begin{singlespace} \vspace{-.5cm}
			\noindent \justify \textbf{Note:} This table shows summary statistics on our sample pooling data from all countries in our sample over 1997--2015.
		\end{singlespace}
	}
\end{table}


% Fairness perceptions by population group pooling data 1997-2015, in %
\clearpage
\begin{table}[H]{\footnotesize
		\begin{center}
			\caption{Fairness views by population group} \label{tab-fairness-grp}
			\newcommand\w{1.70}
			\begin{tabular}{l@{}R{\w cm}R{\w cm}R{\w cm}R{\w cm}}
				\midrule
				& \multicolumn{4}{c}{\% of individuals who believe income distribution is:} \\
				\cmidrule{2-5}  & Very unfair & Unfair & Fair & Very fair \\
				& (1) & (2) & (3) & (4) \\
				\midrule
				All & 28.2 & 51.6 & 17.3 & 2.9 \\
				\textbf{\hspace{-1em} Panel A. Gender} &   &   &   &  \\
				Female & 28.3 & 52.2 & 16.7 & 2.8 \\
				Male & 28.0 & 51.1 & 17.9 & 3.0 \\
				\textbf{\hspace{-1em} Panel B. Age group} &   &   &   &  \\
				15-24 & 25.2 & 52.0 & 19.7 & 3.1 \\
				25-40 & 28.5 & 51.3 & 17.2 & 3.0 \\
				41-64 & 29.5 & 51.7 & 16.0 & 2.8 \\
				65+ & 29.2 & 51.6 & 16.6 & 2.5 \\
				\textbf{\hspace{-1em} Panel C. Civil status} &   &   &   &  \\
				Married & 28.3 & 51.9 & 17.0 & 2.8 \\
				Not married & 27.9 & 51.4 & 17.7 & 3.1 \\
				\textbf{\hspace{-1em} Panel D. Religion} &   &   &   &  \\
				Catholic & 28.2 & 51.7 & 17.2 & 2.9 \\
				Not catholic & 28.0 & 51.5 & 17.5 & 3.0 \\
				\textbf{\hspace{-1em} Panel E. Education level} &   &   &   &  \\
				Less than Primary & 27.7 & 51.6 & 17.7 & 3.0 \\
				Complete Primary & 27.9 & 52.2 & 17.4 & 2.6 \\
				Complete Secondary & 29.1 & 53.2 & 15.0 & 2.7 \\
				Complete Tertiary & 29.0 & 50.8 & 17.1 & 3.1 \\
				\textbf{\hspace{-1em} Panel F. Type of employment} &   &   &   &  \\
				Employee & 28.3 & 51.5 & 17.2 & 2.9 \\
				Employer & 24.3 & 53.9 & 19.0 & 2.8 \\
				Self-employed & 28.0 & 51.4 & 17.5 & 3.1 \\
				Unemployed & 30.3 & 51.6 & 15.1 & 3.0 \\
				\textbf{\hspace{-1em} Panel E. Country} &   &   &   &  \\
				Argentina & 38.17 & 50.74 & 10.26 & 0.83 \\
				Bolivia & 18.01 & 56.13 & 23.39 & 2.48 \\
				Brazil & 31.95 & 53.71 & 12.85 & 1.49 \\
				Chile & 40.20 & 49.93 & 8.42 & 1.45 \\
				Colombia & 35.15 & 51.20 & 11.40 & 2.26 \\
				Costa Rica & 23.20 & 53.55 & 20.13 & 3.12 \\
				Dominican Rep. & 32.31 & 46.52 & 17.61 & 3.56 \\
				Ecuador & 21.45 & 47.46 & 27.58 & 3.51 \\
				El Salvador & 22.73 & 53.16 & 20.45 & 3.65 \\
				Guatemala & 28.29 & 51.34 & 16.70 & 3.66 \\
				Honduras & 28.87 & 53.42 & 14.33 & 3.38 \\
				Mexico & 32.15 & 49.75 & 15.32 & 2.78 \\
				Nicaragua & 18.69 & 51.88 & 24.33 & 5.11 \\
				Panama & 27.39 & 48.01 & 20.25 & 4.34 \\
				Paraguay & 38.31 & 48.80 & 10.95 & 1.93 \\
				Peru & 25.03 & 61.89 & 11.70 & 1.38 \\
				Uruguay & 18.22 & 57.51 & 22.64 & 1.64 \\
				Venezuela & 23.51 & 42.96 & 26.62 & 6.92 \\
				\midrule
			\end{tabular}
		\end{center}
		\begin{singlespace} \vspace{-.5cm}
			\noindent \justify \textbf{Note:} This table shows the fraction of individuals in our sample who perceive the income distribution as very unfair, unfair, fair, or very fair.
		\end{singlespace}
	}
\end{table}



\clearpage
\begin{table}[htpb!]{\footnotesize
		\begin{center}
			\caption{Logit regressions of unfairness perceptions (unfair) and individual characteristics} \label{unfair_logit}
			\newcommand\w{1.30}
			\begin{tabular}{l@{}lR{\w cm}@{}L{0.43cm}R{\w cm}@{}L{0.43cm}R{\w cm}@{}L{0.43cm}R{\w cm}@{}L{0.43cm}R{\w cm}@{}L{0.43cm}R{\w cm}@{}L{0.43cm}}
				\midrule
				&& 	\multicolumn{12}{c}{Dependent Variable: Believes income distribution is unfair or very unfair}  \\\cmidrule{3-14}
				%					&&          &&       &&          && Labor     &&  && Ideology, \\
				%					&& Baseline && Demog.   && Education && status  && Assets && Religion \\
				&& (1) && (2) && (3) && (4) && (5) && (6) \\
				\midrule
				\ExpandableInput{../results/unfair_logit}
				\midrule
			\end{tabular}
		\end{center}
		\begin{singlespace}  \vspace{-.5cm}
			\noindent \justify \textbf{Notes:} This table shows estimates of the relationship between an indicator that equals one for individuals who believe that the income distribution is unfair or very unfair and the Gini coefficient controlling for individuals' characteristics. Coefficients are estimated through Logit regressions and represent the marginal effects evaluated at the mean values of the rest of the variables. Observations are weighted by the individual's probability of being interviewed. All specifications include country and year fixed effects. $^{***}$, $^{**}$ and $^*$ denote significance at 10\%, 5\% and 1\% levels, respectively. Heteroskedasticity-robust standard errors clustered at the country-by-year level in parentheses.
		\end{singlespace}
	}
\end{table}


\clearpage
\begin{table}[htpb!]{\footnotesize
		\begin{center}
			\caption{Logit regressions of unfairness perceptions (very unfair) and different inequality indicators} \label{unfair_ineq_logit}
			\newcommand\w{1.50}
			\begin{tabular}{l@{}lR{\w cm}@{}L{0.43cm}R{\w cm}@{}L{0.43cm}R{\w cm}@{}L{0.43cm}R{\w cm}@{}L{0.43cm}R{\w cm}@{}L{0.43cm}}
				\midrule
				&& 	\multicolumn{10}{c}{Dependent Variable: Believes income distribution is very unfair}  \\\cmidrule{3-12}
				&& (1) && (2) && (3) && (4) && (5)  \\
				\midrule
				\ExpandableInput{../results/very_unfair_ineq_logit}
				\midrule
			\end{tabular}
		\end{center}
		\begin{singlespace}  \vspace{-.5cm}
			\noindent \justify \textbf{Notes:} This table shows estimates of the relationship between an indicator that equals one for individuals who believe income distribution is unfair or very unfair and several inequality indicators controlling for individuals' characteristics. Coefficients are estimated through Logit regressions and represent the marginal effects evaluated at the mean values of the rest of the variables. Observations are weighted by the individual's probability of being interviewed. All specifications include country and year fixed effects. $^{***}$, $^{**}$ and $^*$ denote significance at 10\%, 5\% and 1\% levels, respectively. Heteroskedasticity-robust standard errors clustered at the country-by-year level in parentheses.

		\end{singlespace}
	}
\end{table}




\begin{table}[htpb!]{\footnotesize
		\begin{center}
			\caption{OLS regressions of unfairness perceptions (very unfair) and individual characteristics} \label{very_unfair_lpm}
			\newcommand\w{1.30}
			\begin{tabular}{l@{}lR{\w cm}@{}L{0.43cm}R{\w cm}@{}L{0.43cm}R{\w cm}@{}L{0.43cm}R{\w cm}@{}L{0.43cm}R{\w cm}@{}L{0.43cm}R{\w cm}@{}L{0.43cm}}
				\midrule
				&& 	\multicolumn{12}{c}{Dependent Variable: Believes income distribution is very unfair}  \\\cmidrule{3-14}
				%					&&          &&       &&          && Labor     &&  && Ideology, \\
				%					&& Baseline && Demog.   && Education && status  && Assets && Religion \\
				&& (1) && (2) && (3) && (4) && (5) && (6) \\
				\midrule
				\ExpandableInput{../results/very_unfair_lpm}
				\midrule
			\end{tabular}
		\end{center}
		\begin{singlespace}  \vspace{-.5cm}
			\noindent \justify \textbf{Notes:} This table presents estimates of the correlation between a dummy variable that indicates if the individual believes income distribution is unfair or very unfair and the Gini coefficient controlling for individuals' characteristics. Coefficients are estimated through a linear probability model. Observations are weighted by the individual's probability of being interviewed. All specifications include country and year fixed effects. $^{***}$, $^{**}$ and $^*$ denote significance at 10\%, 5\% and 1\% levels, respectively. Heteroskedasticity-robust standard errors clustered at the country-by-year level in parentheses.
		\end{singlespace}
	}
\end{table}


%%%%%%%%%%%%%%%%%%%%%%%%%%%%%%%%%%%%%%%%%%%%%%%%%%%%%%%%%%%%%%%%%%%%%%%%%%%%%%%
% APPENDIX B: DATA APPENDIX
%%%%%%%%%%%%%%%%%%%%%%%%%%%%%%%%%%%%%%%%%%%%%%%%%%%%%%%%%%%%%%%%%%%%%%%%%%%%%%%

\clearpage
\section{Data Appendix} \label{sec_data}

\setcounter{table}{0}
\setcounter{figure}{0}
\renewcommand{\thetable}{B\arabic{table}}
\renewcommand{\thefigure}{B\arabic{figure}}

The figures presented in this paper are based on two harmonization projects, known as Latinobar\'ometro and SEDLAC (Socio-Economic Database for Latin America and the Caribbean). In this Appendix, we describe how we make both sources compatible.

Our perceptions data come from Latinobar\'ometro, which has conducted opinion surveys in 18 LA countries since the 1990s, interviewing about 1,200 individuals per country about individuals' socioeconomic background, and preferences towards political and social issues. Unfortunately, not all years contain questions about individuals' fairness perceptions. The survey was designed to be representative of the voting-age population at the national level (in most LA countries, individuals aged over 18). In Table \ref{tab-coverage} we show what percentage of the voting-age population is represented by the survey in each country for all the years in which the fairness question is available.



\begin{table}[htbp]{\footnotesize
		\begin{center}
			\caption{Coverage of each country's population in Latinobar\'ometro overtime (in \%)}\label{tab-coverage}
			\begin{tabular}{lrrrrrrrrr}
				\midrule
				& 1997 & 2001 & 2002 & 2007 & 2009 & 2010 & 2011 & 2013 & 2015 \\
				\midrule
				Argentina &         68  &         75  &         75  &       100  &       100  &       100  &       100  &       100  &       100  \\
				Bolivia &         32  &         52  &       100  &       100  &       100  &       100  &       100  &       100  &       100  \\
				Brazil &         12  &       100  &       100  &       100  &       100  &       100  &       100  &       100  &       100  \\
				Chile &         70  &         70  &         70  &       100  &       100  &       100  &       100  &       100  &       100  \\
				Colombia &         25  &         71  &         51  &       100  &       100  &       100  &       100  &       100  &       100  \\
				Costa Rica &       100  &       100  &       100  &       100  &       100  &       100  &       100  &       100  &       100  \\
				Dominican Republic & \multicolumn{1}{c}{ N/A } & \multicolumn{1}{c}{ N/A } & \multicolumn{1}{c}{ N/A } &       100  &       100  &       100  &       100  &       100  &       100  \\
				Ecuador &         97  &         97  &       100  &       100  &       100  &       100  &       100  &       100  &       100  \\
				El Salvador &         65  &       100  &       100  &       100  &       100  &       100  &       100  &       100  &       100  \\
				Guatemala &       100  &       100  &       100  &         97  &       100  &       100  &       100  &       100  &       100  \\
				Honduras &       100  &       100  &       100  &         98  &       100  &         99  &         99  &         99  &         99  \\
				Mexico &         93  &         88  &         95  &       100  &       100  &       100  &       100  &       100  &       100  \\
				Nicaragua &       100  &       100  &       100  &       100  &       100  &       100  &       100  &       100  &       100  \\
				Panama &       100  &       100  &       100  &         99  &         99  &         99  &         99  &         99  &         99  \\
				Paraguay &         46  &         46  &         46  &       100  &       100  &       100  &       100  &       100  &       100  \\
				Peru &         52  &         52  &       100  &       100  &       100  &       100  &       100  &       100  &       100  \\
				Uruguay &         80  &         80  &         80  &       100  &       100  &       100  &       100  &       100  &       100  \\
				Venezuela &       100  &       100  &       100  &       100  &         93  &       100  &       100  &       100  &       100  \\
				Weighted average &         68  &         86  &         91  &       100  &       100  &       100  &       100  &       100  &       100  \\
				\midrule
			\end{tabular}
		\end{center}
		%		\begin{singlespace}  \vspace{-.5cm}
			%			\noindent \justify \textbf{Notes:} This table presents the percentage of the voting-age population represented each year in Latinobar\'ometro overtime. The regional average is calculated by weighting each country's population. N/A means that Latinobar\'ometro did not conduct the opinion poll in that particular country-year.
			%		\end{singlespace}
	}
\end{table}


Since our goal is to analyze how unfairness perceptions evolved vis-\`a-vis changes in income inequality, we put a lot of effort into getting income inequality data for each data point for which we have perceptions data available. We made two partial fixes to increase the number of observations available (without pushing the data too much). First, we filled the data gaps using household surveys of relatively close years in which previously unused data were available (see Appendix Table \ref{tab-circa}). For instance, Chile conducts household surveys on average every two years. In 1997, there is perceptions data available, but no data on income inequality. Therefore, we use the inequality data from an adjacent year (1998). As noted previously, we only use data from close years if the data from the adjacent year correspond to a year in which the perceptions question was not asked (and therefore, inequality data are not needed in that year).

\begin{table}[htbp]{\footnotesize
		\begin{center}
			\caption{Circa years used to fill data gaps}\label{tab-circa}
			\begin{tabular}{lcc}
				\midrule
				Country & Year without household data & Data point used instead \\
				\midrule
				Chile & 1997 & 1998 \\
				Chile & 2001 & 2000 \\
				Chile & 2002 & 2003 \\
				Chile & 2007 & 2006 \\
				Colombia & 2007 & 2008 \\
				Ecuador & 2002 & 2003 \\
				El Salvador & 1997 & 1998 \\
				Guatemala & 2001 & 2000 \\
				Guatemala & 2015 & 2014 \\
				Mexico & 1997 & 1998 \\
				Mexico & 2001 & 2000 \\
				Mexico & 2007 & 2006 \\
				Mexico & 2009 & 2008 \\
				Mexico & 2011 & 2012 \\
				Mexico & 2015 & 2014 \\
				Nicaragua & 1997 & 1998 \\
				Nicaragua & 2007 & 2005 \\
				Nicaragua & 2015 & 2014 \\
				Venezuela & 2013 & 2012 \\
				\midrule
			\end{tabular}
		\end{center}
		\begin{singlespace}  \vspace{-.5cm}
			\noindent \justify %\textbf{Notes:}
		\end{singlespace}
	}
\end{table}

Our second partial fix involves interpolating inequality indicators for some years. For some countries, a few years had perceptions data available but no comparable household survey over time and no close year available. In this case, and to analyze the same set of countries every year, interpolation was applied to the inequality indicators (see Appendix Table \ref{tab-interp}).

\begin{table}[htbp]{\footnotesize
		\begin{center}
			\caption{Years in which inequality indicators were calculated with a linear interpolation}\label{tab-interp}
			\begin{tabular}{ll}
				\midrule
				Country & Years interpolated \\
				\midrule
				Argentina & 1997, 2001, and 2002 \\
				Bolivia & 2010 \\
				Brazil & 2010 \\
				Chile & 2010 \\
				Colombia & 1997 \\
				Costa Rica & 1997, 2001, 2002, 2007, and 2009 \\
				Ecuador & 1997, 2001 \\
				Guatemala & 1997, 2002, 2009, 2010, and 2013 \\
				Mexico & 2013 \\
				Nicaragua & 2002, 2010, 2011, and 2013 \\
				Panama & 1997, 2001, 2002, and 2007 \\
				Peru & 1997, 2001, 2002 \\
				Venezuela & 2015 \\
				\midrule
			\end{tabular}
		\end{center}
		\begin{singlespace}  \vspace{-.5cm}
			\noindent \justify %\textbf{Notes:}
		\end{singlespace}
	}
\end{table}


Overall, the years in which income inequality was calculated using linear interpolations represent a relatively small share of the total data points (17\% of total). The majority of our inequality data points (69\%) are calculated using a household survey from the same year in which the perceptions polls were conducted, while the remaining 14\% of our inequality indicators are calculated using household surveys from adjacent years. Table \ref{tab-summ-data} summarizes the data sources used in years perceptions data are available.

\begin{table}[htbp]{\footnotesize
		\begin{center}
			\caption{Summary of the data used in every country-year}\label{tab-summ-data}
			\begin{tabular}{r|r|llllllll}
				\multicolumn{1}{r}{} & \multicolumn{1}{r}{1997} & \multicolumn{1}{r}{2001} & \multicolumn{1}{r}{2002} & \multicolumn{1}{r}{2007} & \multicolumn{1}{r}{2009} & \multicolumn{1}{r}{2010} & \multicolumn{1}{r}{2011} & \multicolumn{1}{r}{2013} & \multicolumn{1}{r}{2015} \\
				\cmidrule{2-10}    \multicolumn{1}{l|}{Argentina} & \cellcolor[rgb]{ .608,  .733,  .349}\textcolor[rgb]{ .608,  .733,  .349}{3} & \multicolumn{1}{r|}{\cellcolor[rgb]{ .608,  .733,  .349}\textcolor[rgb]{ .608,  .733,  .349}{3}} & \multicolumn{1}{r|}{\cellcolor[rgb]{ .608,  .733,  .349}\textcolor[rgb]{ .608,  .733,  .349}{3}} & \multicolumn{1}{r|}{\cellcolor[rgb]{ .31,  .506,  .741}\textcolor[rgb]{ .31,  .506,  .741}{1}} & \multicolumn{1}{r|}{\cellcolor[rgb]{ .31,  .506,  .741}\textcolor[rgb]{ .31,  .506,  .741}{1}} & \multicolumn{1}{r|}{\cellcolor[rgb]{ .31,  .506,  .741}\textcolor[rgb]{ .31,  .506,  .741}{1}} & \multicolumn{1}{r|}{\cellcolor[rgb]{ .31,  .506,  .741}\textcolor[rgb]{ .31,  .506,  .741}{1}} & \multicolumn{1}{r|}{\cellcolor[rgb]{ .31,  .506,  .741}\textcolor[rgb]{ .31,  .506,  .741}{1}} & \multicolumn{1}{r|}{\cellcolor[rgb]{ .969,  .588,  .275}\textcolor[rgb]{ .969,  .588,  .275}{2}} \\
				\cmidrule{2-10}    \multicolumn{1}{l|}{Bolivia} & \cellcolor[rgb]{ .31,  .506,  .741}\textcolor[rgb]{ .31,  .506,  .741}{1} & \multicolumn{1}{r|}{\cellcolor[rgb]{ .31,  .506,  .741}\textcolor[rgb]{ .31,  .506,  .741}{1}} & \multicolumn{1}{r|}{\cellcolor[rgb]{ .31,  .506,  .741}\textcolor[rgb]{ .31,  .506,  .741}{1}} & \multicolumn{1}{r|}{\cellcolor[rgb]{ .31,  .506,  .741}\textcolor[rgb]{ .31,  .506,  .741}{1}} & \multicolumn{1}{r|}{\cellcolor[rgb]{ .31,  .506,  .741}\textcolor[rgb]{ .31,  .506,  .741}{1}} & \multicolumn{1}{r|}{\cellcolor[rgb]{ .608,  .733,  .349}\textcolor[rgb]{ .608,  .733,  .349}{3}} & \multicolumn{1}{r|}{\cellcolor[rgb]{ .31,  .506,  .741}\textcolor[rgb]{ .31,  .506,  .741}{1}} & \multicolumn{1}{r|}{\cellcolor[rgb]{ .31,  .506,  .741}\textcolor[rgb]{ .31,  .506,  .741}{1}} & \multicolumn{1}{r|}{\cellcolor[rgb]{ .31,  .506,  .741}\textcolor[rgb]{ .31,  .506,  .741}{1}} \\
				\cmidrule{2-10}    \multicolumn{1}{l|}{Brazil} & \cellcolor[rgb]{ .31,  .506,  .741}\textcolor[rgb]{ .31,  .506,  .741}{1} & \multicolumn{1}{r|}{\cellcolor[rgb]{ .31,  .506,  .741}\textcolor[rgb]{ .31,  .506,  .741}{1}} & \multicolumn{1}{r|}{\cellcolor[rgb]{ .31,  .506,  .741}\textcolor[rgb]{ .31,  .506,  .741}{1}} & \multicolumn{1}{r|}{\cellcolor[rgb]{ .31,  .506,  .741}\textcolor[rgb]{ .31,  .506,  .741}{1}} & \multicolumn{1}{r|}{\cellcolor[rgb]{ .31,  .506,  .741}\textcolor[rgb]{ .31,  .506,  .741}{1}} & \multicolumn{1}{r|}{\cellcolor[rgb]{ .608,  .733,  .349}\textcolor[rgb]{ .608,  .733,  .349}{3}} & \multicolumn{1}{r|}{\cellcolor[rgb]{ .31,  .506,  .741}\textcolor[rgb]{ .31,  .506,  .741}{1}} & \multicolumn{1}{r|}{\cellcolor[rgb]{ .31,  .506,  .741}\textcolor[rgb]{ .31,  .506,  .741}{1}} & \multicolumn{1}{r|}{\cellcolor[rgb]{ .31,  .506,  .741}\textcolor[rgb]{ .31,  .506,  .741}{1}} \\
				\cmidrule{2-10}    \multicolumn{1}{l|}{Chile} & \cellcolor[rgb]{ .969,  .588,  .275}\textcolor[rgb]{ .969,  .588,  .275}{2} & \multicolumn{1}{r|}{\cellcolor[rgb]{ .969,  .588,  .275}\textcolor[rgb]{ .969,  .588,  .275}{2}} & \multicolumn{1}{r|}{\cellcolor[rgb]{ .969,  .588,  .275}\textcolor[rgb]{ .969,  .588,  .275}{2}} & \multicolumn{1}{r|}{\cellcolor[rgb]{ .969,  .588,  .275}\textcolor[rgb]{ .969,  .588,  .275}{2}} & \multicolumn{1}{r|}{\cellcolor[rgb]{ .31,  .506,  .741}\textcolor[rgb]{ .31,  .506,  .741}{1}} & \multicolumn{1}{r|}{\cellcolor[rgb]{ .608,  .733,  .349}\textcolor[rgb]{ .608,  .733,  .349}{3}} & \multicolumn{1}{r|}{\cellcolor[rgb]{ .31,  .506,  .741}\textcolor[rgb]{ .31,  .506,  .741}{1}} & \multicolumn{1}{r|}{\cellcolor[rgb]{ .31,  .506,  .741}\textcolor[rgb]{ .31,  .506,  .741}{1}} & \multicolumn{1}{r|}{\cellcolor[rgb]{ .31,  .506,  .741}\textcolor[rgb]{ .31,  .506,  .741}{1}} \\
				\cmidrule{2-10}    \multicolumn{1}{l|}{Colombia} & \cellcolor[rgb]{ .608,  .733,  .349}\textcolor[rgb]{ .608,  .733,  .349}{3} & \multicolumn{1}{r|}{\cellcolor[rgb]{ .31,  .506,  .741}\textcolor[rgb]{ .31,  .506,  .741}{1}} & \multicolumn{1}{r|}{\cellcolor[rgb]{ .31,  .506,  .741}\textcolor[rgb]{ .31,  .506,  .741}{1}} & \multicolumn{1}{r|}{\cellcolor[rgb]{ .969,  .588,  .275}\textcolor[rgb]{ .969,  .588,  .275}{2}} & \multicolumn{1}{r|}{\cellcolor[rgb]{ .31,  .506,  .741}\textcolor[rgb]{ .31,  .506,  .741}{1}} & \multicolumn{1}{r|}{\cellcolor[rgb]{ .31,  .506,  .741}\textcolor[rgb]{ .31,  .506,  .741}{1}} & \multicolumn{1}{r|}{\cellcolor[rgb]{ .31,  .506,  .741}\textcolor[rgb]{ .31,  .506,  .741}{1}} & \multicolumn{1}{r|}{\cellcolor[rgb]{ .31,  .506,  .741}\textcolor[rgb]{ .31,  .506,  .741}{1}} & \multicolumn{1}{r|}{\cellcolor[rgb]{ .31,  .506,  .741}\textcolor[rgb]{ .31,  .506,  .741}{1}} \\
				\cmidrule{2-10}    \multicolumn{1}{l|}{Costa Rica} & \cellcolor[rgb]{ .608,  .733,  .349}\textcolor[rgb]{ .608,  .733,  .349}{3} & \multicolumn{1}{r|}{\cellcolor[rgb]{ .608,  .733,  .349}\textcolor[rgb]{ .608,  .733,  .349}{3}} & \multicolumn{1}{r|}{\cellcolor[rgb]{ .608,  .733,  .349}\textcolor[rgb]{ .608,  .733,  .349}{3}} & \multicolumn{1}{r|}{\cellcolor[rgb]{ .608,  .733,  .349}\textcolor[rgb]{ .608,  .733,  .349}{3}} & \multicolumn{1}{r|}{\cellcolor[rgb]{ .608,  .733,  .349}\textcolor[rgb]{ .608,  .733,  .349}{3}} & \multicolumn{1}{r|}{\cellcolor[rgb]{ .31,  .506,  .741}\textcolor[rgb]{ .31,  .506,  .741}{1}} & \multicolumn{1}{r|}{\cellcolor[rgb]{ .31,  .506,  .741}\textcolor[rgb]{ .31,  .506,  .741}{1}} & \multicolumn{1}{r|}{\cellcolor[rgb]{ .31,  .506,  .741}\textcolor[rgb]{ .31,  .506,  .741}{1}} & \multicolumn{1}{r|}{\cellcolor[rgb]{ .31,  .506,  .741}\textcolor[rgb]{ .31,  .506,  .741}{1}} \\
				\cmidrule{2-10}    \multicolumn{1}{l|}{Dominican Rep.} & \cellcolor[rgb]{ .749,  .749,  .749}\textcolor[rgb]{ .749,  .749,  .749}{0} & \multicolumn{1}{r|}{\cellcolor[rgb]{ .31,  .506,  .741}\textcolor[rgb]{ .31,  .506,  .741}{1}} & \multicolumn{1}{r|}{\cellcolor[rgb]{ .31,  .506,  .741}\textcolor[rgb]{ .31,  .506,  .741}{1}} & \multicolumn{1}{r|}{\cellcolor[rgb]{ .31,  .506,  .741}\textcolor[rgb]{ .31,  .506,  .741}{1}} & \multicolumn{1}{r|}{\cellcolor[rgb]{ .31,  .506,  .741}\textcolor[rgb]{ .31,  .506,  .741}{1}} & \multicolumn{1}{r|}{\cellcolor[rgb]{ .31,  .506,  .741}\textcolor[rgb]{ .31,  .506,  .741}{1}} & \multicolumn{1}{r|}{\cellcolor[rgb]{ .31,  .506,  .741}\textcolor[rgb]{ .31,  .506,  .741}{1}} & \multicolumn{1}{r|}{\cellcolor[rgb]{ .31,  .506,  .741}\textcolor[rgb]{ .31,  .506,  .741}{1}} & \multicolumn{1}{r|}{\cellcolor[rgb]{ .31,  .506,  .741}\textcolor[rgb]{ .31,  .506,  .741}{1}} \\
				\cmidrule{2-10}    \multicolumn{1}{l|}{Ecuador} & \cellcolor[rgb]{ .608,  .733,  .349}\textcolor[rgb]{ .608,  .733,  .349}{3} & \multicolumn{1}{r|}{\cellcolor[rgb]{ .608,  .733,  .349}\textcolor[rgb]{ .608,  .733,  .349}{3}} & \multicolumn{1}{r|}{\cellcolor[rgb]{ .969,  .588,  .275}\textcolor[rgb]{ .969,  .588,  .275}{2}} & \multicolumn{1}{r|}{\cellcolor[rgb]{ .31,  .506,  .741}\textcolor[rgb]{ .31,  .506,  .741}{1}} & \multicolumn{1}{r|}{\cellcolor[rgb]{ .31,  .506,  .741}\textcolor[rgb]{ .31,  .506,  .741}{1}} & \multicolumn{1}{r|}{\cellcolor[rgb]{ .31,  .506,  .741}\textcolor[rgb]{ .31,  .506,  .741}{1}} & \multicolumn{1}{r|}{\cellcolor[rgb]{ .31,  .506,  .741}\textcolor[rgb]{ .31,  .506,  .741}{1}} & \multicolumn{1}{r|}{\cellcolor[rgb]{ .31,  .506,  .741}\textcolor[rgb]{ .31,  .506,  .741}{1}} & \multicolumn{1}{r|}{\cellcolor[rgb]{ .31,  .506,  .741}\textcolor[rgb]{ .31,  .506,  .741}{1}} \\
				\cmidrule{2-10}    \multicolumn{1}{l|}{El Salvador} & \cellcolor[rgb]{ .31,  .506,  .741}\textcolor[rgb]{ .31,  .506,  .741}{1} & \multicolumn{1}{r|}{\cellcolor[rgb]{ .31,  .506,  .741}\textcolor[rgb]{ .31,  .506,  .741}{1}} & \multicolumn{1}{r|}{\cellcolor[rgb]{ .31,  .506,  .741}\textcolor[rgb]{ .31,  .506,  .741}{1}} & \multicolumn{1}{r|}{\cellcolor[rgb]{ .31,  .506,  .741}\textcolor[rgb]{ .31,  .506,  .741}{1}} & \multicolumn{1}{r|}{\cellcolor[rgb]{ .31,  .506,  .741}\textcolor[rgb]{ .31,  .506,  .741}{1}} & \multicolumn{1}{r|}{\cellcolor[rgb]{ .31,  .506,  .741}\textcolor[rgb]{ .31,  .506,  .741}{1}} & \multicolumn{1}{r|}{\cellcolor[rgb]{ .31,  .506,  .741}\textcolor[rgb]{ .31,  .506,  .741}{1}} & \multicolumn{1}{r|}{\cellcolor[rgb]{ .31,  .506,  .741}\textcolor[rgb]{ .31,  .506,  .741}{1}} & \multicolumn{1}{r|}{\cellcolor[rgb]{ .31,  .506,  .741}\textcolor[rgb]{ .31,  .506,  .741}{1}} \\
				\cmidrule{2-10}    \multicolumn{1}{l|}{Guatemala} & \cellcolor[rgb]{ .969,  .588,  .275}\textcolor[rgb]{ .969,  .588,  .275}{2} & \multicolumn{1}{r|}{\cellcolor[rgb]{ .969,  .588,  .275}\textcolor[rgb]{ .969,  .588,  .275}{2}} & \multicolumn{1}{r|}{\cellcolor[rgb]{ .608,  .733,  .349}\textcolor[rgb]{ .608,  .733,  .349}{3}} & \multicolumn{1}{r|}{\cellcolor[rgb]{ .969,  .588,  .275}\textcolor[rgb]{ .969,  .588,  .275}{2}} & \multicolumn{1}{r|}{\cellcolor[rgb]{ .608,  .733,  .349}\textcolor[rgb]{ .608,  .733,  .349}{3}} & \multicolumn{1}{r|}{\cellcolor[rgb]{ .608,  .733,  .349}\textcolor[rgb]{ .608,  .733,  .349}{3}} & \multicolumn{1}{r|}{\cellcolor[rgb]{ .31,  .506,  .741}\textcolor[rgb]{ .31,  .506,  .741}{1}} & \multicolumn{1}{r|}{\cellcolor[rgb]{ .608,  .733,  .349}\textcolor[rgb]{ .608,  .733,  .349}{3}} & \multicolumn{1}{r|}{\cellcolor[rgb]{ .969,  .588,  .275}\textcolor[rgb]{ .969,  .588,  .275}{2}} \\
				\cmidrule{2-10}    \multicolumn{1}{l|}{Honduras} & \cellcolor[rgb]{ .31,  .506,  .741}\textcolor[rgb]{ .31,  .506,  .741}{1} & \multicolumn{1}{r|}{\cellcolor[rgb]{ .31,  .506,  .741}\textcolor[rgb]{ .31,  .506,  .741}{1}} & \multicolumn{1}{r|}{\cellcolor[rgb]{ .31,  .506,  .741}\textcolor[rgb]{ .31,  .506,  .741}{1}} & \multicolumn{1}{r|}{\cellcolor[rgb]{ .31,  .506,  .741}\textcolor[rgb]{ .31,  .506,  .741}{1}} & \multicolumn{1}{r|}{\cellcolor[rgb]{ .31,  .506,  .741}\textcolor[rgb]{ .31,  .506,  .741}{1}} & \multicolumn{1}{r|}{\cellcolor[rgb]{ .31,  .506,  .741}\textcolor[rgb]{ .31,  .506,  .741}{1}} & \multicolumn{1}{r|}{\cellcolor[rgb]{ .31,  .506,  .741}\textcolor[rgb]{ .31,  .506,  .741}{1}} & \multicolumn{1}{r|}{\cellcolor[rgb]{ .31,  .506,  .741}\textcolor[rgb]{ .31,  .506,  .741}{1}} & \multicolumn{1}{r|}{\cellcolor[rgb]{ .31,  .506,  .741}\textcolor[rgb]{ .31,  .506,  .741}{1}} \\
				\cmidrule{2-10}    \multicolumn{1}{l|}{Mexico} & \cellcolor[rgb]{ .969,  .588,  .275}\textcolor[rgb]{ .969,  .588,  .275}{2} & \multicolumn{1}{r|}{\cellcolor[rgb]{ .969,  .588,  .275}\textcolor[rgb]{ .969,  .588,  .275}{2}} & \multicolumn{1}{r|}{\cellcolor[rgb]{ .31,  .506,  .741}\textcolor[rgb]{ .31,  .506,  .741}{1}} & \multicolumn{1}{r|}{\cellcolor[rgb]{ .969,  .588,  .275}\textcolor[rgb]{ .969,  .588,  .275}{2}} & \multicolumn{1}{r|}{\cellcolor[rgb]{ .969,  .588,  .275}\textcolor[rgb]{ .969,  .588,  .275}{2}} & \multicolumn{1}{r|}{\cellcolor[rgb]{ .31,  .506,  .741}\textcolor[rgb]{ .31,  .506,  .741}{1}} & \multicolumn{1}{r|}{\cellcolor[rgb]{ .969,  .588,  .275}\textcolor[rgb]{ .969,  .588,  .275}{2}} & \multicolumn{1}{r|}{\cellcolor[rgb]{ .608,  .733,  .349}\textcolor[rgb]{ .608,  .733,  .349}{3}} & \multicolumn{1}{r|}{\cellcolor[rgb]{ .969,  .588,  .275}\textcolor[rgb]{ .969,  .588,  .275}{2}} \\
				\cmidrule{2-10}    \multicolumn{1}{l|}{Nicaragua} & \cellcolor[rgb]{ .969,  .588,  .275}\textcolor[rgb]{ .969,  .588,  .275}{2} & \multicolumn{1}{r|}{\cellcolor[rgb]{ .31,  .506,  .741}\textcolor[rgb]{ .31,  .506,  .741}{1}} & \multicolumn{1}{r|}{\cellcolor[rgb]{ .608,  .733,  .349}\textcolor[rgb]{ .608,  .733,  .349}{3}} & \multicolumn{1}{r|}{\cellcolor[rgb]{ .969,  .588,  .275}\textcolor[rgb]{ .969,  .588,  .275}{2}} & \multicolumn{1}{r|}{\cellcolor[rgb]{ .31,  .506,  .741}\textcolor[rgb]{ .31,  .506,  .741}{1}} & \multicolumn{1}{r|}{\cellcolor[rgb]{ .608,  .733,  .349}\textcolor[rgb]{ .608,  .733,  .349}{3}} & \multicolumn{1}{r|}{\cellcolor[rgb]{ .608,  .733,  .349}\textcolor[rgb]{ .608,  .733,  .349}{3}} & \multicolumn{1}{r|}{\cellcolor[rgb]{ .608,  .733,  .349}\textcolor[rgb]{ .608,  .733,  .349}{3}} & \multicolumn{1}{r|}{\cellcolor[rgb]{ .969,  .588,  .275}\textcolor[rgb]{ .969,  .588,  .275}{2}} \\
				\cmidrule{2-10}    \multicolumn{1}{l|}{Panama} & \cellcolor[rgb]{ .608,  .733,  .349}\textcolor[rgb]{ .608,  .733,  .349}{3} & \multicolumn{1}{r|}{\cellcolor[rgb]{ .608,  .733,  .349}\textcolor[rgb]{ .608,  .733,  .349}{3}} & \multicolumn{1}{r|}{\cellcolor[rgb]{ .608,  .733,  .349}\textcolor[rgb]{ .608,  .733,  .349}{3}} & \multicolumn{1}{r|}{\cellcolor[rgb]{ .608,  .733,  .349}\textcolor[rgb]{ .608,  .733,  .349}{3}} & \multicolumn{1}{r|}{\cellcolor[rgb]{ .31,  .506,  .741}\textcolor[rgb]{ .31,  .506,  .741}{1}} & \multicolumn{1}{r|}{\cellcolor[rgb]{ .31,  .506,  .741}\textcolor[rgb]{ .31,  .506,  .741}{1}} & \multicolumn{1}{r|}{\cellcolor[rgb]{ .31,  .506,  .741}\textcolor[rgb]{ .31,  .506,  .741}{1}} & \multicolumn{1}{r|}{\cellcolor[rgb]{ .31,  .506,  .741}\textcolor[rgb]{ .31,  .506,  .741}{1}} & \multicolumn{1}{r|}{\cellcolor[rgb]{ .31,  .506,  .741}\textcolor[rgb]{ .31,  .506,  .741}{1}} \\
				\cmidrule{2-10}    \multicolumn{1}{l|}{Paraguay} & \cellcolor[rgb]{ .31,  .506,  .741}\textcolor[rgb]{ .31,  .506,  .741}{1} & \multicolumn{1}{r|}{\cellcolor[rgb]{ .31,  .506,  .741}\textcolor[rgb]{ .31,  .506,  .741}{1}} & \multicolumn{1}{r|}{\cellcolor[rgb]{ .31,  .506,  .741}\textcolor[rgb]{ .31,  .506,  .741}{1}} & \multicolumn{1}{r|}{\cellcolor[rgb]{ .31,  .506,  .741}\textcolor[rgb]{ .31,  .506,  .741}{1}} & \multicolumn{1}{r|}{\cellcolor[rgb]{ .31,  .506,  .741}\textcolor[rgb]{ .31,  .506,  .741}{1}} & \multicolumn{1}{r|}{\cellcolor[rgb]{ .31,  .506,  .741}\textcolor[rgb]{ .31,  .506,  .741}{1}} & \multicolumn{1}{r|}{\cellcolor[rgb]{ .31,  .506,  .741}\textcolor[rgb]{ .31,  .506,  .741}{1}} & \multicolumn{1}{r|}{\cellcolor[rgb]{ .31,  .506,  .741}\textcolor[rgb]{ .31,  .506,  .741}{1}} & \multicolumn{1}{r|}{\cellcolor[rgb]{ .31,  .506,  .741}\textcolor[rgb]{ .31,  .506,  .741}{1}} \\
				\cmidrule{2-10}    \multicolumn{1}{l|}{Peru} & \cellcolor[rgb]{ .969,  .588,  .275}\textcolor[rgb]{ .969,  .588,  .275}{2} & \multicolumn{1}{r|}{\cellcolor[rgb]{ .31,  .506,  .741}\textcolor[rgb]{ .31,  .506,  .741}{1}} & \multicolumn{1}{r|}{\cellcolor[rgb]{ .31,  .506,  .741}\textcolor[rgb]{ .31,  .506,  .741}{1}} & \multicolumn{1}{r|}{\cellcolor[rgb]{ .31,  .506,  .741}\textcolor[rgb]{ .31,  .506,  .741}{1}} & \multicolumn{1}{r|}{\cellcolor[rgb]{ .31,  .506,  .741}\textcolor[rgb]{ .31,  .506,  .741}{1}} & \multicolumn{1}{r|}{\cellcolor[rgb]{ .31,  .506,  .741}\textcolor[rgb]{ .31,  .506,  .741}{1}} & \multicolumn{1}{r|}{\cellcolor[rgb]{ .31,  .506,  .741}\textcolor[rgb]{ .31,  .506,  .741}{1}} & \multicolumn{1}{r|}{\cellcolor[rgb]{ .31,  .506,  .741}\textcolor[rgb]{ .31,  .506,  .741}{1}} & \multicolumn{1}{r|}{\cellcolor[rgb]{ .31,  .506,  .741}\textcolor[rgb]{ .31,  .506,  .741}{1}} \\
				\cmidrule{2-10}    \multicolumn{1}{l|}{Uruguay} & \cellcolor[rgb]{ .31,  .506,  .741}\textcolor[rgb]{ .31,  .506,  .741}{1} & \multicolumn{1}{r|}{\cellcolor[rgb]{ .31,  .506,  .741}\textcolor[rgb]{ .31,  .506,  .741}{1}} & \multicolumn{1}{r|}{\cellcolor[rgb]{ .31,  .506,  .741}\textcolor[rgb]{ .31,  .506,  .741}{1}} & \multicolumn{1}{r|}{\cellcolor[rgb]{ .31,  .506,  .741}\textcolor[rgb]{ .31,  .506,  .741}{1}} & \multicolumn{1}{r|}{\cellcolor[rgb]{ .31,  .506,  .741}\textcolor[rgb]{ .31,  .506,  .741}{1}} & \multicolumn{1}{r|}{\cellcolor[rgb]{ .31,  .506,  .741}\textcolor[rgb]{ .31,  .506,  .741}{1}} & \multicolumn{1}{r|}{\cellcolor[rgb]{ .31,  .506,  .741}\textcolor[rgb]{ .31,  .506,  .741}{1}} & \multicolumn{1}{r|}{\cellcolor[rgb]{ .31,  .506,  .741}\textcolor[rgb]{ .31,  .506,  .741}{1}} & \multicolumn{1}{r|}{\cellcolor[rgb]{ .31,  .506,  .741}\textcolor[rgb]{ .31,  .506,  .741}{1}} \\
				\cmidrule{2-10}    \multicolumn{1}{l|}{Venezuela} & \cellcolor[rgb]{ .31,  .506,  .741}\textcolor[rgb]{ .31,  .506,  .741}{1} & \multicolumn{1}{r|}{\cellcolor[rgb]{ .31,  .506,  .741}\textcolor[rgb]{ .31,  .506,  .741}{1}} & \multicolumn{1}{r|}{\cellcolor[rgb]{ .31,  .506,  .741}\textcolor[rgb]{ .31,  .506,  .741}{1}} & \multicolumn{1}{r|}{\cellcolor[rgb]{ .31,  .506,  .741}\textcolor[rgb]{ .31,  .506,  .741}{1}} & \multicolumn{1}{r|}{\cellcolor[rgb]{ .31,  .506,  .741}\textcolor[rgb]{ .31,  .506,  .741}{1}} & \multicolumn{1}{r|}{\cellcolor[rgb]{ .31,  .506,  .741}\textcolor[rgb]{ .31,  .506,  .741}{1}} & \multicolumn{1}{r|}{\cellcolor[rgb]{ .31,  .506,  .741}\textcolor[rgb]{ .31,  .506,  .741}{1}} & \multicolumn{1}{r|}{\cellcolor[rgb]{ .969,  .588,  .275}\textcolor[rgb]{ .969,  .588,  .275}{2}} & \multicolumn{1}{r|}{\cellcolor[rgb]{ .608,  .733,  .349}\textcolor[rgb]{ .608,  .733,  .349}{3}} \\
				\cmidrule{2-10}    \multicolumn{1}{r}{} & \multicolumn{1}{r}{} &   &   &   &   &   &   &   &  \\
				\cmidrule{2-2}      & \cellcolor[rgb]{ .31,  .506,  .741} & \multicolumn{8}{l}{Both perceptions and inequality data available} \\
				\cmidrule{2-2}      & \cellcolor[rgb]{ .969,  .588,  .275} & \multicolumn{8}{l}{Inequality was calculated with a close survey} \\
				\cmidrule{2-2}      & \cellcolor[rgb]{ .608,  .733,  .349} & \multicolumn{8}{l}{Inequality was calculated with a linear interpolation} \\
				\cmidrule{2-2}      & \cellcolor[rgb]{ .749,  .749,  .749} & \multicolumn{8}{l}{Latinobar\'ometro did not conduct survey in this year} \\
				\cmidrule{2-2}    \end{tabular}%
		\end{center}
		\begin{singlespace}  \vspace{-.5cm}
			\noindent \justify %\textbf{Notes:}
		\end{singlespace}
	}
\end{table}

\subsection{Imputation of Missing Values for the Regression Analysis}

Two of our individual-level variables (political ideology and religion) have many missing values in some country-years. To deal with this in our regressions, we imputed the average value of each variable to individuals with a missing value. In those cases, we included in the regression a dummy that takes the value one if the value of the variable was imputed and zero otherwise. The results are similar if we do not impute the values, but the sample size of the regressions is smaller.

\subsection{Comparison between Latinobar\'ometro's and SEDLAC's samples} \label{app_lb_sedlac}

To assess whether there are systematic differences between Latinobar\'ometro's sample and the household surveys' sample, in Appendix Table \ref{tab-lat-sedlac} we compare a set of variables available in both datasets during 2013. To ensure comparability across databases, we restrict the calculations to individuals over age 18 and countries with data available in both harmonization projects.

In general, the samples are similar in observable characteristics. For instance, the average age in Latinobar\'ometro's 2013 sample is 40.6 years, while in SEDLAC it is 42.7 years. Similarly, the percentage of males is 48.9\% in Latinobar\'ometro and 47.6\% in SEDLAC. The main difference arises from educational attainment. On average, the SEDLAC sample is more educated: 46.1\% of the population has secondary education or more, while this figure is 38.8\% in Latinobar\'ometro.


% Comparison of descriptive statistics in Latinobar\'ometro and SEDLAC, 2013
\begin{table}[H]{\scriptsize
		\begin{center}
			\caption{Descriptive statistics in Latinobar\'ometro and SEDLAC, 2013 (selected countries)} \label{tab-lat-sedlac}
			\begin{tabular}{lcccccccc}
				\midrule
				& \multicolumn{2}{c}{Mean} &   & \multicolumn{2}{c}{Standard Dev.} &   & \multicolumn{2}{c}{Observations} \\
				\cmidrule{2-3}\cmidrule{5-6}\cmidrule{8-9}      &  Latinob.  &  SEDLAC  &   &  Latinob.  &  SEDLAC  &   &  Latinob.  &  SEDLAC  \\
				& (1) & (2) &   & (3) & (4) &   & (5) & (6) \\
				\midrule
				\textbf{\hspace{-1em} Panel A. Sociodemographic} &   &   &   &   &   &   &   &  \\
				Age & 40.59 & 42.68 &   & 16.43 & 17.25 &   &      14,855  &   1,004,894  \\
				Male (\%) & 48.97 & 47.63 &   & 0.50 & 0.50 &   &      14,855  &   1,004,894  \\
				Married or civil union (\%) & 56.77 & 36.41 &   & 0.50 & 0.48 &   &      14,804  &      915,117  \\
				\multicolumn{3}{l}{\textbf{\hspace{-1em} Panel B. Education and Labor market}} &   &   &   &   &   &  \\
				Literate (\%) & 91.18 & 92.17 &   & 0.28 & 0.27 &   &      14,855  &   1,004,744  \\
				Secondary education or more (\%) & 38.83 & 46.11 &   & 0.49 & 0.50 &   &      14,855  &   1,001,672  \\
				Economically active (\%) & 65.14 & 68.66 &   & 0.48 & 0.46 &   &      14,855  &   1,004,718  \\
				Unemployed (\%) & 5.78 & 4.08 &   & 0.23 & 0.20 &   &      14,855  &   1,004,718  \\
				\textbf{\hspace{-1em} Panel C. Assets and Services} &   &   &   &   &   &   &   &  \\
				Access to a sewerage (\%) & 68.76 & 63.41 &   & 0.46 & 0.48 &   &      13,799  &      975,726  \\
				Car (\%) & 26.37 & 21.09 &   & 0.44 & 0.41 &   &      11,612  &      643,350  \\
				Computer (\%) & 46.55 & 47.82 &   & 0.50 & 0.50 &   &      12,747  &      894,003  \\
				Fridge (\%) & 82.76 & 88.89 &   & 0.38 & 0.31 &   &      12,763  &      894,003  \\
				Homeowner (\%) & 74.09 & 69.64 &   & 0.44 & 0.46 &   &      14,761  &   1,003,306  \\
				Mobile (\%) & 86.91 & 91.78 &   & 0.34 & 0.27 &   &      12,754  &      896,079  \\
				Washing machine (\%) & 60.49 & 56.88 &   & 0.49 & 0.50 &   &      11,816  &      848,350  \\
				Landline (\%) & 40.22 & 39.47 &   & 0.49 & 0.49 &   &      12,736  &      896,425  \\
				\midrule
			\end{tabular}
		\end{center}
		\begin{singlespace}
			\noindent \justify \textbf{Note:} This table compares the observable characteristics of individuals in Latinobar\'ometro and SEDLAC. Summary statistics were calculated on a restricted sample (individuals aged over 18) to ensure comparability between both datasets, pooling data from 14 countries in 2013: Argentina, Bolivia, Brazil, Chile, Colombia, Costa Rica, Dominican Republic, Ecuador, El Salvador, Honduras, Panama, Peru, Paraguay, and Uruguay.
		\end{singlespace}

	}
\end{table}


%%%%%%%%%%%%%%%%%%%%%%%%%%%%%%%%%%%%%%%%%%%%%%%%%%%%%%%%%%%%%%%%%%%%%%%%%%%%%%%
% APPENDIX C: OAXACA-BLINDER DECOMPOSITION
%%%%%%%%%%%%%%%%%%%%%%%%%%%%%%%%%%%%%%%%%%%%%%%%%%%%%%%%%%%%%%%%%%%%%%%%%%%%%%%

\clearpage
\section{The Oaxaca-Blinder Decomposition} \label{sec_oaxaca}

\setcounter{table}{0}
\setcounter{figure}{0}
\setcounter{equation}{0}
\renewcommand{\thetable}{C\arabic{table}}
\renewcommand{\thefigure}{C\arabic{figure}}
\renewcommand{\theequation}{C\arabic{equation}}

The starting point to decompose changes in unfairness perceptions between 2002 and 2013 is the following equation:
%
\begin{align}
	\text{Unfair}_{ict} = \beta_t X_{ict} + \gamma_t \text{Gini}_{ct} + \varepsilon_{ict} \quad \text{for} \quad t \in \{2002, 2013\},
\end{align}
%
where $t$ indicates the year in which perceptions are elicited and $X_{ict}$ is a vector that contains individual-level controls. The fraction of individuals who perceive the income distribution as unfair in year $t$ can be calculated as
%
\begin{align}
	\overline{\text{Unfair}}_{t} = \hat{\beta}_{t} \bar{X}_{t} + \hat{\gamma}_{t} \overline{\text{Gini}}_{t}  \quad \text{for} \quad t \in \{2002, 2013\},
\end{align}
%
where $\bar{X}_t$ is a vector of the average values of the explanatory variables in year $t$, and $\hat{\beta}$ the vector of OLS-estimated coefficients. The change in unfairness beliefs between 2013 and 2002 is given by
%
\begin{align} \label{eq-diff-fair}
	\underbrace{\overline{\text{Unfair}}_{2013} - \overline{\text{Unfair}}_{2013}}_{\equiv \Delta \text{Unfair}} = (\hat{\beta}_{2013} \bar{X}_{2013} + \hat{\gamma}_{2013} \overline{\text{Gini}}_{2013}) - (\hat{\beta}_{2002} \bar{X}_{2002} + \hat{\gamma}_{2002} \overline{\text{Gini}}_{2002})
\end{align}

Adding and subtracting $\hat{\beta}_{2002} \bar{X}_{2013} + \hat{\gamma}_{2002} \overline{\text{Gini}}_{2013}$ to equation \eqref{eq-diff-fair} yields
%
\begin{align} \label{eq_oaxaca}
	\Delta \text{Unfair} &=
	\underbrace{\hat{\beta}_{2002} (\bar{X}_{2013} - \bar{X}_{2002})}_{\equiv \Delta \text{Demog.}} +
	\underbrace{\hat{\gamma}_{2002} (\overline{\text{Gini}}_{2013} - \overline{\text{Gini}}_{2002})}_{\equiv \Delta \text{Gini}} \notag \\ &+ \underbrace{\bar{X}_{2013} (\hat{\beta}_{2013} - \hat{\beta}_{2002})+ \overline{\text{Gini}}_{2013} (\hat{\gamma}_{2013} - \hat{\gamma}_{2002})}_{\text{Residual}}
\end{align}
%

The first two terms of equation \eqref{eq_oaxaca} are usually known as the ``composition effect.'' These effects capture the difference between the average perceptions in 2002 and the counterfactual perceptions 2013 had the $\hat{\beta}$'s and $\hat{\gamma}$---i.e., the elasticity of perceptions to the different covariables---remained constant during the 2002--13 period. The first term captures differences in individual-level demographic variables that determine unfairness perceptions in the model (such as educational attainment, age, and employment status). The second term captures changes in aggregate trends in income inequality.

The third term of \eqref{eq_oaxaca} reflects the difference between the average fairness views in 2013 and the counterfactual fairness views in 2002 with the observable attributes of 2013. Thus, this component reflects changes in fairness views due to changes in the elasticity of the different covariables between both years. Since we cannot explain why the coefficients attached to each variable changed, this term is usually viewed as the ``unexplained'' part of the decomposition and treated as the residual of the decomposition.


% End standalone document
\ifdefined\standalonetrue
  \end{document}
\fi
 after \appendix in main document

\ifdefined\appendixstandalone
\else
  % Standalone preamble - only used when compiling this file directly
  \newif\ifstandalone
  \standalonetrue

  \documentclass[11pt]{article}
  \usepackage[a4paper,pdftex]{geometry}
  \setlength{\oddsidemargin}{3mm}
  \setlength{\evensidemargin}{3mm}
  \usepackage[T1]{fontenc}
  \usepackage[utf8x]{inputenc}
  \usepackage{amsmath,amsthm,amssymb,graphicx,enumerate,booktabs,bigstrut,rotating,multirow,float,caption,etoolbox,hyperref,lmodern,comment,listings,qtree,a4wide,titlesec,moresize,bbm,subcaption,tikz,pdfescape,mathtools,flafter,enumitem,setspace,ragged2e,colortbl,lineno,lscape}
  \hypersetup{colorlinks,linkcolor={red},citecolor={blue},urlcolor={blue}}

  \usepackage[justification=centering,font=small]{caption}

  \usepackage[authoryear]{natbib}
  \usepackage[english]{babel}
  \renewcommand{\baselinestretch}{1.25}
  \DeclareMathOperator{\E}{\mathbb{E}}
  \DeclareMathOperator*{\argmax}{argmax}
  \makeatletter
  \renewcommand\subsubsection{\@startsection{subsubsection}{3}{\z@}%
  	{-3.25ex\@plus -1ex \@minus -.2ex}%
  	{-1.5ex \@plus -.2ex}%
  	{\normalfont\normalsize\bfseries}}
  \def\@biblabel#1{\hspace*{-\labelsep}}

  \newcommand*\circled[1]{\tikz[baseline=(char.base)]{
  		\node[shape=circle,draw,inner sep=2pt] (char) {#1};}}
  \newcommand*\ExpandableInput[1]{\@@input#1 }
  \makeatother

  \def\sym#1{\ifmmode^{#1}\else\(^{#1}\)\fi}
  \onehalfspacing
  \pdfpageheight\paperheight
  \pdfpagewidth\paperwidth

  \newcolumntype{L}[1]{>{\raggedright\let\newline\\\arraybackslash\hspace{0pt}}m{#1}}
  \newcolumntype{C}[1]{>{\centering\let\newline\\\arraybackslash\hspace{0pt}}m{#1}}
  \newcolumntype{R}[1]{>{\raggedleft\let\newline\\\arraybackslash\hspace{0pt}}m{#1}}

  \begin{document}

  \title{Are Fairness Perceptions Shaped by Income Inequality? \\ Evidence from Latin America \\ \vspace{10pt} \Large Online Appendix}
  \author{Leonardo Gasparini \and Germ\'an Reyes}
  \date{\vspace{15pt} February 2022}
  \maketitle

  \appendix
\fi

%%%%%%%%%%%%%%%%%%%%%%%%%%%%%%%%%%%%%%%%%%%%%%%%%%%%%%%%%%%%%%%%%%%%%%%%%%%%%%%
% APPENDIX A: ADDITIONAL FIGURES AND TABLES
%%%%%%%%%%%%%%%%%%%%%%%%%%%%%%%%%%%%%%%%%%%%%%%%%%%%%%%%%%%%%%%%%%%%%%%%%%%%%%%

\begin{center}\noindent{\LARGE \textbf{Appendix}}\end{center}

\section{Additional Figures and Tables} \label{app_add_figs}

\setcounter{table}{0}
\setcounter{figure}{0}
\setcounter{equation}{0}
\renewcommand{\thetable}{A\arabic{table}}
\renewcommand{\thefigure}{A\arabic{figure}}
\renewcommand{\theequation}{A\arabic{equation}}


\begin{figure}[htpb]
	\caption{Perceptions of unfairness and individual characteristics, 1997--2015} \label{fig-fairness-grps}
	\centering
	\begin{subfigure}[t]{.48\textwidth}
		\caption*{Panel A. By age}\label{fig-fairness-age}
		\centering
		\includegraphics[width=\linewidth]{../results/fig-fairness-age}
	\end{subfigure}
	\hfill
	\begin{subfigure}[t]{0.48\textwidth}
		\caption*{Panel B. By sex}\label{fig-fairness-sex}
		\centering
		\includegraphics[width=\linewidth]{../results/fig-fairness-sex}
	\end{subfigure}
	\hfill
	\begin{subfigure}[t]{.48\textwidth}
		\caption*{Panel C. By educational attainment}\label{fig-fairness-educ}
		\centering
		\includegraphics[width=\linewidth]{../results/fig-fairness-educ}
	\end{subfigure}
	\hfill
	\begin{subfigure}[t]{0.48\textwidth}
		\caption*{Panel D. By employment status}\label{fig-fairness-employ}
		\centering
		\includegraphics[width=\linewidth]{../results/fig-fairness-employ}
	\end{subfigure}
	\hfill
	{\footnotesize
		\singlespacing \justify

		\textbf{Notes:} This figure shows the share of individuals who perceive the income distribution as unfair or very unfair according to their age, gender, maximum educational attainment, and employment status.


	}
\end{figure}

\clearpage
\begin{figure}[htp]
	\caption{The evolution of fairness views and income inequality in Latin America}\label{fig-timeseries-gini-unfair}  \centering
	\centering
	\includegraphics[width=.75\linewidth]{../results/fig-timeseries-gini-unfair}
	\hfill
	{\footnotesize
		\singlespacing \justify

		\textbf{Notes:} This figure shows the evolution of the average Gini coefficient across countries in our sample (right-hand-side variable) and the fraction of the population who perceive the income distribution as unfair or very unfair (left-hand-side variable) over 1997--2015. To have a balanced panel of countries over time, we linearly extrapolated the Gini coefficient in years in which income microdata is not available (see Appendix \ref{sec_data}).

	}
\end{figure}



\clearpage
\begin{table}[H]{\footnotesize
		\begin{center}
			\caption{Descriptive statistics of our sample} \label{tab-sum-stats}
			\begin{tabular}{lccc}
				\midrule
				&  Mean  &  Standard Dev.  & Observations \\
				& (1) & (2) & (3) \\
				\midrule
				\textbf{\hspace{-1em} Panel A. Sociodemographic} &   &   &  \\
				Age & 39.75 & 16.23 &         225,551  \\
				Male (\%) & 48.97 & 0.50 &         225,567  \\
				Married or civil union (\%) & 56.27 & 0.50 &         224,081  \\
				Catholic religion (\%) & 68.01 & 0.47 &         222,790  \\
				Ideology (10 = right-wing) & 5.48 & 2.64 &         131,980  \\
				\textbf{\hspace{-1em} Panel B. Education and Labor market} &   &   &  \\
				Literate (\%) & 90.31 & 0.30 &         224,056  \\
				Secondary education or more (\%) & 33.65 & 0.47 &         224,056  \\
				Parents with secondary education (\%) & 17.43 & 0.38 &         184,884  \\
				Economically active (\%) & 64.14 & 0.48 &         225,222  \\
				Unemployed (\% Labor Force) & 9.89 & 0.30 &         225,222  \\
				\textbf{\hspace{-1em} Panel C. Access to services} &   &   &  \\
				Access to a sewerage (\%) & 69.59 & 0.46 &         222,530  \\
				Access to running water (\%) & 88.83 & 0.31 &         204,340  \\
				\textbf{\hspace{-1em} Panel D. Asset ownership} &   &   &  \\
				Car (\%) & 28.21 & 0.45 &         222,338  \\
				Computer (\%) & 33.79 & 0.47 &         222,645  \\
				Fridge (\%) & 79.22 & 0.41 &         146,686  \\
				Homeowner (\%) & 73.92 & 0.44 &         223,603  \\
				Mobile (\%) & 80.61 & 0.40 &         172,253  \\
				Washing machine (\%) & 54.71 & 0.50 &         223,122  \\
				Landline (\%) & 42.28 & 0.49 &         222,968  \\
				\midrule
			\end{tabular}
		\end{center}
		\begin{singlespace} \vspace{-.5cm}
			\noindent \justify \textbf{Note:} This table shows summary statistics on our sample pooling data from all countries in our sample over 1997--2015.
		\end{singlespace}
	}
\end{table}


% Fairness perceptions by population group pooling data 1997-2015, in %
\clearpage
\begin{table}[H]{\footnotesize
		\begin{center}
			\caption{Fairness views by population group} \label{tab-fairness-grp}
			\newcommand\w{1.70}
			\begin{tabular}{l@{}R{\w cm}R{\w cm}R{\w cm}R{\w cm}}
				\midrule
				& \multicolumn{4}{c}{\% of individuals who believe income distribution is:} \\
				\cmidrule{2-5}  & Very unfair & Unfair & Fair & Very fair \\
				& (1) & (2) & (3) & (4) \\
				\midrule
				All & 28.2 & 51.6 & 17.3 & 2.9 \\
				\textbf{\hspace{-1em} Panel A. Gender} &   &   &   &  \\
				Female & 28.3 & 52.2 & 16.7 & 2.8 \\
				Male & 28.0 & 51.1 & 17.9 & 3.0 \\
				\textbf{\hspace{-1em} Panel B. Age group} &   &   &   &  \\
				15-24 & 25.2 & 52.0 & 19.7 & 3.1 \\
				25-40 & 28.5 & 51.3 & 17.2 & 3.0 \\
				41-64 & 29.5 & 51.7 & 16.0 & 2.8 \\
				65+ & 29.2 & 51.6 & 16.6 & 2.5 \\
				\textbf{\hspace{-1em} Panel C. Civil status} &   &   &   &  \\
				Married & 28.3 & 51.9 & 17.0 & 2.8 \\
				Not married & 27.9 & 51.4 & 17.7 & 3.1 \\
				\textbf{\hspace{-1em} Panel D. Religion} &   &   &   &  \\
				Catholic & 28.2 & 51.7 & 17.2 & 2.9 \\
				Not catholic & 28.0 & 51.5 & 17.5 & 3.0 \\
				\textbf{\hspace{-1em} Panel E. Education level} &   &   &   &  \\
				Less than Primary & 27.7 & 51.6 & 17.7 & 3.0 \\
				Complete Primary & 27.9 & 52.2 & 17.4 & 2.6 \\
				Complete Secondary & 29.1 & 53.2 & 15.0 & 2.7 \\
				Complete Tertiary & 29.0 & 50.8 & 17.1 & 3.1 \\
				\textbf{\hspace{-1em} Panel F. Type of employment} &   &   &   &  \\
				Employee & 28.3 & 51.5 & 17.2 & 2.9 \\
				Employer & 24.3 & 53.9 & 19.0 & 2.8 \\
				Self-employed & 28.0 & 51.4 & 17.5 & 3.1 \\
				Unemployed & 30.3 & 51.6 & 15.1 & 3.0 \\
				\textbf{\hspace{-1em} Panel E. Country} &   &   &   &  \\
				Argentina & 38.17 & 50.74 & 10.26 & 0.83 \\
				Bolivia & 18.01 & 56.13 & 23.39 & 2.48 \\
				Brazil & 31.95 & 53.71 & 12.85 & 1.49 \\
				Chile & 40.20 & 49.93 & 8.42 & 1.45 \\
				Colombia & 35.15 & 51.20 & 11.40 & 2.26 \\
				Costa Rica & 23.20 & 53.55 & 20.13 & 3.12 \\
				Dominican Rep. & 32.31 & 46.52 & 17.61 & 3.56 \\
				Ecuador & 21.45 & 47.46 & 27.58 & 3.51 \\
				El Salvador & 22.73 & 53.16 & 20.45 & 3.65 \\
				Guatemala & 28.29 & 51.34 & 16.70 & 3.66 \\
				Honduras & 28.87 & 53.42 & 14.33 & 3.38 \\
				Mexico & 32.15 & 49.75 & 15.32 & 2.78 \\
				Nicaragua & 18.69 & 51.88 & 24.33 & 5.11 \\
				Panama & 27.39 & 48.01 & 20.25 & 4.34 \\
				Paraguay & 38.31 & 48.80 & 10.95 & 1.93 \\
				Peru & 25.03 & 61.89 & 11.70 & 1.38 \\
				Uruguay & 18.22 & 57.51 & 22.64 & 1.64 \\
				Venezuela & 23.51 & 42.96 & 26.62 & 6.92 \\
				\midrule
			\end{tabular}
		\end{center}
		\begin{singlespace} \vspace{-.5cm}
			\noindent \justify \textbf{Note:} This table shows the fraction of individuals in our sample who perceive the income distribution as very unfair, unfair, fair, or very fair.
		\end{singlespace}
	}
\end{table}



\clearpage
\begin{table}[htpb!]{\footnotesize
		\begin{center}
			\caption{Logit regressions of unfairness perceptions (unfair) and individual characteristics} \label{unfair_logit}
			\newcommand\w{1.30}
			\begin{tabular}{l@{}lR{\w cm}@{}L{0.43cm}R{\w cm}@{}L{0.43cm}R{\w cm}@{}L{0.43cm}R{\w cm}@{}L{0.43cm}R{\w cm}@{}L{0.43cm}R{\w cm}@{}L{0.43cm}}
				\midrule
				&& 	\multicolumn{12}{c}{Dependent Variable: Believes income distribution is unfair or very unfair}  \\\cmidrule{3-14}
				%					&&          &&       &&          && Labor     &&  && Ideology, \\
				%					&& Baseline && Demog.   && Education && status  && Assets && Religion \\
				&& (1) && (2) && (3) && (4) && (5) && (6) \\
				\midrule
				\ExpandableInput{../results/unfair_logit}
				\midrule
			\end{tabular}
		\end{center}
		\begin{singlespace}  \vspace{-.5cm}
			\noindent \justify \textbf{Notes:} This table shows estimates of the relationship between an indicator that equals one for individuals who believe that the income distribution is unfair or very unfair and the Gini coefficient controlling for individuals' characteristics. Coefficients are estimated through Logit regressions and represent the marginal effects evaluated at the mean values of the rest of the variables. Observations are weighted by the individual's probability of being interviewed. All specifications include country and year fixed effects. $^{***}$, $^{**}$ and $^*$ denote significance at 10\%, 5\% and 1\% levels, respectively. Heteroskedasticity-robust standard errors clustered at the country-by-year level in parentheses.
		\end{singlespace}
	}
\end{table}


\clearpage
\begin{table}[htpb!]{\footnotesize
		\begin{center}
			\caption{Logit regressions of unfairness perceptions (very unfair) and different inequality indicators} \label{unfair_ineq_logit}
			\newcommand\w{1.50}
			\begin{tabular}{l@{}lR{\w cm}@{}L{0.43cm}R{\w cm}@{}L{0.43cm}R{\w cm}@{}L{0.43cm}R{\w cm}@{}L{0.43cm}R{\w cm}@{}L{0.43cm}}
				\midrule
				&& 	\multicolumn{10}{c}{Dependent Variable: Believes income distribution is very unfair}  \\\cmidrule{3-12}
				&& (1) && (2) && (3) && (4) && (5)  \\
				\midrule
				\ExpandableInput{../results/very_unfair_ineq_logit}
				\midrule
			\end{tabular}
		\end{center}
		\begin{singlespace}  \vspace{-.5cm}
			\noindent \justify \textbf{Notes:} This table shows estimates of the relationship between an indicator that equals one for individuals who believe income distribution is unfair or very unfair and several inequality indicators controlling for individuals' characteristics. Coefficients are estimated through Logit regressions and represent the marginal effects evaluated at the mean values of the rest of the variables. Observations are weighted by the individual's probability of being interviewed. All specifications include country and year fixed effects. $^{***}$, $^{**}$ and $^*$ denote significance at 10\%, 5\% and 1\% levels, respectively. Heteroskedasticity-robust standard errors clustered at the country-by-year level in parentheses.

		\end{singlespace}
	}
\end{table}




\begin{table}[htpb!]{\footnotesize
		\begin{center}
			\caption{OLS regressions of unfairness perceptions (very unfair) and individual characteristics} \label{very_unfair_lpm}
			\newcommand\w{1.30}
			\begin{tabular}{l@{}lR{\w cm}@{}L{0.43cm}R{\w cm}@{}L{0.43cm}R{\w cm}@{}L{0.43cm}R{\w cm}@{}L{0.43cm}R{\w cm}@{}L{0.43cm}R{\w cm}@{}L{0.43cm}}
				\midrule
				&& 	\multicolumn{12}{c}{Dependent Variable: Believes income distribution is very unfair}  \\\cmidrule{3-14}
				%					&&          &&       &&          && Labor     &&  && Ideology, \\
				%					&& Baseline && Demog.   && Education && status  && Assets && Religion \\
				&& (1) && (2) && (3) && (4) && (5) && (6) \\
				\midrule
				\ExpandableInput{../results/very_unfair_lpm}
				\midrule
			\end{tabular}
		\end{center}
		\begin{singlespace}  \vspace{-.5cm}
			\noindent \justify \textbf{Notes:} This table presents estimates of the correlation between a dummy variable that indicates if the individual believes income distribution is unfair or very unfair and the Gini coefficient controlling for individuals' characteristics. Coefficients are estimated through a linear probability model. Observations are weighted by the individual's probability of being interviewed. All specifications include country and year fixed effects. $^{***}$, $^{**}$ and $^*$ denote significance at 10\%, 5\% and 1\% levels, respectively. Heteroskedasticity-robust standard errors clustered at the country-by-year level in parentheses.
		\end{singlespace}
	}
\end{table}


%%%%%%%%%%%%%%%%%%%%%%%%%%%%%%%%%%%%%%%%%%%%%%%%%%%%%%%%%%%%%%%%%%%%%%%%%%%%%%%
% APPENDIX B: DATA APPENDIX
%%%%%%%%%%%%%%%%%%%%%%%%%%%%%%%%%%%%%%%%%%%%%%%%%%%%%%%%%%%%%%%%%%%%%%%%%%%%%%%

\clearpage
\section{Data Appendix} \label{sec_data}

\setcounter{table}{0}
\setcounter{figure}{0}
\renewcommand{\thetable}{B\arabic{table}}
\renewcommand{\thefigure}{B\arabic{figure}}

The figures presented in this paper are based on two harmonization projects, known as Latinobar\'ometro and SEDLAC (Socio-Economic Database for Latin America and the Caribbean). In this Appendix, we describe how we make both sources compatible.

Our perceptions data come from Latinobar\'ometro, which has conducted opinion surveys in 18 LA countries since the 1990s, interviewing about 1,200 individuals per country about individuals' socioeconomic background, and preferences towards political and social issues. Unfortunately, not all years contain questions about individuals' fairness perceptions. The survey was designed to be representative of the voting-age population at the national level (in most LA countries, individuals aged over 18). In Table \ref{tab-coverage} we show what percentage of the voting-age population is represented by the survey in each country for all the years in which the fairness question is available.



\begin{table}[htbp]{\footnotesize
		\begin{center}
			\caption{Coverage of each country's population in Latinobar\'ometro overtime (in \%)}\label{tab-coverage}
			\begin{tabular}{lrrrrrrrrr}
				\midrule
				& 1997 & 2001 & 2002 & 2007 & 2009 & 2010 & 2011 & 2013 & 2015 \\
				\midrule
				Argentina &         68  &         75  &         75  &       100  &       100  &       100  &       100  &       100  &       100  \\
				Bolivia &         32  &         52  &       100  &       100  &       100  &       100  &       100  &       100  &       100  \\
				Brazil &         12  &       100  &       100  &       100  &       100  &       100  &       100  &       100  &       100  \\
				Chile &         70  &         70  &         70  &       100  &       100  &       100  &       100  &       100  &       100  \\
				Colombia &         25  &         71  &         51  &       100  &       100  &       100  &       100  &       100  &       100  \\
				Costa Rica &       100  &       100  &       100  &       100  &       100  &       100  &       100  &       100  &       100  \\
				Dominican Republic & \multicolumn{1}{c}{ N/A } & \multicolumn{1}{c}{ N/A } & \multicolumn{1}{c}{ N/A } &       100  &       100  &       100  &       100  &       100  &       100  \\
				Ecuador &         97  &         97  &       100  &       100  &       100  &       100  &       100  &       100  &       100  \\
				El Salvador &         65  &       100  &       100  &       100  &       100  &       100  &       100  &       100  &       100  \\
				Guatemala &       100  &       100  &       100  &         97  &       100  &       100  &       100  &       100  &       100  \\
				Honduras &       100  &       100  &       100  &         98  &       100  &         99  &         99  &         99  &         99  \\
				Mexico &         93  &         88  &         95  &       100  &       100  &       100  &       100  &       100  &       100  \\
				Nicaragua &       100  &       100  &       100  &       100  &       100  &       100  &       100  &       100  &       100  \\
				Panama &       100  &       100  &       100  &         99  &         99  &         99  &         99  &         99  &         99  \\
				Paraguay &         46  &         46  &         46  &       100  &       100  &       100  &       100  &       100  &       100  \\
				Peru &         52  &         52  &       100  &       100  &       100  &       100  &       100  &       100  &       100  \\
				Uruguay &         80  &         80  &         80  &       100  &       100  &       100  &       100  &       100  &       100  \\
				Venezuela &       100  &       100  &       100  &       100  &         93  &       100  &       100  &       100  &       100  \\
				Weighted average &         68  &         86  &         91  &       100  &       100  &       100  &       100  &       100  &       100  \\
				\midrule
			\end{tabular}
		\end{center}
		%		\begin{singlespace}  \vspace{-.5cm}
			%			\noindent \justify \textbf{Notes:} This table presents the percentage of the voting-age population represented each year in Latinobar\'ometro overtime. The regional average is calculated by weighting each country's population. N/A means that Latinobar\'ometro did not conduct the opinion poll in that particular country-year.
			%		\end{singlespace}
	}
\end{table}


Since our goal is to analyze how unfairness perceptions evolved vis-\`a-vis changes in income inequality, we put a lot of effort into getting income inequality data for each data point for which we have perceptions data available. We made two partial fixes to increase the number of observations available (without pushing the data too much). First, we filled the data gaps using household surveys of relatively close years in which previously unused data were available (see Appendix Table \ref{tab-circa}). For instance, Chile conducts household surveys on average every two years. In 1997, there is perceptions data available, but no data on income inequality. Therefore, we use the inequality data from an adjacent year (1998). As noted previously, we only use data from close years if the data from the adjacent year correspond to a year in which the perceptions question was not asked (and therefore, inequality data are not needed in that year).

\begin{table}[htbp]{\footnotesize
		\begin{center}
			\caption{Circa years used to fill data gaps}\label{tab-circa}
			\begin{tabular}{lcc}
				\midrule
				Country & Year without household data & Data point used instead \\
				\midrule
				Chile & 1997 & 1998 \\
				Chile & 2001 & 2000 \\
				Chile & 2002 & 2003 \\
				Chile & 2007 & 2006 \\
				Colombia & 2007 & 2008 \\
				Ecuador & 2002 & 2003 \\
				El Salvador & 1997 & 1998 \\
				Guatemala & 2001 & 2000 \\
				Guatemala & 2015 & 2014 \\
				Mexico & 1997 & 1998 \\
				Mexico & 2001 & 2000 \\
				Mexico & 2007 & 2006 \\
				Mexico & 2009 & 2008 \\
				Mexico & 2011 & 2012 \\
				Mexico & 2015 & 2014 \\
				Nicaragua & 1997 & 1998 \\
				Nicaragua & 2007 & 2005 \\
				Nicaragua & 2015 & 2014 \\
				Venezuela & 2013 & 2012 \\
				\midrule
			\end{tabular}
		\end{center}
		\begin{singlespace}  \vspace{-.5cm}
			\noindent \justify %\textbf{Notes:}
		\end{singlespace}
	}
\end{table}

Our second partial fix involves interpolating inequality indicators for some years. For some countries, a few years had perceptions data available but no comparable household survey over time and no close year available. In this case, and to analyze the same set of countries every year, interpolation was applied to the inequality indicators (see Appendix Table \ref{tab-interp}).

\begin{table}[htbp]{\footnotesize
		\begin{center}
			\caption{Years in which inequality indicators were calculated with a linear interpolation}\label{tab-interp}
			\begin{tabular}{ll}
				\midrule
				Country & Years interpolated \\
				\midrule
				Argentina & 1997, 2001, and 2002 \\
				Bolivia & 2010 \\
				Brazil & 2010 \\
				Chile & 2010 \\
				Colombia & 1997 \\
				Costa Rica & 1997, 2001, 2002, 2007, and 2009 \\
				Ecuador & 1997, 2001 \\
				Guatemala & 1997, 2002, 2009, 2010, and 2013 \\
				Mexico & 2013 \\
				Nicaragua & 2002, 2010, 2011, and 2013 \\
				Panama & 1997, 2001, 2002, and 2007 \\
				Peru & 1997, 2001, 2002 \\
				Venezuela & 2015 \\
				\midrule
			\end{tabular}
		\end{center}
		\begin{singlespace}  \vspace{-.5cm}
			\noindent \justify %\textbf{Notes:}
		\end{singlespace}
	}
\end{table}


Overall, the years in which income inequality was calculated using linear interpolations represent a relatively small share of the total data points (17\% of total). The majority of our inequality data points (69\%) are calculated using a household survey from the same year in which the perceptions polls were conducted, while the remaining 14\% of our inequality indicators are calculated using household surveys from adjacent years. Table \ref{tab-summ-data} summarizes the data sources used in years perceptions data are available.

\begin{table}[htbp]{\footnotesize
		\begin{center}
			\caption{Summary of the data used in every country-year}\label{tab-summ-data}
			\begin{tabular}{r|r|llllllll}
				\multicolumn{1}{r}{} & \multicolumn{1}{r}{1997} & \multicolumn{1}{r}{2001} & \multicolumn{1}{r}{2002} & \multicolumn{1}{r}{2007} & \multicolumn{1}{r}{2009} & \multicolumn{1}{r}{2010} & \multicolumn{1}{r}{2011} & \multicolumn{1}{r}{2013} & \multicolumn{1}{r}{2015} \\
				\cmidrule{2-10}    \multicolumn{1}{l|}{Argentina} & \cellcolor[rgb]{ .608,  .733,  .349}\textcolor[rgb]{ .608,  .733,  .349}{3} & \multicolumn{1}{r|}{\cellcolor[rgb]{ .608,  .733,  .349}\textcolor[rgb]{ .608,  .733,  .349}{3}} & \multicolumn{1}{r|}{\cellcolor[rgb]{ .608,  .733,  .349}\textcolor[rgb]{ .608,  .733,  .349}{3}} & \multicolumn{1}{r|}{\cellcolor[rgb]{ .31,  .506,  .741}\textcolor[rgb]{ .31,  .506,  .741}{1}} & \multicolumn{1}{r|}{\cellcolor[rgb]{ .31,  .506,  .741}\textcolor[rgb]{ .31,  .506,  .741}{1}} & \multicolumn{1}{r|}{\cellcolor[rgb]{ .31,  .506,  .741}\textcolor[rgb]{ .31,  .506,  .741}{1}} & \multicolumn{1}{r|}{\cellcolor[rgb]{ .31,  .506,  .741}\textcolor[rgb]{ .31,  .506,  .741}{1}} & \multicolumn{1}{r|}{\cellcolor[rgb]{ .31,  .506,  .741}\textcolor[rgb]{ .31,  .506,  .741}{1}} & \multicolumn{1}{r|}{\cellcolor[rgb]{ .969,  .588,  .275}\textcolor[rgb]{ .969,  .588,  .275}{2}} \\
				\cmidrule{2-10}    \multicolumn{1}{l|}{Bolivia} & \cellcolor[rgb]{ .31,  .506,  .741}\textcolor[rgb]{ .31,  .506,  .741}{1} & \multicolumn{1}{r|}{\cellcolor[rgb]{ .31,  .506,  .741}\textcolor[rgb]{ .31,  .506,  .741}{1}} & \multicolumn{1}{r|}{\cellcolor[rgb]{ .31,  .506,  .741}\textcolor[rgb]{ .31,  .506,  .741}{1}} & \multicolumn{1}{r|}{\cellcolor[rgb]{ .31,  .506,  .741}\textcolor[rgb]{ .31,  .506,  .741}{1}} & \multicolumn{1}{r|}{\cellcolor[rgb]{ .31,  .506,  .741}\textcolor[rgb]{ .31,  .506,  .741}{1}} & \multicolumn{1}{r|}{\cellcolor[rgb]{ .608,  .733,  .349}\textcolor[rgb]{ .608,  .733,  .349}{3}} & \multicolumn{1}{r|}{\cellcolor[rgb]{ .31,  .506,  .741}\textcolor[rgb]{ .31,  .506,  .741}{1}} & \multicolumn{1}{r|}{\cellcolor[rgb]{ .31,  .506,  .741}\textcolor[rgb]{ .31,  .506,  .741}{1}} & \multicolumn{1}{r|}{\cellcolor[rgb]{ .31,  .506,  .741}\textcolor[rgb]{ .31,  .506,  .741}{1}} \\
				\cmidrule{2-10}    \multicolumn{1}{l|}{Brazil} & \cellcolor[rgb]{ .31,  .506,  .741}\textcolor[rgb]{ .31,  .506,  .741}{1} & \multicolumn{1}{r|}{\cellcolor[rgb]{ .31,  .506,  .741}\textcolor[rgb]{ .31,  .506,  .741}{1}} & \multicolumn{1}{r|}{\cellcolor[rgb]{ .31,  .506,  .741}\textcolor[rgb]{ .31,  .506,  .741}{1}} & \multicolumn{1}{r|}{\cellcolor[rgb]{ .31,  .506,  .741}\textcolor[rgb]{ .31,  .506,  .741}{1}} & \multicolumn{1}{r|}{\cellcolor[rgb]{ .31,  .506,  .741}\textcolor[rgb]{ .31,  .506,  .741}{1}} & \multicolumn{1}{r|}{\cellcolor[rgb]{ .608,  .733,  .349}\textcolor[rgb]{ .608,  .733,  .349}{3}} & \multicolumn{1}{r|}{\cellcolor[rgb]{ .31,  .506,  .741}\textcolor[rgb]{ .31,  .506,  .741}{1}} & \multicolumn{1}{r|}{\cellcolor[rgb]{ .31,  .506,  .741}\textcolor[rgb]{ .31,  .506,  .741}{1}} & \multicolumn{1}{r|}{\cellcolor[rgb]{ .31,  .506,  .741}\textcolor[rgb]{ .31,  .506,  .741}{1}} \\
				\cmidrule{2-10}    \multicolumn{1}{l|}{Chile} & \cellcolor[rgb]{ .969,  .588,  .275}\textcolor[rgb]{ .969,  .588,  .275}{2} & \multicolumn{1}{r|}{\cellcolor[rgb]{ .969,  .588,  .275}\textcolor[rgb]{ .969,  .588,  .275}{2}} & \multicolumn{1}{r|}{\cellcolor[rgb]{ .969,  .588,  .275}\textcolor[rgb]{ .969,  .588,  .275}{2}} & \multicolumn{1}{r|}{\cellcolor[rgb]{ .969,  .588,  .275}\textcolor[rgb]{ .969,  .588,  .275}{2}} & \multicolumn{1}{r|}{\cellcolor[rgb]{ .31,  .506,  .741}\textcolor[rgb]{ .31,  .506,  .741}{1}} & \multicolumn{1}{r|}{\cellcolor[rgb]{ .608,  .733,  .349}\textcolor[rgb]{ .608,  .733,  .349}{3}} & \multicolumn{1}{r|}{\cellcolor[rgb]{ .31,  .506,  .741}\textcolor[rgb]{ .31,  .506,  .741}{1}} & \multicolumn{1}{r|}{\cellcolor[rgb]{ .31,  .506,  .741}\textcolor[rgb]{ .31,  .506,  .741}{1}} & \multicolumn{1}{r|}{\cellcolor[rgb]{ .31,  .506,  .741}\textcolor[rgb]{ .31,  .506,  .741}{1}} \\
				\cmidrule{2-10}    \multicolumn{1}{l|}{Colombia} & \cellcolor[rgb]{ .608,  .733,  .349}\textcolor[rgb]{ .608,  .733,  .349}{3} & \multicolumn{1}{r|}{\cellcolor[rgb]{ .31,  .506,  .741}\textcolor[rgb]{ .31,  .506,  .741}{1}} & \multicolumn{1}{r|}{\cellcolor[rgb]{ .31,  .506,  .741}\textcolor[rgb]{ .31,  .506,  .741}{1}} & \multicolumn{1}{r|}{\cellcolor[rgb]{ .969,  .588,  .275}\textcolor[rgb]{ .969,  .588,  .275}{2}} & \multicolumn{1}{r|}{\cellcolor[rgb]{ .31,  .506,  .741}\textcolor[rgb]{ .31,  .506,  .741}{1}} & \multicolumn{1}{r|}{\cellcolor[rgb]{ .31,  .506,  .741}\textcolor[rgb]{ .31,  .506,  .741}{1}} & \multicolumn{1}{r|}{\cellcolor[rgb]{ .31,  .506,  .741}\textcolor[rgb]{ .31,  .506,  .741}{1}} & \multicolumn{1}{r|}{\cellcolor[rgb]{ .31,  .506,  .741}\textcolor[rgb]{ .31,  .506,  .741}{1}} & \multicolumn{1}{r|}{\cellcolor[rgb]{ .31,  .506,  .741}\textcolor[rgb]{ .31,  .506,  .741}{1}} \\
				\cmidrule{2-10}    \multicolumn{1}{l|}{Costa Rica} & \cellcolor[rgb]{ .608,  .733,  .349}\textcolor[rgb]{ .608,  .733,  .349}{3} & \multicolumn{1}{r|}{\cellcolor[rgb]{ .608,  .733,  .349}\textcolor[rgb]{ .608,  .733,  .349}{3}} & \multicolumn{1}{r|}{\cellcolor[rgb]{ .608,  .733,  .349}\textcolor[rgb]{ .608,  .733,  .349}{3}} & \multicolumn{1}{r|}{\cellcolor[rgb]{ .608,  .733,  .349}\textcolor[rgb]{ .608,  .733,  .349}{3}} & \multicolumn{1}{r|}{\cellcolor[rgb]{ .608,  .733,  .349}\textcolor[rgb]{ .608,  .733,  .349}{3}} & \multicolumn{1}{r|}{\cellcolor[rgb]{ .31,  .506,  .741}\textcolor[rgb]{ .31,  .506,  .741}{1}} & \multicolumn{1}{r|}{\cellcolor[rgb]{ .31,  .506,  .741}\textcolor[rgb]{ .31,  .506,  .741}{1}} & \multicolumn{1}{r|}{\cellcolor[rgb]{ .31,  .506,  .741}\textcolor[rgb]{ .31,  .506,  .741}{1}} & \multicolumn{1}{r|}{\cellcolor[rgb]{ .31,  .506,  .741}\textcolor[rgb]{ .31,  .506,  .741}{1}} \\
				\cmidrule{2-10}    \multicolumn{1}{l|}{Dominican Rep.} & \cellcolor[rgb]{ .749,  .749,  .749}\textcolor[rgb]{ .749,  .749,  .749}{0} & \multicolumn{1}{r|}{\cellcolor[rgb]{ .31,  .506,  .741}\textcolor[rgb]{ .31,  .506,  .741}{1}} & \multicolumn{1}{r|}{\cellcolor[rgb]{ .31,  .506,  .741}\textcolor[rgb]{ .31,  .506,  .741}{1}} & \multicolumn{1}{r|}{\cellcolor[rgb]{ .31,  .506,  .741}\textcolor[rgb]{ .31,  .506,  .741}{1}} & \multicolumn{1}{r|}{\cellcolor[rgb]{ .31,  .506,  .741}\textcolor[rgb]{ .31,  .506,  .741}{1}} & \multicolumn{1}{r|}{\cellcolor[rgb]{ .31,  .506,  .741}\textcolor[rgb]{ .31,  .506,  .741}{1}} & \multicolumn{1}{r|}{\cellcolor[rgb]{ .31,  .506,  .741}\textcolor[rgb]{ .31,  .506,  .741}{1}} & \multicolumn{1}{r|}{\cellcolor[rgb]{ .31,  .506,  .741}\textcolor[rgb]{ .31,  .506,  .741}{1}} & \multicolumn{1}{r|}{\cellcolor[rgb]{ .31,  .506,  .741}\textcolor[rgb]{ .31,  .506,  .741}{1}} \\
				\cmidrule{2-10}    \multicolumn{1}{l|}{Ecuador} & \cellcolor[rgb]{ .608,  .733,  .349}\textcolor[rgb]{ .608,  .733,  .349}{3} & \multicolumn{1}{r|}{\cellcolor[rgb]{ .608,  .733,  .349}\textcolor[rgb]{ .608,  .733,  .349}{3}} & \multicolumn{1}{r|}{\cellcolor[rgb]{ .969,  .588,  .275}\textcolor[rgb]{ .969,  .588,  .275}{2}} & \multicolumn{1}{r|}{\cellcolor[rgb]{ .31,  .506,  .741}\textcolor[rgb]{ .31,  .506,  .741}{1}} & \multicolumn{1}{r|}{\cellcolor[rgb]{ .31,  .506,  .741}\textcolor[rgb]{ .31,  .506,  .741}{1}} & \multicolumn{1}{r|}{\cellcolor[rgb]{ .31,  .506,  .741}\textcolor[rgb]{ .31,  .506,  .741}{1}} & \multicolumn{1}{r|}{\cellcolor[rgb]{ .31,  .506,  .741}\textcolor[rgb]{ .31,  .506,  .741}{1}} & \multicolumn{1}{r|}{\cellcolor[rgb]{ .31,  .506,  .741}\textcolor[rgb]{ .31,  .506,  .741}{1}} & \multicolumn{1}{r|}{\cellcolor[rgb]{ .31,  .506,  .741}\textcolor[rgb]{ .31,  .506,  .741}{1}} \\
				\cmidrule{2-10}    \multicolumn{1}{l|}{El Salvador} & \cellcolor[rgb]{ .31,  .506,  .741}\textcolor[rgb]{ .31,  .506,  .741}{1} & \multicolumn{1}{r|}{\cellcolor[rgb]{ .31,  .506,  .741}\textcolor[rgb]{ .31,  .506,  .741}{1}} & \multicolumn{1}{r|}{\cellcolor[rgb]{ .31,  .506,  .741}\textcolor[rgb]{ .31,  .506,  .741}{1}} & \multicolumn{1}{r|}{\cellcolor[rgb]{ .31,  .506,  .741}\textcolor[rgb]{ .31,  .506,  .741}{1}} & \multicolumn{1}{r|}{\cellcolor[rgb]{ .31,  .506,  .741}\textcolor[rgb]{ .31,  .506,  .741}{1}} & \multicolumn{1}{r|}{\cellcolor[rgb]{ .31,  .506,  .741}\textcolor[rgb]{ .31,  .506,  .741}{1}} & \multicolumn{1}{r|}{\cellcolor[rgb]{ .31,  .506,  .741}\textcolor[rgb]{ .31,  .506,  .741}{1}} & \multicolumn{1}{r|}{\cellcolor[rgb]{ .31,  .506,  .741}\textcolor[rgb]{ .31,  .506,  .741}{1}} & \multicolumn{1}{r|}{\cellcolor[rgb]{ .31,  .506,  .741}\textcolor[rgb]{ .31,  .506,  .741}{1}} \\
				\cmidrule{2-10}    \multicolumn{1}{l|}{Guatemala} & \cellcolor[rgb]{ .969,  .588,  .275}\textcolor[rgb]{ .969,  .588,  .275}{2} & \multicolumn{1}{r|}{\cellcolor[rgb]{ .969,  .588,  .275}\textcolor[rgb]{ .969,  .588,  .275}{2}} & \multicolumn{1}{r|}{\cellcolor[rgb]{ .608,  .733,  .349}\textcolor[rgb]{ .608,  .733,  .349}{3}} & \multicolumn{1}{r|}{\cellcolor[rgb]{ .969,  .588,  .275}\textcolor[rgb]{ .969,  .588,  .275}{2}} & \multicolumn{1}{r|}{\cellcolor[rgb]{ .608,  .733,  .349}\textcolor[rgb]{ .608,  .733,  .349}{3}} & \multicolumn{1}{r|}{\cellcolor[rgb]{ .608,  .733,  .349}\textcolor[rgb]{ .608,  .733,  .349}{3}} & \multicolumn{1}{r|}{\cellcolor[rgb]{ .31,  .506,  .741}\textcolor[rgb]{ .31,  .506,  .741}{1}} & \multicolumn{1}{r|}{\cellcolor[rgb]{ .608,  .733,  .349}\textcolor[rgb]{ .608,  .733,  .349}{3}} & \multicolumn{1}{r|}{\cellcolor[rgb]{ .969,  .588,  .275}\textcolor[rgb]{ .969,  .588,  .275}{2}} \\
				\cmidrule{2-10}    \multicolumn{1}{l|}{Honduras} & \cellcolor[rgb]{ .31,  .506,  .741}\textcolor[rgb]{ .31,  .506,  .741}{1} & \multicolumn{1}{r|}{\cellcolor[rgb]{ .31,  .506,  .741}\textcolor[rgb]{ .31,  .506,  .741}{1}} & \multicolumn{1}{r|}{\cellcolor[rgb]{ .31,  .506,  .741}\textcolor[rgb]{ .31,  .506,  .741}{1}} & \multicolumn{1}{r|}{\cellcolor[rgb]{ .31,  .506,  .741}\textcolor[rgb]{ .31,  .506,  .741}{1}} & \multicolumn{1}{r|}{\cellcolor[rgb]{ .31,  .506,  .741}\textcolor[rgb]{ .31,  .506,  .741}{1}} & \multicolumn{1}{r|}{\cellcolor[rgb]{ .31,  .506,  .741}\textcolor[rgb]{ .31,  .506,  .741}{1}} & \multicolumn{1}{r|}{\cellcolor[rgb]{ .31,  .506,  .741}\textcolor[rgb]{ .31,  .506,  .741}{1}} & \multicolumn{1}{r|}{\cellcolor[rgb]{ .31,  .506,  .741}\textcolor[rgb]{ .31,  .506,  .741}{1}} & \multicolumn{1}{r|}{\cellcolor[rgb]{ .31,  .506,  .741}\textcolor[rgb]{ .31,  .506,  .741}{1}} \\
				\cmidrule{2-10}    \multicolumn{1}{l|}{Mexico} & \cellcolor[rgb]{ .969,  .588,  .275}\textcolor[rgb]{ .969,  .588,  .275}{2} & \multicolumn{1}{r|}{\cellcolor[rgb]{ .969,  .588,  .275}\textcolor[rgb]{ .969,  .588,  .275}{2}} & \multicolumn{1}{r|}{\cellcolor[rgb]{ .31,  .506,  .741}\textcolor[rgb]{ .31,  .506,  .741}{1}} & \multicolumn{1}{r|}{\cellcolor[rgb]{ .969,  .588,  .275}\textcolor[rgb]{ .969,  .588,  .275}{2}} & \multicolumn{1}{r|}{\cellcolor[rgb]{ .969,  .588,  .275}\textcolor[rgb]{ .969,  .588,  .275}{2}} & \multicolumn{1}{r|}{\cellcolor[rgb]{ .31,  .506,  .741}\textcolor[rgb]{ .31,  .506,  .741}{1}} & \multicolumn{1}{r|}{\cellcolor[rgb]{ .969,  .588,  .275}\textcolor[rgb]{ .969,  .588,  .275}{2}} & \multicolumn{1}{r|}{\cellcolor[rgb]{ .608,  .733,  .349}\textcolor[rgb]{ .608,  .733,  .349}{3}} & \multicolumn{1}{r|}{\cellcolor[rgb]{ .969,  .588,  .275}\textcolor[rgb]{ .969,  .588,  .275}{2}} \\
				\cmidrule{2-10}    \multicolumn{1}{l|}{Nicaragua} & \cellcolor[rgb]{ .969,  .588,  .275}\textcolor[rgb]{ .969,  .588,  .275}{2} & \multicolumn{1}{r|}{\cellcolor[rgb]{ .31,  .506,  .741}\textcolor[rgb]{ .31,  .506,  .741}{1}} & \multicolumn{1}{r|}{\cellcolor[rgb]{ .608,  .733,  .349}\textcolor[rgb]{ .608,  .733,  .349}{3}} & \multicolumn{1}{r|}{\cellcolor[rgb]{ .969,  .588,  .275}\textcolor[rgb]{ .969,  .588,  .275}{2}} & \multicolumn{1}{r|}{\cellcolor[rgb]{ .31,  .506,  .741}\textcolor[rgb]{ .31,  .506,  .741}{1}} & \multicolumn{1}{r|}{\cellcolor[rgb]{ .608,  .733,  .349}\textcolor[rgb]{ .608,  .733,  .349}{3}} & \multicolumn{1}{r|}{\cellcolor[rgb]{ .608,  .733,  .349}\textcolor[rgb]{ .608,  .733,  .349}{3}} & \multicolumn{1}{r|}{\cellcolor[rgb]{ .608,  .733,  .349}\textcolor[rgb]{ .608,  .733,  .349}{3}} & \multicolumn{1}{r|}{\cellcolor[rgb]{ .969,  .588,  .275}\textcolor[rgb]{ .969,  .588,  .275}{2}} \\
				\cmidrule{2-10}    \multicolumn{1}{l|}{Panama} & \cellcolor[rgb]{ .608,  .733,  .349}\textcolor[rgb]{ .608,  .733,  .349}{3} & \multicolumn{1}{r|}{\cellcolor[rgb]{ .608,  .733,  .349}\textcolor[rgb]{ .608,  .733,  .349}{3}} & \multicolumn{1}{r|}{\cellcolor[rgb]{ .608,  .733,  .349}\textcolor[rgb]{ .608,  .733,  .349}{3}} & \multicolumn{1}{r|}{\cellcolor[rgb]{ .608,  .733,  .349}\textcolor[rgb]{ .608,  .733,  .349}{3}} & \multicolumn{1}{r|}{\cellcolor[rgb]{ .31,  .506,  .741}\textcolor[rgb]{ .31,  .506,  .741}{1}} & \multicolumn{1}{r|}{\cellcolor[rgb]{ .31,  .506,  .741}\textcolor[rgb]{ .31,  .506,  .741}{1}} & \multicolumn{1}{r|}{\cellcolor[rgb]{ .31,  .506,  .741}\textcolor[rgb]{ .31,  .506,  .741}{1}} & \multicolumn{1}{r|}{\cellcolor[rgb]{ .31,  .506,  .741}\textcolor[rgb]{ .31,  .506,  .741}{1}} & \multicolumn{1}{r|}{\cellcolor[rgb]{ .31,  .506,  .741}\textcolor[rgb]{ .31,  .506,  .741}{1}} \\
				\cmidrule{2-10}    \multicolumn{1}{l|}{Paraguay} & \cellcolor[rgb]{ .31,  .506,  .741}\textcolor[rgb]{ .31,  .506,  .741}{1} & \multicolumn{1}{r|}{\cellcolor[rgb]{ .31,  .506,  .741}\textcolor[rgb]{ .31,  .506,  .741}{1}} & \multicolumn{1}{r|}{\cellcolor[rgb]{ .31,  .506,  .741}\textcolor[rgb]{ .31,  .506,  .741}{1}} & \multicolumn{1}{r|}{\cellcolor[rgb]{ .31,  .506,  .741}\textcolor[rgb]{ .31,  .506,  .741}{1}} & \multicolumn{1}{r|}{\cellcolor[rgb]{ .31,  .506,  .741}\textcolor[rgb]{ .31,  .506,  .741}{1}} & \multicolumn{1}{r|}{\cellcolor[rgb]{ .31,  .506,  .741}\textcolor[rgb]{ .31,  .506,  .741}{1}} & \multicolumn{1}{r|}{\cellcolor[rgb]{ .31,  .506,  .741}\textcolor[rgb]{ .31,  .506,  .741}{1}} & \multicolumn{1}{r|}{\cellcolor[rgb]{ .31,  .506,  .741}\textcolor[rgb]{ .31,  .506,  .741}{1}} & \multicolumn{1}{r|}{\cellcolor[rgb]{ .31,  .506,  .741}\textcolor[rgb]{ .31,  .506,  .741}{1}} \\
				\cmidrule{2-10}    \multicolumn{1}{l|}{Peru} & \cellcolor[rgb]{ .969,  .588,  .275}\textcolor[rgb]{ .969,  .588,  .275}{2} & \multicolumn{1}{r|}{\cellcolor[rgb]{ .31,  .506,  .741}\textcolor[rgb]{ .31,  .506,  .741}{1}} & \multicolumn{1}{r|}{\cellcolor[rgb]{ .31,  .506,  .741}\textcolor[rgb]{ .31,  .506,  .741}{1}} & \multicolumn{1}{r|}{\cellcolor[rgb]{ .31,  .506,  .741}\textcolor[rgb]{ .31,  .506,  .741}{1}} & \multicolumn{1}{r|}{\cellcolor[rgb]{ .31,  .506,  .741}\textcolor[rgb]{ .31,  .506,  .741}{1}} & \multicolumn{1}{r|}{\cellcolor[rgb]{ .31,  .506,  .741}\textcolor[rgb]{ .31,  .506,  .741}{1}} & \multicolumn{1}{r|}{\cellcolor[rgb]{ .31,  .506,  .741}\textcolor[rgb]{ .31,  .506,  .741}{1}} & \multicolumn{1}{r|}{\cellcolor[rgb]{ .31,  .506,  .741}\textcolor[rgb]{ .31,  .506,  .741}{1}} & \multicolumn{1}{r|}{\cellcolor[rgb]{ .31,  .506,  .741}\textcolor[rgb]{ .31,  .506,  .741}{1}} \\
				\cmidrule{2-10}    \multicolumn{1}{l|}{Uruguay} & \cellcolor[rgb]{ .31,  .506,  .741}\textcolor[rgb]{ .31,  .506,  .741}{1} & \multicolumn{1}{r|}{\cellcolor[rgb]{ .31,  .506,  .741}\textcolor[rgb]{ .31,  .506,  .741}{1}} & \multicolumn{1}{r|}{\cellcolor[rgb]{ .31,  .506,  .741}\textcolor[rgb]{ .31,  .506,  .741}{1}} & \multicolumn{1}{r|}{\cellcolor[rgb]{ .31,  .506,  .741}\textcolor[rgb]{ .31,  .506,  .741}{1}} & \multicolumn{1}{r|}{\cellcolor[rgb]{ .31,  .506,  .741}\textcolor[rgb]{ .31,  .506,  .741}{1}} & \multicolumn{1}{r|}{\cellcolor[rgb]{ .31,  .506,  .741}\textcolor[rgb]{ .31,  .506,  .741}{1}} & \multicolumn{1}{r|}{\cellcolor[rgb]{ .31,  .506,  .741}\textcolor[rgb]{ .31,  .506,  .741}{1}} & \multicolumn{1}{r|}{\cellcolor[rgb]{ .31,  .506,  .741}\textcolor[rgb]{ .31,  .506,  .741}{1}} & \multicolumn{1}{r|}{\cellcolor[rgb]{ .31,  .506,  .741}\textcolor[rgb]{ .31,  .506,  .741}{1}} \\
				\cmidrule{2-10}    \multicolumn{1}{l|}{Venezuela} & \cellcolor[rgb]{ .31,  .506,  .741}\textcolor[rgb]{ .31,  .506,  .741}{1} & \multicolumn{1}{r|}{\cellcolor[rgb]{ .31,  .506,  .741}\textcolor[rgb]{ .31,  .506,  .741}{1}} & \multicolumn{1}{r|}{\cellcolor[rgb]{ .31,  .506,  .741}\textcolor[rgb]{ .31,  .506,  .741}{1}} & \multicolumn{1}{r|}{\cellcolor[rgb]{ .31,  .506,  .741}\textcolor[rgb]{ .31,  .506,  .741}{1}} & \multicolumn{1}{r|}{\cellcolor[rgb]{ .31,  .506,  .741}\textcolor[rgb]{ .31,  .506,  .741}{1}} & \multicolumn{1}{r|}{\cellcolor[rgb]{ .31,  .506,  .741}\textcolor[rgb]{ .31,  .506,  .741}{1}} & \multicolumn{1}{r|}{\cellcolor[rgb]{ .31,  .506,  .741}\textcolor[rgb]{ .31,  .506,  .741}{1}} & \multicolumn{1}{r|}{\cellcolor[rgb]{ .969,  .588,  .275}\textcolor[rgb]{ .969,  .588,  .275}{2}} & \multicolumn{1}{r|}{\cellcolor[rgb]{ .608,  .733,  .349}\textcolor[rgb]{ .608,  .733,  .349}{3}} \\
				\cmidrule{2-10}    \multicolumn{1}{r}{} & \multicolumn{1}{r}{} &   &   &   &   &   &   &   &  \\
				\cmidrule{2-2}      & \cellcolor[rgb]{ .31,  .506,  .741} & \multicolumn{8}{l}{Both perceptions and inequality data available} \\
				\cmidrule{2-2}      & \cellcolor[rgb]{ .969,  .588,  .275} & \multicolumn{8}{l}{Inequality was calculated with a close survey} \\
				\cmidrule{2-2}      & \cellcolor[rgb]{ .608,  .733,  .349} & \multicolumn{8}{l}{Inequality was calculated with a linear interpolation} \\
				\cmidrule{2-2}      & \cellcolor[rgb]{ .749,  .749,  .749} & \multicolumn{8}{l}{Latinobar\'ometro did not conduct survey in this year} \\
				\cmidrule{2-2}    \end{tabular}%
		\end{center}
		\begin{singlespace}  \vspace{-.5cm}
			\noindent \justify %\textbf{Notes:}
		\end{singlespace}
	}
\end{table}

\subsection{Imputation of Missing Values for the Regression Analysis}

Two of our individual-level variables (political ideology and religion) have many missing values in some country-years. To deal with this in our regressions, we imputed the average value of each variable to individuals with a missing value. In those cases, we included in the regression a dummy that takes the value one if the value of the variable was imputed and zero otherwise. The results are similar if we do not impute the values, but the sample size of the regressions is smaller.

\subsection{Comparison between Latinobar\'ometro's and SEDLAC's samples} \label{app_lb_sedlac}

To assess whether there are systematic differences between Latinobar\'ometro's sample and the household surveys' sample, in Appendix Table \ref{tab-lat-sedlac} we compare a set of variables available in both datasets during 2013. To ensure comparability across databases, we restrict the calculations to individuals over age 18 and countries with data available in both harmonization projects.

In general, the samples are similar in observable characteristics. For instance, the average age in Latinobar\'ometro's 2013 sample is 40.6 years, while in SEDLAC it is 42.7 years. Similarly, the percentage of males is 48.9\% in Latinobar\'ometro and 47.6\% in SEDLAC. The main difference arises from educational attainment. On average, the SEDLAC sample is more educated: 46.1\% of the population has secondary education or more, while this figure is 38.8\% in Latinobar\'ometro.


% Comparison of descriptive statistics in Latinobar\'ometro and SEDLAC, 2013
\begin{table}[H]{\scriptsize
		\begin{center}
			\caption{Descriptive statistics in Latinobar\'ometro and SEDLAC, 2013 (selected countries)} \label{tab-lat-sedlac}
			\begin{tabular}{lcccccccc}
				\midrule
				& \multicolumn{2}{c}{Mean} &   & \multicolumn{2}{c}{Standard Dev.} &   & \multicolumn{2}{c}{Observations} \\
				\cmidrule{2-3}\cmidrule{5-6}\cmidrule{8-9}      &  Latinob.  &  SEDLAC  &   &  Latinob.  &  SEDLAC  &   &  Latinob.  &  SEDLAC  \\
				& (1) & (2) &   & (3) & (4) &   & (5) & (6) \\
				\midrule
				\textbf{\hspace{-1em} Panel A. Sociodemographic} &   &   &   &   &   &   &   &  \\
				Age & 40.59 & 42.68 &   & 16.43 & 17.25 &   &      14,855  &   1,004,894  \\
				Male (\%) & 48.97 & 47.63 &   & 0.50 & 0.50 &   &      14,855  &   1,004,894  \\
				Married or civil union (\%) & 56.77 & 36.41 &   & 0.50 & 0.48 &   &      14,804  &      915,117  \\
				\multicolumn{3}{l}{\textbf{\hspace{-1em} Panel B. Education and Labor market}} &   &   &   &   &   &  \\
				Literate (\%) & 91.18 & 92.17 &   & 0.28 & 0.27 &   &      14,855  &   1,004,744  \\
				Secondary education or more (\%) & 38.83 & 46.11 &   & 0.49 & 0.50 &   &      14,855  &   1,001,672  \\
				Economically active (\%) & 65.14 & 68.66 &   & 0.48 & 0.46 &   &      14,855  &   1,004,718  \\
				Unemployed (\%) & 5.78 & 4.08 &   & 0.23 & 0.20 &   &      14,855  &   1,004,718  \\
				\textbf{\hspace{-1em} Panel C. Assets and Services} &   &   &   &   &   &   &   &  \\
				Access to a sewerage (\%) & 68.76 & 63.41 &   & 0.46 & 0.48 &   &      13,799  &      975,726  \\
				Car (\%) & 26.37 & 21.09 &   & 0.44 & 0.41 &   &      11,612  &      643,350  \\
				Computer (\%) & 46.55 & 47.82 &   & 0.50 & 0.50 &   &      12,747  &      894,003  \\
				Fridge (\%) & 82.76 & 88.89 &   & 0.38 & 0.31 &   &      12,763  &      894,003  \\
				Homeowner (\%) & 74.09 & 69.64 &   & 0.44 & 0.46 &   &      14,761  &   1,003,306  \\
				Mobile (\%) & 86.91 & 91.78 &   & 0.34 & 0.27 &   &      12,754  &      896,079  \\
				Washing machine (\%) & 60.49 & 56.88 &   & 0.49 & 0.50 &   &      11,816  &      848,350  \\
				Landline (\%) & 40.22 & 39.47 &   & 0.49 & 0.49 &   &      12,736  &      896,425  \\
				\midrule
			\end{tabular}
		\end{center}
		\begin{singlespace}
			\noindent \justify \textbf{Note:} This table compares the observable characteristics of individuals in Latinobar\'ometro and SEDLAC. Summary statistics were calculated on a restricted sample (individuals aged over 18) to ensure comparability between both datasets, pooling data from 14 countries in 2013: Argentina, Bolivia, Brazil, Chile, Colombia, Costa Rica, Dominican Republic, Ecuador, El Salvador, Honduras, Panama, Peru, Paraguay, and Uruguay.
		\end{singlespace}

	}
\end{table}


%%%%%%%%%%%%%%%%%%%%%%%%%%%%%%%%%%%%%%%%%%%%%%%%%%%%%%%%%%%%%%%%%%%%%%%%%%%%%%%
% APPENDIX C: OAXACA-BLINDER DECOMPOSITION
%%%%%%%%%%%%%%%%%%%%%%%%%%%%%%%%%%%%%%%%%%%%%%%%%%%%%%%%%%%%%%%%%%%%%%%%%%%%%%%

\clearpage
\section{The Oaxaca-Blinder Decomposition} \label{sec_oaxaca}

\setcounter{table}{0}
\setcounter{figure}{0}
\setcounter{equation}{0}
\renewcommand{\thetable}{C\arabic{table}}
\renewcommand{\thefigure}{C\arabic{figure}}
\renewcommand{\theequation}{C\arabic{equation}}

The starting point to decompose changes in unfairness perceptions between 2002 and 2013 is the following equation:
%
\begin{align}
	\text{Unfair}_{ict} = \beta_t X_{ict} + \gamma_t \text{Gini}_{ct} + \varepsilon_{ict} \quad \text{for} \quad t \in \{2002, 2013\},
\end{align}
%
where $t$ indicates the year in which perceptions are elicited and $X_{ict}$ is a vector that contains individual-level controls. The fraction of individuals who perceive the income distribution as unfair in year $t$ can be calculated as
%
\begin{align}
	\overline{\text{Unfair}}_{t} = \hat{\beta}_{t} \bar{X}_{t} + \hat{\gamma}_{t} \overline{\text{Gini}}_{t}  \quad \text{for} \quad t \in \{2002, 2013\},
\end{align}
%
where $\bar{X}_t$ is a vector of the average values of the explanatory variables in year $t$, and $\hat{\beta}$ the vector of OLS-estimated coefficients. The change in unfairness beliefs between 2013 and 2002 is given by
%
\begin{align} \label{eq-diff-fair}
	\underbrace{\overline{\text{Unfair}}_{2013} - \overline{\text{Unfair}}_{2013}}_{\equiv \Delta \text{Unfair}} = (\hat{\beta}_{2013} \bar{X}_{2013} + \hat{\gamma}_{2013} \overline{\text{Gini}}_{2013}) - (\hat{\beta}_{2002} \bar{X}_{2002} + \hat{\gamma}_{2002} \overline{\text{Gini}}_{2002})
\end{align}

Adding and subtracting $\hat{\beta}_{2002} \bar{X}_{2013} + \hat{\gamma}_{2002} \overline{\text{Gini}}_{2013}$ to equation \eqref{eq-diff-fair} yields
%
\begin{align} \label{eq_oaxaca}
	\Delta \text{Unfair} &=
	\underbrace{\hat{\beta}_{2002} (\bar{X}_{2013} - \bar{X}_{2002})}_{\equiv \Delta \text{Demog.}} +
	\underbrace{\hat{\gamma}_{2002} (\overline{\text{Gini}}_{2013} - \overline{\text{Gini}}_{2002})}_{\equiv \Delta \text{Gini}} \notag \\ &+ \underbrace{\bar{X}_{2013} (\hat{\beta}_{2013} - \hat{\beta}_{2002})+ \overline{\text{Gini}}_{2013} (\hat{\gamma}_{2013} - \hat{\gamma}_{2002})}_{\text{Residual}}
\end{align}
%

The first two terms of equation \eqref{eq_oaxaca} are usually known as the ``composition effect.'' These effects capture the difference between the average perceptions in 2002 and the counterfactual perceptions 2013 had the $\hat{\beta}$'s and $\hat{\gamma}$---i.e., the elasticity of perceptions to the different covariables---remained constant during the 2002--13 period. The first term captures differences in individual-level demographic variables that determine unfairness perceptions in the model (such as educational attainment, age, and employment status). The second term captures changes in aggregate trends in income inequality.

The third term of \eqref{eq_oaxaca} reflects the difference between the average fairness views in 2013 and the counterfactual fairness views in 2002 with the observable attributes of 2013. Thus, this component reflects changes in fairness views due to changes in the elasticity of the different covariables between both years. Since we cannot explain why the coefficients attached to each variable changed, this term is usually viewed as the ``unexplained'' part of the decomposition and treated as the residual of the decomposition.


% End standalone document
\ifdefined\standalonetrue
  \end{document}
\fi
 after \appendix in main document

\ifdefined\appendixstandalone
\else
  % Standalone preamble - only used when compiling this file directly
  \newif\ifstandalone
  \standalonetrue

  \documentclass[11pt]{article}
  \usepackage[a4paper,pdftex]{geometry}
  \setlength{\oddsidemargin}{3mm}
  \setlength{\evensidemargin}{3mm}
  \usepackage[T1]{fontenc}
  \usepackage[utf8x]{inputenc}
  \usepackage{amsmath,amsthm,amssymb,graphicx,enumerate,booktabs,bigstrut,rotating,multirow,float,caption,etoolbox,hyperref,lmodern,comment,listings,qtree,a4wide,titlesec,moresize,bbm,subcaption,tikz,pdfescape,mathtools,flafter,enumitem,setspace,ragged2e,colortbl,lineno,lscape}
  \hypersetup{colorlinks,linkcolor={red},citecolor={blue},urlcolor={blue}}

  \usepackage[justification=centering,font=small]{caption}

  \usepackage[authoryear]{natbib}
  \usepackage[english]{babel}
  \renewcommand{\baselinestretch}{1.25}
  \DeclareMathOperator{\E}{\mathbb{E}}
  \DeclareMathOperator*{\argmax}{argmax}
  \makeatletter
  \renewcommand\subsubsection{\@startsection{subsubsection}{3}{\z@}%
  	{-3.25ex\@plus -1ex \@minus -.2ex}%
  	{-1.5ex \@plus -.2ex}%
  	{\normalfont\normalsize\bfseries}}
  \def\@biblabel#1{\hspace*{-\labelsep}}

  \newcommand*\circled[1]{\tikz[baseline=(char.base)]{
  		\node[shape=circle,draw,inner sep=2pt] (char) {#1};}}
  \newcommand*\ExpandableInput[1]{\@@input#1 }
  \makeatother

  \def\sym#1{\ifmmode^{#1}\else\(^{#1}\)\fi}
  \onehalfspacing
  \pdfpageheight\paperheight
  \pdfpagewidth\paperwidth

  \newcolumntype{L}[1]{>{\raggedright\let\newline\\\arraybackslash\hspace{0pt}}m{#1}}
  \newcolumntype{C}[1]{>{\centering\let\newline\\\arraybackslash\hspace{0pt}}m{#1}}
  \newcolumntype{R}[1]{>{\raggedleft\let\newline\\\arraybackslash\hspace{0pt}}m{#1}}

  \begin{document}

  \title{Are Fairness Perceptions Shaped by Income Inequality? \\ Evidence from Latin America \\ \vspace{10pt} \Large Online Appendix}
  \author{Leonardo Gasparini \and Germ\'an Reyes}
  \date{\vspace{15pt} February 2022}
  \maketitle

  \appendix
\fi

%%%%%%%%%%%%%%%%%%%%%%%%%%%%%%%%%%%%%%%%%%%%%%%%%%%%%%%%%%%%%%%%%%%%%%%%%%%%%%%
% APPENDIX A: ADDITIONAL FIGURES AND TABLES
%%%%%%%%%%%%%%%%%%%%%%%%%%%%%%%%%%%%%%%%%%%%%%%%%%%%%%%%%%%%%%%%%%%%%%%%%%%%%%%

\begin{center}\noindent{\LARGE \textbf{Appendix}}\end{center}

\section{Additional Figures and Tables} \label{app_add_figs}

\setcounter{table}{0}
\setcounter{figure}{0}
\setcounter{equation}{0}
\renewcommand{\thetable}{A\arabic{table}}
\renewcommand{\thefigure}{A\arabic{figure}}
\renewcommand{\theequation}{A\arabic{equation}}


\begin{figure}[htpb]
	\caption{Perceptions of unfairness and individual characteristics, 1997--2015} \label{fig-fairness-grps}
	\centering
	\begin{subfigure}[t]{.48\textwidth}
		\caption*{Panel A. By age}\label{fig-fairness-age}
		\centering
		\includegraphics[width=\linewidth]{../results/fig-fairness-age}
	\end{subfigure}
	\hfill
	\begin{subfigure}[t]{0.48\textwidth}
		\caption*{Panel B. By sex}\label{fig-fairness-sex}
		\centering
		\includegraphics[width=\linewidth]{../results/fig-fairness-sex}
	\end{subfigure}
	\hfill
	\begin{subfigure}[t]{.48\textwidth}
		\caption*{Panel C. By educational attainment}\label{fig-fairness-educ}
		\centering
		\includegraphics[width=\linewidth]{../results/fig-fairness-educ}
	\end{subfigure}
	\hfill
	\begin{subfigure}[t]{0.48\textwidth}
		\caption*{Panel D. By employment status}\label{fig-fairness-employ}
		\centering
		\includegraphics[width=\linewidth]{../results/fig-fairness-employ}
	\end{subfigure}
	\hfill
	{\footnotesize
		\singlespacing \justify

		\textbf{Notes:} This figure shows the share of individuals who perceive the income distribution as unfair or very unfair according to their age, gender, maximum educational attainment, and employment status.


	}
\end{figure}

\clearpage
\begin{figure}[htp]
	\caption{The evolution of fairness views and income inequality in Latin America}\label{fig-timeseries-gini-unfair}  \centering
	\centering
	\includegraphics[width=.75\linewidth]{../results/fig-timeseries-gini-unfair}
	\hfill
	{\footnotesize
		\singlespacing \justify

		\textbf{Notes:} This figure shows the evolution of the average Gini coefficient across countries in our sample (right-hand-side variable) and the fraction of the population who perceive the income distribution as unfair or very unfair (left-hand-side variable) over 1997--2015. To have a balanced panel of countries over time, we linearly extrapolated the Gini coefficient in years in which income microdata is not available (see Appendix \ref{sec_data}).

	}
\end{figure}



\clearpage
\begin{table}[H]{\footnotesize
		\begin{center}
			\caption{Descriptive statistics of our sample} \label{tab-sum-stats}
			\begin{tabular}{lccc}
				\midrule
				&  Mean  &  Standard Dev.  & Observations \\
				& (1) & (2) & (3) \\
				\midrule
				\textbf{\hspace{-1em} Panel A. Sociodemographic} &   &   &  \\
				Age & 39.75 & 16.23 &         225,551  \\
				Male (\%) & 48.97 & 0.50 &         225,567  \\
				Married or civil union (\%) & 56.27 & 0.50 &         224,081  \\
				Catholic religion (\%) & 68.01 & 0.47 &         222,790  \\
				Ideology (10 = right-wing) & 5.48 & 2.64 &         131,980  \\
				\textbf{\hspace{-1em} Panel B. Education and Labor market} &   &   &  \\
				Literate (\%) & 90.31 & 0.30 &         224,056  \\
				Secondary education or more (\%) & 33.65 & 0.47 &         224,056  \\
				Parents with secondary education (\%) & 17.43 & 0.38 &         184,884  \\
				Economically active (\%) & 64.14 & 0.48 &         225,222  \\
				Unemployed (\% Labor Force) & 9.89 & 0.30 &         225,222  \\
				\textbf{\hspace{-1em} Panel C. Access to services} &   &   &  \\
				Access to a sewerage (\%) & 69.59 & 0.46 &         222,530  \\
				Access to running water (\%) & 88.83 & 0.31 &         204,340  \\
				\textbf{\hspace{-1em} Panel D. Asset ownership} &   &   &  \\
				Car (\%) & 28.21 & 0.45 &         222,338  \\
				Computer (\%) & 33.79 & 0.47 &         222,645  \\
				Fridge (\%) & 79.22 & 0.41 &         146,686  \\
				Homeowner (\%) & 73.92 & 0.44 &         223,603  \\
				Mobile (\%) & 80.61 & 0.40 &         172,253  \\
				Washing machine (\%) & 54.71 & 0.50 &         223,122  \\
				Landline (\%) & 42.28 & 0.49 &         222,968  \\
				\midrule
			\end{tabular}
		\end{center}
		\begin{singlespace} \vspace{-.5cm}
			\noindent \justify \textbf{Note:} This table shows summary statistics on our sample pooling data from all countries in our sample over 1997--2015.
		\end{singlespace}
	}
\end{table}


% Fairness perceptions by population group pooling data 1997-2015, in %
\clearpage
\begin{table}[H]{\footnotesize
		\begin{center}
			\caption{Fairness views by population group} \label{tab-fairness-grp}
			\newcommand\w{1.70}
			\begin{tabular}{l@{}R{\w cm}R{\w cm}R{\w cm}R{\w cm}}
				\midrule
				& \multicolumn{4}{c}{\% of individuals who believe income distribution is:} \\
				\cmidrule{2-5}  & Very unfair & Unfair & Fair & Very fair \\
				& (1) & (2) & (3) & (4) \\
				\midrule
				All & 28.2 & 51.6 & 17.3 & 2.9 \\
				\textbf{\hspace{-1em} Panel A. Gender} &   &   &   &  \\
				Female & 28.3 & 52.2 & 16.7 & 2.8 \\
				Male & 28.0 & 51.1 & 17.9 & 3.0 \\
				\textbf{\hspace{-1em} Panel B. Age group} &   &   &   &  \\
				15-24 & 25.2 & 52.0 & 19.7 & 3.1 \\
				25-40 & 28.5 & 51.3 & 17.2 & 3.0 \\
				41-64 & 29.5 & 51.7 & 16.0 & 2.8 \\
				65+ & 29.2 & 51.6 & 16.6 & 2.5 \\
				\textbf{\hspace{-1em} Panel C. Civil status} &   &   &   &  \\
				Married & 28.3 & 51.9 & 17.0 & 2.8 \\
				Not married & 27.9 & 51.4 & 17.7 & 3.1 \\
				\textbf{\hspace{-1em} Panel D. Religion} &   &   &   &  \\
				Catholic & 28.2 & 51.7 & 17.2 & 2.9 \\
				Not catholic & 28.0 & 51.5 & 17.5 & 3.0 \\
				\textbf{\hspace{-1em} Panel E. Education level} &   &   &   &  \\
				Less than Primary & 27.7 & 51.6 & 17.7 & 3.0 \\
				Complete Primary & 27.9 & 52.2 & 17.4 & 2.6 \\
				Complete Secondary & 29.1 & 53.2 & 15.0 & 2.7 \\
				Complete Tertiary & 29.0 & 50.8 & 17.1 & 3.1 \\
				\textbf{\hspace{-1em} Panel F. Type of employment} &   &   &   &  \\
				Employee & 28.3 & 51.5 & 17.2 & 2.9 \\
				Employer & 24.3 & 53.9 & 19.0 & 2.8 \\
				Self-employed & 28.0 & 51.4 & 17.5 & 3.1 \\
				Unemployed & 30.3 & 51.6 & 15.1 & 3.0 \\
				\textbf{\hspace{-1em} Panel E. Country} &   &   &   &  \\
				Argentina & 38.17 & 50.74 & 10.26 & 0.83 \\
				Bolivia & 18.01 & 56.13 & 23.39 & 2.48 \\
				Brazil & 31.95 & 53.71 & 12.85 & 1.49 \\
				Chile & 40.20 & 49.93 & 8.42 & 1.45 \\
				Colombia & 35.15 & 51.20 & 11.40 & 2.26 \\
				Costa Rica & 23.20 & 53.55 & 20.13 & 3.12 \\
				Dominican Rep. & 32.31 & 46.52 & 17.61 & 3.56 \\
				Ecuador & 21.45 & 47.46 & 27.58 & 3.51 \\
				El Salvador & 22.73 & 53.16 & 20.45 & 3.65 \\
				Guatemala & 28.29 & 51.34 & 16.70 & 3.66 \\
				Honduras & 28.87 & 53.42 & 14.33 & 3.38 \\
				Mexico & 32.15 & 49.75 & 15.32 & 2.78 \\
				Nicaragua & 18.69 & 51.88 & 24.33 & 5.11 \\
				Panama & 27.39 & 48.01 & 20.25 & 4.34 \\
				Paraguay & 38.31 & 48.80 & 10.95 & 1.93 \\
				Peru & 25.03 & 61.89 & 11.70 & 1.38 \\
				Uruguay & 18.22 & 57.51 & 22.64 & 1.64 \\
				Venezuela & 23.51 & 42.96 & 26.62 & 6.92 \\
				\midrule
			\end{tabular}
		\end{center}
		\begin{singlespace} \vspace{-.5cm}
			\noindent \justify \textbf{Note:} This table shows the fraction of individuals in our sample who perceive the income distribution as very unfair, unfair, fair, or very fair.
		\end{singlespace}
	}
\end{table}



\clearpage
\begin{table}[htpb!]{\footnotesize
		\begin{center}
			\caption{Logit regressions of unfairness perceptions (unfair) and individual characteristics} \label{unfair_logit}
			\newcommand\w{1.30}
			\begin{tabular}{l@{}lR{\w cm}@{}L{0.43cm}R{\w cm}@{}L{0.43cm}R{\w cm}@{}L{0.43cm}R{\w cm}@{}L{0.43cm}R{\w cm}@{}L{0.43cm}R{\w cm}@{}L{0.43cm}}
				\midrule
				&& 	\multicolumn{12}{c}{Dependent Variable: Believes income distribution is unfair or very unfair}  \\\cmidrule{3-14}
				%					&&          &&       &&          && Labor     &&  && Ideology, \\
				%					&& Baseline && Demog.   && Education && status  && Assets && Religion \\
				&& (1) && (2) && (3) && (4) && (5) && (6) \\
				\midrule
				\ExpandableInput{../results/unfair_logit}
				\midrule
			\end{tabular}
		\end{center}
		\begin{singlespace}  \vspace{-.5cm}
			\noindent \justify \textbf{Notes:} This table shows estimates of the relationship between an indicator that equals one for individuals who believe that the income distribution is unfair or very unfair and the Gini coefficient controlling for individuals' characteristics. Coefficients are estimated through Logit regressions and represent the marginal effects evaluated at the mean values of the rest of the variables. Observations are weighted by the individual's probability of being interviewed. All specifications include country and year fixed effects. $^{***}$, $^{**}$ and $^*$ denote significance at 10\%, 5\% and 1\% levels, respectively. Heteroskedasticity-robust standard errors clustered at the country-by-year level in parentheses.
		\end{singlespace}
	}
\end{table}


\clearpage
\begin{table}[htpb!]{\footnotesize
		\begin{center}
			\caption{Logit regressions of unfairness perceptions (very unfair) and different inequality indicators} \label{unfair_ineq_logit}
			\newcommand\w{1.50}
			\begin{tabular}{l@{}lR{\w cm}@{}L{0.43cm}R{\w cm}@{}L{0.43cm}R{\w cm}@{}L{0.43cm}R{\w cm}@{}L{0.43cm}R{\w cm}@{}L{0.43cm}}
				\midrule
				&& 	\multicolumn{10}{c}{Dependent Variable: Believes income distribution is very unfair}  \\\cmidrule{3-12}
				&& (1) && (2) && (3) && (4) && (5)  \\
				\midrule
				\ExpandableInput{../results/very_unfair_ineq_logit}
				\midrule
			\end{tabular}
		\end{center}
		\begin{singlespace}  \vspace{-.5cm}
			\noindent \justify \textbf{Notes:} This table shows estimates of the relationship between an indicator that equals one for individuals who believe income distribution is unfair or very unfair and several inequality indicators controlling for individuals' characteristics. Coefficients are estimated through Logit regressions and represent the marginal effects evaluated at the mean values of the rest of the variables. Observations are weighted by the individual's probability of being interviewed. All specifications include country and year fixed effects. $^{***}$, $^{**}$ and $^*$ denote significance at 10\%, 5\% and 1\% levels, respectively. Heteroskedasticity-robust standard errors clustered at the country-by-year level in parentheses.

		\end{singlespace}
	}
\end{table}




\begin{table}[htpb!]{\footnotesize
		\begin{center}
			\caption{OLS regressions of unfairness perceptions (very unfair) and individual characteristics} \label{very_unfair_lpm}
			\newcommand\w{1.30}
			\begin{tabular}{l@{}lR{\w cm}@{}L{0.43cm}R{\w cm}@{}L{0.43cm}R{\w cm}@{}L{0.43cm}R{\w cm}@{}L{0.43cm}R{\w cm}@{}L{0.43cm}R{\w cm}@{}L{0.43cm}}
				\midrule
				&& 	\multicolumn{12}{c}{Dependent Variable: Believes income distribution is very unfair}  \\\cmidrule{3-14}
				%					&&          &&       &&          && Labor     &&  && Ideology, \\
				%					&& Baseline && Demog.   && Education && status  && Assets && Religion \\
				&& (1) && (2) && (3) && (4) && (5) && (6) \\
				\midrule
				\ExpandableInput{../results/very_unfair_lpm}
				\midrule
			\end{tabular}
		\end{center}
		\begin{singlespace}  \vspace{-.5cm}
			\noindent \justify \textbf{Notes:} This table presents estimates of the correlation between a dummy variable that indicates if the individual believes income distribution is unfair or very unfair and the Gini coefficient controlling for individuals' characteristics. Coefficients are estimated through a linear probability model. Observations are weighted by the individual's probability of being interviewed. All specifications include country and year fixed effects. $^{***}$, $^{**}$ and $^*$ denote significance at 10\%, 5\% and 1\% levels, respectively. Heteroskedasticity-robust standard errors clustered at the country-by-year level in parentheses.
		\end{singlespace}
	}
\end{table}


%%%%%%%%%%%%%%%%%%%%%%%%%%%%%%%%%%%%%%%%%%%%%%%%%%%%%%%%%%%%%%%%%%%%%%%%%%%%%%%
% APPENDIX B: DATA APPENDIX
%%%%%%%%%%%%%%%%%%%%%%%%%%%%%%%%%%%%%%%%%%%%%%%%%%%%%%%%%%%%%%%%%%%%%%%%%%%%%%%

\clearpage
\section{Data Appendix} \label{sec_data}

\setcounter{table}{0}
\setcounter{figure}{0}
\renewcommand{\thetable}{B\arabic{table}}
\renewcommand{\thefigure}{B\arabic{figure}}

The figures presented in this paper are based on two harmonization projects, known as Latinobar\'ometro and SEDLAC (Socio-Economic Database for Latin America and the Caribbean). In this Appendix, we describe how we make both sources compatible.

Our perceptions data come from Latinobar\'ometro, which has conducted opinion surveys in 18 LA countries since the 1990s, interviewing about 1,200 individuals per country about individuals' socioeconomic background, and preferences towards political and social issues. Unfortunately, not all years contain questions about individuals' fairness perceptions. The survey was designed to be representative of the voting-age population at the national level (in most LA countries, individuals aged over 18). In Table \ref{tab-coverage} we show what percentage of the voting-age population is represented by the survey in each country for all the years in which the fairness question is available.



\begin{table}[htbp]{\footnotesize
		\begin{center}
			\caption{Coverage of each country's population in Latinobar\'ometro overtime (in \%)}\label{tab-coverage}
			\begin{tabular}{lrrrrrrrrr}
				\midrule
				& 1997 & 2001 & 2002 & 2007 & 2009 & 2010 & 2011 & 2013 & 2015 \\
				\midrule
				Argentina &         68  &         75  &         75  &       100  &       100  &       100  &       100  &       100  &       100  \\
				Bolivia &         32  &         52  &       100  &       100  &       100  &       100  &       100  &       100  &       100  \\
				Brazil &         12  &       100  &       100  &       100  &       100  &       100  &       100  &       100  &       100  \\
				Chile &         70  &         70  &         70  &       100  &       100  &       100  &       100  &       100  &       100  \\
				Colombia &         25  &         71  &         51  &       100  &       100  &       100  &       100  &       100  &       100  \\
				Costa Rica &       100  &       100  &       100  &       100  &       100  &       100  &       100  &       100  &       100  \\
				Dominican Republic & \multicolumn{1}{c}{ N/A } & \multicolumn{1}{c}{ N/A } & \multicolumn{1}{c}{ N/A } &       100  &       100  &       100  &       100  &       100  &       100  \\
				Ecuador &         97  &         97  &       100  &       100  &       100  &       100  &       100  &       100  &       100  \\
				El Salvador &         65  &       100  &       100  &       100  &       100  &       100  &       100  &       100  &       100  \\
				Guatemala &       100  &       100  &       100  &         97  &       100  &       100  &       100  &       100  &       100  \\
				Honduras &       100  &       100  &       100  &         98  &       100  &         99  &         99  &         99  &         99  \\
				Mexico &         93  &         88  &         95  &       100  &       100  &       100  &       100  &       100  &       100  \\
				Nicaragua &       100  &       100  &       100  &       100  &       100  &       100  &       100  &       100  &       100  \\
				Panama &       100  &       100  &       100  &         99  &         99  &         99  &         99  &         99  &         99  \\
				Paraguay &         46  &         46  &         46  &       100  &       100  &       100  &       100  &       100  &       100  \\
				Peru &         52  &         52  &       100  &       100  &       100  &       100  &       100  &       100  &       100  \\
				Uruguay &         80  &         80  &         80  &       100  &       100  &       100  &       100  &       100  &       100  \\
				Venezuela &       100  &       100  &       100  &       100  &         93  &       100  &       100  &       100  &       100  \\
				Weighted average &         68  &         86  &         91  &       100  &       100  &       100  &       100  &       100  &       100  \\
				\midrule
			\end{tabular}
		\end{center}
		%		\begin{singlespace}  \vspace{-.5cm}
			%			\noindent \justify \textbf{Notes:} This table presents the percentage of the voting-age population represented each year in Latinobar\'ometro overtime. The regional average is calculated by weighting each country's population. N/A means that Latinobar\'ometro did not conduct the opinion poll in that particular country-year.
			%		\end{singlespace}
	}
\end{table}


Since our goal is to analyze how unfairness perceptions evolved vis-\`a-vis changes in income inequality, we put a lot of effort into getting income inequality data for each data point for which we have perceptions data available. We made two partial fixes to increase the number of observations available (without pushing the data too much). First, we filled the data gaps using household surveys of relatively close years in which previously unused data were available (see Appendix Table \ref{tab-circa}). For instance, Chile conducts household surveys on average every two years. In 1997, there is perceptions data available, but no data on income inequality. Therefore, we use the inequality data from an adjacent year (1998). As noted previously, we only use data from close years if the data from the adjacent year correspond to a year in which the perceptions question was not asked (and therefore, inequality data are not needed in that year).

\begin{table}[htbp]{\footnotesize
		\begin{center}
			\caption{Circa years used to fill data gaps}\label{tab-circa}
			\begin{tabular}{lcc}
				\midrule
				Country & Year without household data & Data point used instead \\
				\midrule
				Chile & 1997 & 1998 \\
				Chile & 2001 & 2000 \\
				Chile & 2002 & 2003 \\
				Chile & 2007 & 2006 \\
				Colombia & 2007 & 2008 \\
				Ecuador & 2002 & 2003 \\
				El Salvador & 1997 & 1998 \\
				Guatemala & 2001 & 2000 \\
				Guatemala & 2015 & 2014 \\
				Mexico & 1997 & 1998 \\
				Mexico & 2001 & 2000 \\
				Mexico & 2007 & 2006 \\
				Mexico & 2009 & 2008 \\
				Mexico & 2011 & 2012 \\
				Mexico & 2015 & 2014 \\
				Nicaragua & 1997 & 1998 \\
				Nicaragua & 2007 & 2005 \\
				Nicaragua & 2015 & 2014 \\
				Venezuela & 2013 & 2012 \\
				\midrule
			\end{tabular}
		\end{center}
		\begin{singlespace}  \vspace{-.5cm}
			\noindent \justify %\textbf{Notes:}
		\end{singlespace}
	}
\end{table}

Our second partial fix involves interpolating inequality indicators for some years. For some countries, a few years had perceptions data available but no comparable household survey over time and no close year available. In this case, and to analyze the same set of countries every year, interpolation was applied to the inequality indicators (see Appendix Table \ref{tab-interp}).

\begin{table}[htbp]{\footnotesize
		\begin{center}
			\caption{Years in which inequality indicators were calculated with a linear interpolation}\label{tab-interp}
			\begin{tabular}{ll}
				\midrule
				Country & Years interpolated \\
				\midrule
				Argentina & 1997, 2001, and 2002 \\
				Bolivia & 2010 \\
				Brazil & 2010 \\
				Chile & 2010 \\
				Colombia & 1997 \\
				Costa Rica & 1997, 2001, 2002, 2007, and 2009 \\
				Ecuador & 1997, 2001 \\
				Guatemala & 1997, 2002, 2009, 2010, and 2013 \\
				Mexico & 2013 \\
				Nicaragua & 2002, 2010, 2011, and 2013 \\
				Panama & 1997, 2001, 2002, and 2007 \\
				Peru & 1997, 2001, 2002 \\
				Venezuela & 2015 \\
				\midrule
			\end{tabular}
		\end{center}
		\begin{singlespace}  \vspace{-.5cm}
			\noindent \justify %\textbf{Notes:}
		\end{singlespace}
	}
\end{table}


Overall, the years in which income inequality was calculated using linear interpolations represent a relatively small share of the total data points (17\% of total). The majority of our inequality data points (69\%) are calculated using a household survey from the same year in which the perceptions polls were conducted, while the remaining 14\% of our inequality indicators are calculated using household surveys from adjacent years. Table \ref{tab-summ-data} summarizes the data sources used in years perceptions data are available.

\begin{table}[htbp]{\footnotesize
		\begin{center}
			\caption{Summary of the data used in every country-year}\label{tab-summ-data}
			\begin{tabular}{r|r|llllllll}
				\multicolumn{1}{r}{} & \multicolumn{1}{r}{1997} & \multicolumn{1}{r}{2001} & \multicolumn{1}{r}{2002} & \multicolumn{1}{r}{2007} & \multicolumn{1}{r}{2009} & \multicolumn{1}{r}{2010} & \multicolumn{1}{r}{2011} & \multicolumn{1}{r}{2013} & \multicolumn{1}{r}{2015} \\
				\cmidrule{2-10}    \multicolumn{1}{l|}{Argentina} & \cellcolor[rgb]{ .608,  .733,  .349}\textcolor[rgb]{ .608,  .733,  .349}{3} & \multicolumn{1}{r|}{\cellcolor[rgb]{ .608,  .733,  .349}\textcolor[rgb]{ .608,  .733,  .349}{3}} & \multicolumn{1}{r|}{\cellcolor[rgb]{ .608,  .733,  .349}\textcolor[rgb]{ .608,  .733,  .349}{3}} & \multicolumn{1}{r|}{\cellcolor[rgb]{ .31,  .506,  .741}\textcolor[rgb]{ .31,  .506,  .741}{1}} & \multicolumn{1}{r|}{\cellcolor[rgb]{ .31,  .506,  .741}\textcolor[rgb]{ .31,  .506,  .741}{1}} & \multicolumn{1}{r|}{\cellcolor[rgb]{ .31,  .506,  .741}\textcolor[rgb]{ .31,  .506,  .741}{1}} & \multicolumn{1}{r|}{\cellcolor[rgb]{ .31,  .506,  .741}\textcolor[rgb]{ .31,  .506,  .741}{1}} & \multicolumn{1}{r|}{\cellcolor[rgb]{ .31,  .506,  .741}\textcolor[rgb]{ .31,  .506,  .741}{1}} & \multicolumn{1}{r|}{\cellcolor[rgb]{ .969,  .588,  .275}\textcolor[rgb]{ .969,  .588,  .275}{2}} \\
				\cmidrule{2-10}    \multicolumn{1}{l|}{Bolivia} & \cellcolor[rgb]{ .31,  .506,  .741}\textcolor[rgb]{ .31,  .506,  .741}{1} & \multicolumn{1}{r|}{\cellcolor[rgb]{ .31,  .506,  .741}\textcolor[rgb]{ .31,  .506,  .741}{1}} & \multicolumn{1}{r|}{\cellcolor[rgb]{ .31,  .506,  .741}\textcolor[rgb]{ .31,  .506,  .741}{1}} & \multicolumn{1}{r|}{\cellcolor[rgb]{ .31,  .506,  .741}\textcolor[rgb]{ .31,  .506,  .741}{1}} & \multicolumn{1}{r|}{\cellcolor[rgb]{ .31,  .506,  .741}\textcolor[rgb]{ .31,  .506,  .741}{1}} & \multicolumn{1}{r|}{\cellcolor[rgb]{ .608,  .733,  .349}\textcolor[rgb]{ .608,  .733,  .349}{3}} & \multicolumn{1}{r|}{\cellcolor[rgb]{ .31,  .506,  .741}\textcolor[rgb]{ .31,  .506,  .741}{1}} & \multicolumn{1}{r|}{\cellcolor[rgb]{ .31,  .506,  .741}\textcolor[rgb]{ .31,  .506,  .741}{1}} & \multicolumn{1}{r|}{\cellcolor[rgb]{ .31,  .506,  .741}\textcolor[rgb]{ .31,  .506,  .741}{1}} \\
				\cmidrule{2-10}    \multicolumn{1}{l|}{Brazil} & \cellcolor[rgb]{ .31,  .506,  .741}\textcolor[rgb]{ .31,  .506,  .741}{1} & \multicolumn{1}{r|}{\cellcolor[rgb]{ .31,  .506,  .741}\textcolor[rgb]{ .31,  .506,  .741}{1}} & \multicolumn{1}{r|}{\cellcolor[rgb]{ .31,  .506,  .741}\textcolor[rgb]{ .31,  .506,  .741}{1}} & \multicolumn{1}{r|}{\cellcolor[rgb]{ .31,  .506,  .741}\textcolor[rgb]{ .31,  .506,  .741}{1}} & \multicolumn{1}{r|}{\cellcolor[rgb]{ .31,  .506,  .741}\textcolor[rgb]{ .31,  .506,  .741}{1}} & \multicolumn{1}{r|}{\cellcolor[rgb]{ .608,  .733,  .349}\textcolor[rgb]{ .608,  .733,  .349}{3}} & \multicolumn{1}{r|}{\cellcolor[rgb]{ .31,  .506,  .741}\textcolor[rgb]{ .31,  .506,  .741}{1}} & \multicolumn{1}{r|}{\cellcolor[rgb]{ .31,  .506,  .741}\textcolor[rgb]{ .31,  .506,  .741}{1}} & \multicolumn{1}{r|}{\cellcolor[rgb]{ .31,  .506,  .741}\textcolor[rgb]{ .31,  .506,  .741}{1}} \\
				\cmidrule{2-10}    \multicolumn{1}{l|}{Chile} & \cellcolor[rgb]{ .969,  .588,  .275}\textcolor[rgb]{ .969,  .588,  .275}{2} & \multicolumn{1}{r|}{\cellcolor[rgb]{ .969,  .588,  .275}\textcolor[rgb]{ .969,  .588,  .275}{2}} & \multicolumn{1}{r|}{\cellcolor[rgb]{ .969,  .588,  .275}\textcolor[rgb]{ .969,  .588,  .275}{2}} & \multicolumn{1}{r|}{\cellcolor[rgb]{ .969,  .588,  .275}\textcolor[rgb]{ .969,  .588,  .275}{2}} & \multicolumn{1}{r|}{\cellcolor[rgb]{ .31,  .506,  .741}\textcolor[rgb]{ .31,  .506,  .741}{1}} & \multicolumn{1}{r|}{\cellcolor[rgb]{ .608,  .733,  .349}\textcolor[rgb]{ .608,  .733,  .349}{3}} & \multicolumn{1}{r|}{\cellcolor[rgb]{ .31,  .506,  .741}\textcolor[rgb]{ .31,  .506,  .741}{1}} & \multicolumn{1}{r|}{\cellcolor[rgb]{ .31,  .506,  .741}\textcolor[rgb]{ .31,  .506,  .741}{1}} & \multicolumn{1}{r|}{\cellcolor[rgb]{ .31,  .506,  .741}\textcolor[rgb]{ .31,  .506,  .741}{1}} \\
				\cmidrule{2-10}    \multicolumn{1}{l|}{Colombia} & \cellcolor[rgb]{ .608,  .733,  .349}\textcolor[rgb]{ .608,  .733,  .349}{3} & \multicolumn{1}{r|}{\cellcolor[rgb]{ .31,  .506,  .741}\textcolor[rgb]{ .31,  .506,  .741}{1}} & \multicolumn{1}{r|}{\cellcolor[rgb]{ .31,  .506,  .741}\textcolor[rgb]{ .31,  .506,  .741}{1}} & \multicolumn{1}{r|}{\cellcolor[rgb]{ .969,  .588,  .275}\textcolor[rgb]{ .969,  .588,  .275}{2}} & \multicolumn{1}{r|}{\cellcolor[rgb]{ .31,  .506,  .741}\textcolor[rgb]{ .31,  .506,  .741}{1}} & \multicolumn{1}{r|}{\cellcolor[rgb]{ .31,  .506,  .741}\textcolor[rgb]{ .31,  .506,  .741}{1}} & \multicolumn{1}{r|}{\cellcolor[rgb]{ .31,  .506,  .741}\textcolor[rgb]{ .31,  .506,  .741}{1}} & \multicolumn{1}{r|}{\cellcolor[rgb]{ .31,  .506,  .741}\textcolor[rgb]{ .31,  .506,  .741}{1}} & \multicolumn{1}{r|}{\cellcolor[rgb]{ .31,  .506,  .741}\textcolor[rgb]{ .31,  .506,  .741}{1}} \\
				\cmidrule{2-10}    \multicolumn{1}{l|}{Costa Rica} & \cellcolor[rgb]{ .608,  .733,  .349}\textcolor[rgb]{ .608,  .733,  .349}{3} & \multicolumn{1}{r|}{\cellcolor[rgb]{ .608,  .733,  .349}\textcolor[rgb]{ .608,  .733,  .349}{3}} & \multicolumn{1}{r|}{\cellcolor[rgb]{ .608,  .733,  .349}\textcolor[rgb]{ .608,  .733,  .349}{3}} & \multicolumn{1}{r|}{\cellcolor[rgb]{ .608,  .733,  .349}\textcolor[rgb]{ .608,  .733,  .349}{3}} & \multicolumn{1}{r|}{\cellcolor[rgb]{ .608,  .733,  .349}\textcolor[rgb]{ .608,  .733,  .349}{3}} & \multicolumn{1}{r|}{\cellcolor[rgb]{ .31,  .506,  .741}\textcolor[rgb]{ .31,  .506,  .741}{1}} & \multicolumn{1}{r|}{\cellcolor[rgb]{ .31,  .506,  .741}\textcolor[rgb]{ .31,  .506,  .741}{1}} & \multicolumn{1}{r|}{\cellcolor[rgb]{ .31,  .506,  .741}\textcolor[rgb]{ .31,  .506,  .741}{1}} & \multicolumn{1}{r|}{\cellcolor[rgb]{ .31,  .506,  .741}\textcolor[rgb]{ .31,  .506,  .741}{1}} \\
				\cmidrule{2-10}    \multicolumn{1}{l|}{Dominican Rep.} & \cellcolor[rgb]{ .749,  .749,  .749}\textcolor[rgb]{ .749,  .749,  .749}{0} & \multicolumn{1}{r|}{\cellcolor[rgb]{ .31,  .506,  .741}\textcolor[rgb]{ .31,  .506,  .741}{1}} & \multicolumn{1}{r|}{\cellcolor[rgb]{ .31,  .506,  .741}\textcolor[rgb]{ .31,  .506,  .741}{1}} & \multicolumn{1}{r|}{\cellcolor[rgb]{ .31,  .506,  .741}\textcolor[rgb]{ .31,  .506,  .741}{1}} & \multicolumn{1}{r|}{\cellcolor[rgb]{ .31,  .506,  .741}\textcolor[rgb]{ .31,  .506,  .741}{1}} & \multicolumn{1}{r|}{\cellcolor[rgb]{ .31,  .506,  .741}\textcolor[rgb]{ .31,  .506,  .741}{1}} & \multicolumn{1}{r|}{\cellcolor[rgb]{ .31,  .506,  .741}\textcolor[rgb]{ .31,  .506,  .741}{1}} & \multicolumn{1}{r|}{\cellcolor[rgb]{ .31,  .506,  .741}\textcolor[rgb]{ .31,  .506,  .741}{1}} & \multicolumn{1}{r|}{\cellcolor[rgb]{ .31,  .506,  .741}\textcolor[rgb]{ .31,  .506,  .741}{1}} \\
				\cmidrule{2-10}    \multicolumn{1}{l|}{Ecuador} & \cellcolor[rgb]{ .608,  .733,  .349}\textcolor[rgb]{ .608,  .733,  .349}{3} & \multicolumn{1}{r|}{\cellcolor[rgb]{ .608,  .733,  .349}\textcolor[rgb]{ .608,  .733,  .349}{3}} & \multicolumn{1}{r|}{\cellcolor[rgb]{ .969,  .588,  .275}\textcolor[rgb]{ .969,  .588,  .275}{2}} & \multicolumn{1}{r|}{\cellcolor[rgb]{ .31,  .506,  .741}\textcolor[rgb]{ .31,  .506,  .741}{1}} & \multicolumn{1}{r|}{\cellcolor[rgb]{ .31,  .506,  .741}\textcolor[rgb]{ .31,  .506,  .741}{1}} & \multicolumn{1}{r|}{\cellcolor[rgb]{ .31,  .506,  .741}\textcolor[rgb]{ .31,  .506,  .741}{1}} & \multicolumn{1}{r|}{\cellcolor[rgb]{ .31,  .506,  .741}\textcolor[rgb]{ .31,  .506,  .741}{1}} & \multicolumn{1}{r|}{\cellcolor[rgb]{ .31,  .506,  .741}\textcolor[rgb]{ .31,  .506,  .741}{1}} & \multicolumn{1}{r|}{\cellcolor[rgb]{ .31,  .506,  .741}\textcolor[rgb]{ .31,  .506,  .741}{1}} \\
				\cmidrule{2-10}    \multicolumn{1}{l|}{El Salvador} & \cellcolor[rgb]{ .31,  .506,  .741}\textcolor[rgb]{ .31,  .506,  .741}{1} & \multicolumn{1}{r|}{\cellcolor[rgb]{ .31,  .506,  .741}\textcolor[rgb]{ .31,  .506,  .741}{1}} & \multicolumn{1}{r|}{\cellcolor[rgb]{ .31,  .506,  .741}\textcolor[rgb]{ .31,  .506,  .741}{1}} & \multicolumn{1}{r|}{\cellcolor[rgb]{ .31,  .506,  .741}\textcolor[rgb]{ .31,  .506,  .741}{1}} & \multicolumn{1}{r|}{\cellcolor[rgb]{ .31,  .506,  .741}\textcolor[rgb]{ .31,  .506,  .741}{1}} & \multicolumn{1}{r|}{\cellcolor[rgb]{ .31,  .506,  .741}\textcolor[rgb]{ .31,  .506,  .741}{1}} & \multicolumn{1}{r|}{\cellcolor[rgb]{ .31,  .506,  .741}\textcolor[rgb]{ .31,  .506,  .741}{1}} & \multicolumn{1}{r|}{\cellcolor[rgb]{ .31,  .506,  .741}\textcolor[rgb]{ .31,  .506,  .741}{1}} & \multicolumn{1}{r|}{\cellcolor[rgb]{ .31,  .506,  .741}\textcolor[rgb]{ .31,  .506,  .741}{1}} \\
				\cmidrule{2-10}    \multicolumn{1}{l|}{Guatemala} & \cellcolor[rgb]{ .969,  .588,  .275}\textcolor[rgb]{ .969,  .588,  .275}{2} & \multicolumn{1}{r|}{\cellcolor[rgb]{ .969,  .588,  .275}\textcolor[rgb]{ .969,  .588,  .275}{2}} & \multicolumn{1}{r|}{\cellcolor[rgb]{ .608,  .733,  .349}\textcolor[rgb]{ .608,  .733,  .349}{3}} & \multicolumn{1}{r|}{\cellcolor[rgb]{ .969,  .588,  .275}\textcolor[rgb]{ .969,  .588,  .275}{2}} & \multicolumn{1}{r|}{\cellcolor[rgb]{ .608,  .733,  .349}\textcolor[rgb]{ .608,  .733,  .349}{3}} & \multicolumn{1}{r|}{\cellcolor[rgb]{ .608,  .733,  .349}\textcolor[rgb]{ .608,  .733,  .349}{3}} & \multicolumn{1}{r|}{\cellcolor[rgb]{ .31,  .506,  .741}\textcolor[rgb]{ .31,  .506,  .741}{1}} & \multicolumn{1}{r|}{\cellcolor[rgb]{ .608,  .733,  .349}\textcolor[rgb]{ .608,  .733,  .349}{3}} & \multicolumn{1}{r|}{\cellcolor[rgb]{ .969,  .588,  .275}\textcolor[rgb]{ .969,  .588,  .275}{2}} \\
				\cmidrule{2-10}    \multicolumn{1}{l|}{Honduras} & \cellcolor[rgb]{ .31,  .506,  .741}\textcolor[rgb]{ .31,  .506,  .741}{1} & \multicolumn{1}{r|}{\cellcolor[rgb]{ .31,  .506,  .741}\textcolor[rgb]{ .31,  .506,  .741}{1}} & \multicolumn{1}{r|}{\cellcolor[rgb]{ .31,  .506,  .741}\textcolor[rgb]{ .31,  .506,  .741}{1}} & \multicolumn{1}{r|}{\cellcolor[rgb]{ .31,  .506,  .741}\textcolor[rgb]{ .31,  .506,  .741}{1}} & \multicolumn{1}{r|}{\cellcolor[rgb]{ .31,  .506,  .741}\textcolor[rgb]{ .31,  .506,  .741}{1}} & \multicolumn{1}{r|}{\cellcolor[rgb]{ .31,  .506,  .741}\textcolor[rgb]{ .31,  .506,  .741}{1}} & \multicolumn{1}{r|}{\cellcolor[rgb]{ .31,  .506,  .741}\textcolor[rgb]{ .31,  .506,  .741}{1}} & \multicolumn{1}{r|}{\cellcolor[rgb]{ .31,  .506,  .741}\textcolor[rgb]{ .31,  .506,  .741}{1}} & \multicolumn{1}{r|}{\cellcolor[rgb]{ .31,  .506,  .741}\textcolor[rgb]{ .31,  .506,  .741}{1}} \\
				\cmidrule{2-10}    \multicolumn{1}{l|}{Mexico} & \cellcolor[rgb]{ .969,  .588,  .275}\textcolor[rgb]{ .969,  .588,  .275}{2} & \multicolumn{1}{r|}{\cellcolor[rgb]{ .969,  .588,  .275}\textcolor[rgb]{ .969,  .588,  .275}{2}} & \multicolumn{1}{r|}{\cellcolor[rgb]{ .31,  .506,  .741}\textcolor[rgb]{ .31,  .506,  .741}{1}} & \multicolumn{1}{r|}{\cellcolor[rgb]{ .969,  .588,  .275}\textcolor[rgb]{ .969,  .588,  .275}{2}} & \multicolumn{1}{r|}{\cellcolor[rgb]{ .969,  .588,  .275}\textcolor[rgb]{ .969,  .588,  .275}{2}} & \multicolumn{1}{r|}{\cellcolor[rgb]{ .31,  .506,  .741}\textcolor[rgb]{ .31,  .506,  .741}{1}} & \multicolumn{1}{r|}{\cellcolor[rgb]{ .969,  .588,  .275}\textcolor[rgb]{ .969,  .588,  .275}{2}} & \multicolumn{1}{r|}{\cellcolor[rgb]{ .608,  .733,  .349}\textcolor[rgb]{ .608,  .733,  .349}{3}} & \multicolumn{1}{r|}{\cellcolor[rgb]{ .969,  .588,  .275}\textcolor[rgb]{ .969,  .588,  .275}{2}} \\
				\cmidrule{2-10}    \multicolumn{1}{l|}{Nicaragua} & \cellcolor[rgb]{ .969,  .588,  .275}\textcolor[rgb]{ .969,  .588,  .275}{2} & \multicolumn{1}{r|}{\cellcolor[rgb]{ .31,  .506,  .741}\textcolor[rgb]{ .31,  .506,  .741}{1}} & \multicolumn{1}{r|}{\cellcolor[rgb]{ .608,  .733,  .349}\textcolor[rgb]{ .608,  .733,  .349}{3}} & \multicolumn{1}{r|}{\cellcolor[rgb]{ .969,  .588,  .275}\textcolor[rgb]{ .969,  .588,  .275}{2}} & \multicolumn{1}{r|}{\cellcolor[rgb]{ .31,  .506,  .741}\textcolor[rgb]{ .31,  .506,  .741}{1}} & \multicolumn{1}{r|}{\cellcolor[rgb]{ .608,  .733,  .349}\textcolor[rgb]{ .608,  .733,  .349}{3}} & \multicolumn{1}{r|}{\cellcolor[rgb]{ .608,  .733,  .349}\textcolor[rgb]{ .608,  .733,  .349}{3}} & \multicolumn{1}{r|}{\cellcolor[rgb]{ .608,  .733,  .349}\textcolor[rgb]{ .608,  .733,  .349}{3}} & \multicolumn{1}{r|}{\cellcolor[rgb]{ .969,  .588,  .275}\textcolor[rgb]{ .969,  .588,  .275}{2}} \\
				\cmidrule{2-10}    \multicolumn{1}{l|}{Panama} & \cellcolor[rgb]{ .608,  .733,  .349}\textcolor[rgb]{ .608,  .733,  .349}{3} & \multicolumn{1}{r|}{\cellcolor[rgb]{ .608,  .733,  .349}\textcolor[rgb]{ .608,  .733,  .349}{3}} & \multicolumn{1}{r|}{\cellcolor[rgb]{ .608,  .733,  .349}\textcolor[rgb]{ .608,  .733,  .349}{3}} & \multicolumn{1}{r|}{\cellcolor[rgb]{ .608,  .733,  .349}\textcolor[rgb]{ .608,  .733,  .349}{3}} & \multicolumn{1}{r|}{\cellcolor[rgb]{ .31,  .506,  .741}\textcolor[rgb]{ .31,  .506,  .741}{1}} & \multicolumn{1}{r|}{\cellcolor[rgb]{ .31,  .506,  .741}\textcolor[rgb]{ .31,  .506,  .741}{1}} & \multicolumn{1}{r|}{\cellcolor[rgb]{ .31,  .506,  .741}\textcolor[rgb]{ .31,  .506,  .741}{1}} & \multicolumn{1}{r|}{\cellcolor[rgb]{ .31,  .506,  .741}\textcolor[rgb]{ .31,  .506,  .741}{1}} & \multicolumn{1}{r|}{\cellcolor[rgb]{ .31,  .506,  .741}\textcolor[rgb]{ .31,  .506,  .741}{1}} \\
				\cmidrule{2-10}    \multicolumn{1}{l|}{Paraguay} & \cellcolor[rgb]{ .31,  .506,  .741}\textcolor[rgb]{ .31,  .506,  .741}{1} & \multicolumn{1}{r|}{\cellcolor[rgb]{ .31,  .506,  .741}\textcolor[rgb]{ .31,  .506,  .741}{1}} & \multicolumn{1}{r|}{\cellcolor[rgb]{ .31,  .506,  .741}\textcolor[rgb]{ .31,  .506,  .741}{1}} & \multicolumn{1}{r|}{\cellcolor[rgb]{ .31,  .506,  .741}\textcolor[rgb]{ .31,  .506,  .741}{1}} & \multicolumn{1}{r|}{\cellcolor[rgb]{ .31,  .506,  .741}\textcolor[rgb]{ .31,  .506,  .741}{1}} & \multicolumn{1}{r|}{\cellcolor[rgb]{ .31,  .506,  .741}\textcolor[rgb]{ .31,  .506,  .741}{1}} & \multicolumn{1}{r|}{\cellcolor[rgb]{ .31,  .506,  .741}\textcolor[rgb]{ .31,  .506,  .741}{1}} & \multicolumn{1}{r|}{\cellcolor[rgb]{ .31,  .506,  .741}\textcolor[rgb]{ .31,  .506,  .741}{1}} & \multicolumn{1}{r|}{\cellcolor[rgb]{ .31,  .506,  .741}\textcolor[rgb]{ .31,  .506,  .741}{1}} \\
				\cmidrule{2-10}    \multicolumn{1}{l|}{Peru} & \cellcolor[rgb]{ .969,  .588,  .275}\textcolor[rgb]{ .969,  .588,  .275}{2} & \multicolumn{1}{r|}{\cellcolor[rgb]{ .31,  .506,  .741}\textcolor[rgb]{ .31,  .506,  .741}{1}} & \multicolumn{1}{r|}{\cellcolor[rgb]{ .31,  .506,  .741}\textcolor[rgb]{ .31,  .506,  .741}{1}} & \multicolumn{1}{r|}{\cellcolor[rgb]{ .31,  .506,  .741}\textcolor[rgb]{ .31,  .506,  .741}{1}} & \multicolumn{1}{r|}{\cellcolor[rgb]{ .31,  .506,  .741}\textcolor[rgb]{ .31,  .506,  .741}{1}} & \multicolumn{1}{r|}{\cellcolor[rgb]{ .31,  .506,  .741}\textcolor[rgb]{ .31,  .506,  .741}{1}} & \multicolumn{1}{r|}{\cellcolor[rgb]{ .31,  .506,  .741}\textcolor[rgb]{ .31,  .506,  .741}{1}} & \multicolumn{1}{r|}{\cellcolor[rgb]{ .31,  .506,  .741}\textcolor[rgb]{ .31,  .506,  .741}{1}} & \multicolumn{1}{r|}{\cellcolor[rgb]{ .31,  .506,  .741}\textcolor[rgb]{ .31,  .506,  .741}{1}} \\
				\cmidrule{2-10}    \multicolumn{1}{l|}{Uruguay} & \cellcolor[rgb]{ .31,  .506,  .741}\textcolor[rgb]{ .31,  .506,  .741}{1} & \multicolumn{1}{r|}{\cellcolor[rgb]{ .31,  .506,  .741}\textcolor[rgb]{ .31,  .506,  .741}{1}} & \multicolumn{1}{r|}{\cellcolor[rgb]{ .31,  .506,  .741}\textcolor[rgb]{ .31,  .506,  .741}{1}} & \multicolumn{1}{r|}{\cellcolor[rgb]{ .31,  .506,  .741}\textcolor[rgb]{ .31,  .506,  .741}{1}} & \multicolumn{1}{r|}{\cellcolor[rgb]{ .31,  .506,  .741}\textcolor[rgb]{ .31,  .506,  .741}{1}} & \multicolumn{1}{r|}{\cellcolor[rgb]{ .31,  .506,  .741}\textcolor[rgb]{ .31,  .506,  .741}{1}} & \multicolumn{1}{r|}{\cellcolor[rgb]{ .31,  .506,  .741}\textcolor[rgb]{ .31,  .506,  .741}{1}} & \multicolumn{1}{r|}{\cellcolor[rgb]{ .31,  .506,  .741}\textcolor[rgb]{ .31,  .506,  .741}{1}} & \multicolumn{1}{r|}{\cellcolor[rgb]{ .31,  .506,  .741}\textcolor[rgb]{ .31,  .506,  .741}{1}} \\
				\cmidrule{2-10}    \multicolumn{1}{l|}{Venezuela} & \cellcolor[rgb]{ .31,  .506,  .741}\textcolor[rgb]{ .31,  .506,  .741}{1} & \multicolumn{1}{r|}{\cellcolor[rgb]{ .31,  .506,  .741}\textcolor[rgb]{ .31,  .506,  .741}{1}} & \multicolumn{1}{r|}{\cellcolor[rgb]{ .31,  .506,  .741}\textcolor[rgb]{ .31,  .506,  .741}{1}} & \multicolumn{1}{r|}{\cellcolor[rgb]{ .31,  .506,  .741}\textcolor[rgb]{ .31,  .506,  .741}{1}} & \multicolumn{1}{r|}{\cellcolor[rgb]{ .31,  .506,  .741}\textcolor[rgb]{ .31,  .506,  .741}{1}} & \multicolumn{1}{r|}{\cellcolor[rgb]{ .31,  .506,  .741}\textcolor[rgb]{ .31,  .506,  .741}{1}} & \multicolumn{1}{r|}{\cellcolor[rgb]{ .31,  .506,  .741}\textcolor[rgb]{ .31,  .506,  .741}{1}} & \multicolumn{1}{r|}{\cellcolor[rgb]{ .969,  .588,  .275}\textcolor[rgb]{ .969,  .588,  .275}{2}} & \multicolumn{1}{r|}{\cellcolor[rgb]{ .608,  .733,  .349}\textcolor[rgb]{ .608,  .733,  .349}{3}} \\
				\cmidrule{2-10}    \multicolumn{1}{r}{} & \multicolumn{1}{r}{} &   &   &   &   &   &   &   &  \\
				\cmidrule{2-2}      & \cellcolor[rgb]{ .31,  .506,  .741} & \multicolumn{8}{l}{Both perceptions and inequality data available} \\
				\cmidrule{2-2}      & \cellcolor[rgb]{ .969,  .588,  .275} & \multicolumn{8}{l}{Inequality was calculated with a close survey} \\
				\cmidrule{2-2}      & \cellcolor[rgb]{ .608,  .733,  .349} & \multicolumn{8}{l}{Inequality was calculated with a linear interpolation} \\
				\cmidrule{2-2}      & \cellcolor[rgb]{ .749,  .749,  .749} & \multicolumn{8}{l}{Latinobar\'ometro did not conduct survey in this year} \\
				\cmidrule{2-2}    \end{tabular}%
		\end{center}
		\begin{singlespace}  \vspace{-.5cm}
			\noindent \justify %\textbf{Notes:}
		\end{singlespace}
	}
\end{table}

\subsection{Imputation of Missing Values for the Regression Analysis}

Two of our individual-level variables (political ideology and religion) have many missing values in some country-years. To deal with this in our regressions, we imputed the average value of each variable to individuals with a missing value. In those cases, we included in the regression a dummy that takes the value one if the value of the variable was imputed and zero otherwise. The results are similar if we do not impute the values, but the sample size of the regressions is smaller.

\subsection{Comparison between Latinobar\'ometro's and SEDLAC's samples} \label{app_lb_sedlac}

To assess whether there are systematic differences between Latinobar\'ometro's sample and the household surveys' sample, in Appendix Table \ref{tab-lat-sedlac} we compare a set of variables available in both datasets during 2013. To ensure comparability across databases, we restrict the calculations to individuals over age 18 and countries with data available in both harmonization projects.

In general, the samples are similar in observable characteristics. For instance, the average age in Latinobar\'ometro's 2013 sample is 40.6 years, while in SEDLAC it is 42.7 years. Similarly, the percentage of males is 48.9\% in Latinobar\'ometro and 47.6\% in SEDLAC. The main difference arises from educational attainment. On average, the SEDLAC sample is more educated: 46.1\% of the population has secondary education or more, while this figure is 38.8\% in Latinobar\'ometro.


% Comparison of descriptive statistics in Latinobar\'ometro and SEDLAC, 2013
\begin{table}[H]{\scriptsize
		\begin{center}
			\caption{Descriptive statistics in Latinobar\'ometro and SEDLAC, 2013 (selected countries)} \label{tab-lat-sedlac}
			\begin{tabular}{lcccccccc}
				\midrule
				& \multicolumn{2}{c}{Mean} &   & \multicolumn{2}{c}{Standard Dev.} &   & \multicolumn{2}{c}{Observations} \\
				\cmidrule{2-3}\cmidrule{5-6}\cmidrule{8-9}      &  Latinob.  &  SEDLAC  &   &  Latinob.  &  SEDLAC  &   &  Latinob.  &  SEDLAC  \\
				& (1) & (2) &   & (3) & (4) &   & (5) & (6) \\
				\midrule
				\textbf{\hspace{-1em} Panel A. Sociodemographic} &   &   &   &   &   &   &   &  \\
				Age & 40.59 & 42.68 &   & 16.43 & 17.25 &   &      14,855  &   1,004,894  \\
				Male (\%) & 48.97 & 47.63 &   & 0.50 & 0.50 &   &      14,855  &   1,004,894  \\
				Married or civil union (\%) & 56.77 & 36.41 &   & 0.50 & 0.48 &   &      14,804  &      915,117  \\
				\multicolumn{3}{l}{\textbf{\hspace{-1em} Panel B. Education and Labor market}} &   &   &   &   &   &  \\
				Literate (\%) & 91.18 & 92.17 &   & 0.28 & 0.27 &   &      14,855  &   1,004,744  \\
				Secondary education or more (\%) & 38.83 & 46.11 &   & 0.49 & 0.50 &   &      14,855  &   1,001,672  \\
				Economically active (\%) & 65.14 & 68.66 &   & 0.48 & 0.46 &   &      14,855  &   1,004,718  \\
				Unemployed (\%) & 5.78 & 4.08 &   & 0.23 & 0.20 &   &      14,855  &   1,004,718  \\
				\textbf{\hspace{-1em} Panel C. Assets and Services} &   &   &   &   &   &   &   &  \\
				Access to a sewerage (\%) & 68.76 & 63.41 &   & 0.46 & 0.48 &   &      13,799  &      975,726  \\
				Car (\%) & 26.37 & 21.09 &   & 0.44 & 0.41 &   &      11,612  &      643,350  \\
				Computer (\%) & 46.55 & 47.82 &   & 0.50 & 0.50 &   &      12,747  &      894,003  \\
				Fridge (\%) & 82.76 & 88.89 &   & 0.38 & 0.31 &   &      12,763  &      894,003  \\
				Homeowner (\%) & 74.09 & 69.64 &   & 0.44 & 0.46 &   &      14,761  &   1,003,306  \\
				Mobile (\%) & 86.91 & 91.78 &   & 0.34 & 0.27 &   &      12,754  &      896,079  \\
				Washing machine (\%) & 60.49 & 56.88 &   & 0.49 & 0.50 &   &      11,816  &      848,350  \\
				Landline (\%) & 40.22 & 39.47 &   & 0.49 & 0.49 &   &      12,736  &      896,425  \\
				\midrule
			\end{tabular}
		\end{center}
		\begin{singlespace}
			\noindent \justify \textbf{Note:} This table compares the observable characteristics of individuals in Latinobar\'ometro and SEDLAC. Summary statistics were calculated on a restricted sample (individuals aged over 18) to ensure comparability between both datasets, pooling data from 14 countries in 2013: Argentina, Bolivia, Brazil, Chile, Colombia, Costa Rica, Dominican Republic, Ecuador, El Salvador, Honduras, Panama, Peru, Paraguay, and Uruguay.
		\end{singlespace}

	}
\end{table}


%%%%%%%%%%%%%%%%%%%%%%%%%%%%%%%%%%%%%%%%%%%%%%%%%%%%%%%%%%%%%%%%%%%%%%%%%%%%%%%
% APPENDIX C: OAXACA-BLINDER DECOMPOSITION
%%%%%%%%%%%%%%%%%%%%%%%%%%%%%%%%%%%%%%%%%%%%%%%%%%%%%%%%%%%%%%%%%%%%%%%%%%%%%%%

\clearpage
\section{The Oaxaca-Blinder Decomposition} \label{sec_oaxaca}

\setcounter{table}{0}
\setcounter{figure}{0}
\setcounter{equation}{0}
\renewcommand{\thetable}{C\arabic{table}}
\renewcommand{\thefigure}{C\arabic{figure}}
\renewcommand{\theequation}{C\arabic{equation}}

The starting point to decompose changes in unfairness perceptions between 2002 and 2013 is the following equation:
%
\begin{align}
	\text{Unfair}_{ict} = \beta_t X_{ict} + \gamma_t \text{Gini}_{ct} + \varepsilon_{ict} \quad \text{for} \quad t \in \{2002, 2013\},
\end{align}
%
where $t$ indicates the year in which perceptions are elicited and $X_{ict}$ is a vector that contains individual-level controls. The fraction of individuals who perceive the income distribution as unfair in year $t$ can be calculated as
%
\begin{align}
	\overline{\text{Unfair}}_{t} = \hat{\beta}_{t} \bar{X}_{t} + \hat{\gamma}_{t} \overline{\text{Gini}}_{t}  \quad \text{for} \quad t \in \{2002, 2013\},
\end{align}
%
where $\bar{X}_t$ is a vector of the average values of the explanatory variables in year $t$, and $\hat{\beta}$ the vector of OLS-estimated coefficients. The change in unfairness beliefs between 2013 and 2002 is given by
%
\begin{align} \label{eq-diff-fair}
	\underbrace{\overline{\text{Unfair}}_{2013} - \overline{\text{Unfair}}_{2013}}_{\equiv \Delta \text{Unfair}} = (\hat{\beta}_{2013} \bar{X}_{2013} + \hat{\gamma}_{2013} \overline{\text{Gini}}_{2013}) - (\hat{\beta}_{2002} \bar{X}_{2002} + \hat{\gamma}_{2002} \overline{\text{Gini}}_{2002})
\end{align}

Adding and subtracting $\hat{\beta}_{2002} \bar{X}_{2013} + \hat{\gamma}_{2002} \overline{\text{Gini}}_{2013}$ to equation \eqref{eq-diff-fair} yields
%
\begin{align} \label{eq_oaxaca}
	\Delta \text{Unfair} &=
	\underbrace{\hat{\beta}_{2002} (\bar{X}_{2013} - \bar{X}_{2002})}_{\equiv \Delta \text{Demog.}} +
	\underbrace{\hat{\gamma}_{2002} (\overline{\text{Gini}}_{2013} - \overline{\text{Gini}}_{2002})}_{\equiv \Delta \text{Gini}} \notag \\ &+ \underbrace{\bar{X}_{2013} (\hat{\beta}_{2013} - \hat{\beta}_{2002})+ \overline{\text{Gini}}_{2013} (\hat{\gamma}_{2013} - \hat{\gamma}_{2002})}_{\text{Residual}}
\end{align}
%

The first two terms of equation \eqref{eq_oaxaca} are usually known as the ``composition effect.'' These effects capture the difference between the average perceptions in 2002 and the counterfactual perceptions 2013 had the $\hat{\beta}$'s and $\hat{\gamma}$---i.e., the elasticity of perceptions to the different covariables---remained constant during the 2002--13 period. The first term captures differences in individual-level demographic variables that determine unfairness perceptions in the model (such as educational attainment, age, and employment status). The second term captures changes in aggregate trends in income inequality.

The third term of \eqref{eq_oaxaca} reflects the difference between the average fairness views in 2013 and the counterfactual fairness views in 2002 with the observable attributes of 2013. Thus, this component reflects changes in fairness views due to changes in the elasticity of the different covariables between both years. Since we cannot explain why the coefficients attached to each variable changed, this term is usually viewed as the ``unexplained'' part of the decomposition and treated as the residual of the decomposition.


% End standalone document
\ifdefined\standalonetrue
  \end{document}
\fi
